	
	\lecture[Costruzione di condizioni di stabilità di Bridgeland per superfici, prendendo come esempio il caso K3. Tecnica di tilting per cuori di $t$-strutture per la costruzione di opportuni cuori per le condizioni di stabilità.]{2025-10-21}
	
		\section{Condizioni di stabilità su superfici}
			
		\begin{warn*}
			Questa lezione è stata tenuta da \textbf{Filippo Papallo} (cioè da me).
		\end{warn*}
		
		Lo scopo di oggi è quello di costruire condizioni di stabilità di Bridgeland
		per una varietà $X$ di $\dim X > 1$. Non andremo molto lontano, in quanto
		le costruzioni note riguardano solo $\dim X = 2$ e, congetturalmente, $\dim X =3$;
		vedremo oggi il caso delle superfici.
		
		Nella scorsa lezione, Arvid ci ha parlato della varietà $Stab_{\Lambda,\nu}(C)$
		delle condizioni di stabilità nel caso in cui $C$ sia una curva proiettiva liscia:
		\begin{enumerate}[label=\roman*)]
			\item se $C = \PP^{1}_{\C}$, \textbf{So Okada} ha dimostrato che 
			$Stab_{\Lambda,\nu}(\PP^{1}_{\C}) \simeq \C^{2}$ e i muri sono
			determinati dalle equazioni
			\begin{align*}
				Z(\Oo(k)) = Z(\Oo(k-1)[1])\,, \quad \text{ con } k \in  \Z;
			\end{align*}
			quindi, a meno di autoequivalenze, abbiamo il solo muro $Z(\Oo) = Z(\Oo(-1)[1])$
			che separa la \emph{camera di Gieseker}, data da $Z(\Oo) < Z(\Oo(-1)[1])$,
			con la camera $Z(\Oo) > Z(\Oo(-1)[1])$ in cui
			gli unici oggetti semistabili sono $\Oo^{\oplus m}$ e $\Oo(-1)^{\oplus m'}[1]$.
			\item se $C$ è una curva di genere positivo, 
			allora $Stab_{\Lambda,\nu}(C) = \sigma_{C} \cdot \widetilde{\GL}_{2}^{+}(\R)$,
			dove $\sigma_{C} = (\Coh(C), -\rk + i \deg)$ è la condizione standard
			vista in \textbf{\Cref{ex:curve-bridgeland}}; in altri termini,
			su una curva tutte le condizioni di stabilità di Bridgeland sono
			equivalenti alla nozione di $\mu$-stabilità in senso classico.
		\end{enumerate}
		
		
		Abbiamo visto infatti che in queste situazioni $\Coh(C)$ 
		è un cuore accettabile per una possibile $\sigma \in Stab_{\Lambda,\nu}(C)$, 
		ma non appena $\dim X \ge 2$, per il \textbf{\Cref{lemma:Toda-num}} questo fallisce.
		Quindi, il problema non è che non abbiamo trovato la giusta carica centrale $Z$:
		il fatto è che dobbiamo cambiare categoria!
		
		\begin{oss}
			Questo non è vero nel caso di varietà non-proiettive:
			infatti, se $X$ è una superficie K3 con $\Pic(X) = 0$,
			allora $\Coh(X)$ può essere il cuore per una condizione di stabilità:
			infatti, basta prendere $z:=\alpha + i\beta$ con $\beta > 0$
			e per $\cat{v}(E) = (r,0,a)$ definire $Z(E) = -rz-a$.
			In questa situazione, \textbf{Huybrechts, Macrì} e \textbf{Stellari}
			hanno dimostrato che $Stab(X)$ è connessa e semplicemente connessa.
		\end{oss}
		
			I primi esempi di costruzione di condizioni su una superficie $S$ sono stati 
		trovati da Bridgeland stesso per le superfici K3 proiettive e abeliane, 
		nell'articolo \cite{B-stab-k3}, e poi generalizzato da \textbf{Arcara} e \textbf{Bertram}
		per superfici \emph{K-triviali}\footnote{...anche se Macrì afferma che la costruzione valga per \emph{tutte} le superfici proiettive lisce.}. Data $S$ una superficie K3 proiettiva,
		il suo scopo principale nell'articolo è quello
		di studiare il gruppo delle autoequivalenze $\Aut(\Db{S})$ e, per farlo,
		caratterizza esplicitamente le condizioni di stabilità $Stab^{\dagger}(S)$
		di una speciale componente connessa di $Stab(S)$ e come si comportano
		i vari pezzi della sua frontiera sotto l'azione di $\Aut(\Db{S})$.
		
		\begin{df}
			Una \textbf{superficie K3} è una superficie complessa $S$ compatta, 
			semplicemente connessa, con canonico banale $\Omega_{S}^{2} \simeq \Oo_{S}$.
			Più in generale, si può definire una \textbf{superficie K3 algebrica} su un campo $k$
			come una varietà completa, non singolare, di $\dim S = 2$ tale che
			\begin{align*}
				\Omega_{S|k}^{2} \simeq \Oo_{S} \,, \quad H^{1}(S;\Z) = 0\,.
			\end{align*}
		\end{df}
		
		Prima di entrare nei dettagli tecnici della costruzione, seguiamo l'idea di Bridgeland,
		come spiegata nell'introduzione del suo articolo.
		Data $S$ una K3, vogliamo studiarne le condizioni di stabilità \emph{numeriche},
		quindi consideriamo il suo \textbf{reticolo di Mukai algebrico}
		$$\Lambda := H^{*}_{alg}(S;\Z) = H^{0}(S;\Z) \oplus \operatorname{NS}(S) \oplus H^{4}(S;\Z) \,,$$
		che è isometrico al gruppo di Grothendieck numerico di $S$ 
		tramite il \textbf{carattere di Chern}
		\begin{align*}
			\cat{v} : K_{num}(S) &\longrightarrow H^{*}_{alg}(S;\Z)\,, \\
			\cat{v}([E]) := \ch(E) \smile \sqrt{\td(S)} &= (\rk(E), c_{1}(E), \rk(E) + \ch_{2}(E))\,,
		\end{align*}
		dato che $\sqrt{\td(S)} = (1,0,1)$. 
		Supponiamo ora che $Stab(S) = Stab_{(H^{*}_{alg}, \cat{v})}(S)$ sia non vuoto.
		Siccome siamo in spazi finito-dimensionali con prodotto scalare,
		possiamo riscrivere l'omeomorfismo locale $\Zz$ in \eqref{local-homeo} come
			\begin{equation*}
				\Zz : Stab(S) \longrightarrow (H^{*}_{alg}(S;\Z) \otimes \C)^{\vee} \,,
				\quad \Zz(\sigma) = \langle \cat{v}(\sigma), - \rangle\,, 
			\end{equation*}
		per un unico vettore 
		$\cat{v}(\sigma) \in \C \oplus (\operatorname{NS}(S) \otimes \C) \oplus \C$.
		All'interno di questo spazio vettoriale, andiamo a considerare solo i vettori
		\begin{equation}
			\Pp(S) := \Set{ \cat{w} \in \Lambda \otimes \C \, 
			| \, \Span_{\R}\{ \Re \cat{w}, \Im \cat{w} \} \text{ è un piano definito positivo in } \Lambda \otimes \R }\,.
		\end{equation}
		
		\begin{ex*}
			Per ogni $\omega \in Amp(S)_{\R}$ la classe 
			$e^{i\omega} = \left(1, i\omega, \frac{- \omega^{2}}{2}\right)$
			appartiene in $\Pp(S)$ poiché 
			\begin{align*}
				\langle \Re e^{i\omega}, \Im e^{i\omega} \rangle
				= \langle \left(1, 0, \frac{- \omega^{2}}{2}\right), 
				\left(0, \omega, 0 \right) \rangle = 0\,, \quad \text{ e } \quad
				(\Re e^{i\omega})^{2} = (\Im e^{i\omega})^{2} = \omega^{2} > 0\,.
			\end{align*}
			Inoltre, un conto veloce mostra che, per ogni $B \in \operatorname{NS}(S) \otimes \R$,
			la moltiplicazione $e^{-B} \smile -$ è una 
			trasformazione ortogonale di $H^{*}_{alg}(S;\Z) \otimes \R$, quindi tutti i vettori
			della forma
			$$e^{-B + i\omega} = \left(1, -B + i\omega, \frac{B^{2} - \omega^{2}}{2} - iB \cdot \omega \right)$$
			appartengono a $\Pp(S)$.
		\end{ex*}
		
		Per avere una descrizione più geometrica,
		notiamo che anche $\GL_{2}^{+}(\R)$ agisce liberamente su $\Pp(S)$,
		quindi $\Pp(S)$ può essere visto come un $\GL_{2}^{+}(\R)$-bundle sul quoziente.
		Una possibile sezione di questo fibrato è data da
		\begin{align*}
			\Qq(S) := \Set{ \cat{w} = (w_{0}, w_{2}, w_{4}) \in \Lambda \otimes \C \, 
			| \, \langle \cat{w}, \cat{w} \rangle = 0\,, \,
			\langle \cat{w}, \overline{\cat{w}} \rangle > 0\,, \, w_{0} = 1 }\,,
		\end{align*}
		che ci descrive l'insieme delle condizioni di stabilità 
		\emph{a meno dell'azione di $\GL_{2}^{+}(\R)$}. 
		Questo insieme $\Qq(X)$ si identifica con il dominio tubolare 
		$\Set{-B + i\omega \in \operatorname{NS}(S) \otimes \C \,|\, \omega^{2} > 0}$
		tramite la mappa esponenziale. Detta $\Pp^{+}(S)$ la componente connessa
		di $\Pp(S)$ che contiene le classi $e^{-B + i\omega}$, 
		con $\omega$ ampio come nell'\textbf{Esempio}, questa ci dà una descrizione
		esplicita di un pezzo di $Stab(S)$.
		
		\begin{thm}[\textbf{Bridgeland}]
			Sia $\Delta := \Set{\delta \in H^{*}_{alg}(S;\Z) \, | \, \delta^{2} = -2}$ l'insieme
			delle \emph{classi sferiche} e poniamo
			\begin{equation*}
				\Pp^{+}_{0}(S) := \Pp^{+} \setminus \bigcup_{\delta \in \Delta} \delta^{\perp}\,,
				\quad Stab^{\dagger}(S) := \Zz^{-1}(\Pp^{+}_{0}(S))\,.
			\end{equation*}
			La mappa $\Zz : Stab^{\dagger}(S) \to \Pp^{+}_{0}(S)$ è un rivestimento.
		\end{thm}
		
%		Le classi sferiche hanno un ruolo importante nella geometria delle K3
%		e indotte dalle classi di \textbf{(-2)-curve}.
%		La loro importanza, a livello derivato, è che inducono particolari
%		autoequivalenze, dette \textbf{twist sferici}, la cui composizione
%		può essere invisibile a livello coomologico (ma non a livello derivato).
%		

		In sintesi, questa linea di pensiero giustifica l'aspettativa che,
		al variare di $\omega \in Amp(S)_{\R}$ e $B \in \operatorname{NS}(S)_{\R}$,
		le funzioni della forma
		\begin{equation*}
			Z_{\omega, B} = \langle e^{-B + i \omega}, - \rangle
		\end{equation*}
		siano cariche centrali di una qualche condizione $\sigma_{\omega, B} \in Stab^{\dagger}(S)$.
		In effetti sarà così, ma la cosa complicata da capire, a priori, 
		è come costruire il cuore della condizione $\sigma_{\omega, B}$.
		
		
		\section{Coppie di torsione}
		
			Gli appassionati di algebra non-commutativa e categorie triangolate
			che ci sono a Padova, Verona, Praga e Murcia, conoscono
			un metodo, chiamato \textbf{tilting}, 
			per produrre $t$-strutture nuove da una già conosciuta.
			Questo metodo prevede di sfruttare la relazione tra $t$-strutture 
			e \textbf{teoria di torsione} che esistono per un cuore $\Aa$.
			
			\begin{df}
				Sia $\Aa$ una categoria abeliana. Una \textbf{teoria di torsione} $(\Ff, \Tt)$
				(o \textbf{coppia di torsione}) è una coppia di sottocategorie additive piene
				$\Ff, \Tt \subset \Aa$ tali che $\Hom(\Tt,\Ff) = 0$ e,
				per ogni oggetto $E \in \Aa$, esiste una (unica) sequenza esatta corta
				\begin{equation*}
					0 \longrightarrow T_{E} \longrightarrow E \longrightarrow F_{E} \longrightarrow 0\,,
					\quad \text{ con } T_{E} \in \Tt\,, \, F_{E} \in \Ff\,.
				\end{equation*}
			\end{df}
			
			\begin{ex}
				Data $X$ una varietà proiettiva liscia, 
				la coppia $\Tt = \Set{\text{fasci su $X$ di torsione}}$
				e $\Ff = \Set{\text{fasci su $X$ senza torsione}}$ definisce una teoria di torsione
				in $\Coh(X)$.
			\end{ex}
			
			
			\begin{thm}[\textbf{Tilting}]
				Sia $X$ una varietà proiettiva liscia. 
				Dato $\Aa \subset \Db{X}$ il cuore di una $t$-struttura,
				con coomologico $H_{\Aa}$,
				e $(\Ff, \Tt)$ una teoria di torsione su $\Aa$, allora la
				sottocategoria piena $\Aa^{\#}$, i cui oggetti sono i complessi $E \in \Db{X}$
				tali che
				\begin{equation}
					H^{\Aa}_{0}(E) \in \Tt\,, 
					\quad H_{\Aa}^{-1}(E) \in \Ff
					\quad \text{ e }  H_{\Aa}^{p}(E) = 0 \text{ per ogni } p \notin \{0,-1\}\,,
				\end{equation}
				è il cuore di una $t$-struttura. 
				Denoteremo il \textbf{cuore tiltato} $\Aa^{\#} = \langle \Ff[1], \Tt \rangle$.
%				\begin{proof}[Idea]
%					Infatti, ogni teoria di torsione definisce una $t$-struttura
%					\begin{align*}
%					\Dd^{\le 0} := \Set{ E \in \Db{\Aa} \, | \, \forall_{j > 0} \, H^{j}(E) = 0 
%					\text{ e } H^{0}(E) \in \Tt}\,, \\
%					\Dd^{\ge 0} := \Set{ E \in \Db{\Aa} \, | \, \forall_{j < -1} \, H^{j}(E) = 0 
%					\text{ e } H^{-1}(E) \in \Ff}\,.
%				\end{align*}
%				\end{proof}
			\end{thm}
			
			\begin{oss}
				Essenzialmente, in $\Aa$ ogni oggetto $E$ siede in un triangolo esatto
				$T \to E \to F$, con $(F,T) \in (\Ff, \Tt)$, e quindi rappresenta un elemento 
				di $\Ext_{\Aa}^{1}(T,F)$. Quando tiltiamo, 
				stiamo considerando estensioni nel senso opposto, cioè $E' \in \Db{X}$ che
				siedono in mezzo a un triangolo della forma $F_{E'}[1] \to E'\to T_{E'}$,
				con $F_{E'} \in \Ff$ e $T_{E'} \in \Tt$,
				quindi rappresentano degli elementi in $\Ext^{1}_{\Aa}(T_{E'},F_{E'}[1])
				= \Ext^{2}_{\Aa}(T_{E'},F_{E'})$.
			\end{oss}
			
			Tornando al caso di $X = S$ una superficie (qualsiasi),
			costruiamo una famiglia di teorie di torsione che dipendono da
			$\omega \in Amp_{\R}$ e $B \in \operatorname{NS}(X) \otimes \R$.
			Per prima cosa, twistando il carattere di Chern $\ch^{B} := e^{-B}\ch$,
			possiamo definire una versione \emph{twistata} della classica \textbf{slope-stability}
			considerando la pendenza
			\begin{equation*}
				\mu_{\omega,B}(E) := \frac{\omega \cdot \ch^{B}_{1}(E)}{\omega^{2} \rk(E)} =
				\frac{\omega \cdot c_{1}(E)}{\omega^{2} \rk(E)} - \frac{\omega \cdot B}{\omega^{2}}\,;
			\end{equation*}
			siccome il secondo termine non dipende da $E$, è chiaro che la 
			$\mu_{\omega,B}$-(semi)stabilità coincide con la classica $\mu_{\omega}$-(semi)stabilità.
			L'utilità di $B$ risiede nel poter definire diverse teorie di torsione:
			infatti, definiamo una coppia $(\Ff_{\omega,B}, \Tt_{\omega,B})$ `\emph{troncando}'
			la filtrazione di HN rispetto all'equazione $\mu_{\omega,B} = 0$,
			o più precisamente definiamo
			\begin{align*}
				\Tt_{\omega, B} :=& \Set{ E \in \Coh(X) \,| \, \forall_{q} \: \mu_{\omega}(\operatorname{HN}_{q}(E)) > \omega \cdot B } \\
				\Ff_{\omega, B} :=& \Set{E \in \Coh(X) \,| \, \forall_{q} \: \mu_{\omega}(\operatorname{HN}_{q}(E)) \le \omega \cdot B}\,.
			\end{align*}
			Allora $(\Ff_{\omega,B}, \Tt_{\omega,B})$ è una teoria di torsione
			per quanto sappiamo su $\mu_{\omega}$, e quindi possiamo tiltare 
			$\Coh(X)$ rispetto a questa coppia per ottenere il cuore
			\begin{equation}\label{Coh-wB}
				\Coh^{\omega,B}(S) := \langle \Ff_{\omega,B}[1], \Tt_{\omega,B} \rangle\,.
			\end{equation}
			
			Adesso che abbiamo un cuore e una possibile carica centrale,
			vogliamo convincerci che la coppia
			 $\sigma_{\omega,B} = (\Coh^{\omega,B}(S), Z_{\omega, B})$
			sia una condizione di stabilità di Bridgeland,
			usando il \textbf{\Cref{thm:equivalent-df-stab}}.
			Si dimostra che, per una \textbf{classe razionale} 
			$B \in \operatorname{NS}(S) \otimes \Q$,
			la categoria $\Coh^{\omega,B}(S)$ è noetheriana;
			quindi, alla luce del \textbf{\Cref{thm:discrete-Z}}, per costruire
			una condizione di stabilità nel senso della \textbf{\Cref{df:stab}}
			è sufficiente trovare una funzione
			di stabilità $Z_{\omega,B}$ con $\Im Z_{\omega,B}$ discreto in $\R$.
			Dall'introduzione, sappiamo che deve essere della forma 
			$Z_{\omega,B}([E]) = \langle \cat{v}(\omega,B), \cat{v}(E) \rangle$
			e un candidato appetibile è
			\begin{equation*}
				Z_{\omega,B}([E]) = \langle e^{-B+i\omega}, \cat{v}(E) \rangle
				= - \int_{X} e^{i\omega} \smile \ch^{B}(E)\,,
			\end{equation*}
			che esplicitamente, per $\cat{v} = (r, c, a) \in \Lambda$ appare come 
			\begin{equation}\label{ZwB-explicit}
				Z_{\omega,B}(\cat{v}) = \left( - a + c \cdot B + r\frac{\omega^{2} - B^{2}}{2} \right) + i \omega \cdot (c - rB)\,.
			\end{equation}
			
			\begin{prop}
				Per ogni $\omega \in Amp_{\R}$ e $B \in \operatorname{NS}(S) \otimes \Q$,
				la $Z_{\omega, B}$ è una funzione di stabilità su $\Coh^{\omega,B}(S)$.
%				\begin{proof}
%					Idea...
%				\end{proof}
			\end{prop}
			
			Per avere una condizione di stabilità di Bridgeland è necessario verificare, infine,
			la \textbf{proprietà del supporto}: equivalentemente, 
			è sufficiente trovare una forma quadratica 
			definita positiva sugli oggetti semistabili, ma questa è data\footnote{Non è proprio questo il discriminante che va bene per tutti gli $\omega,B$,
			 ma è quello che funziona per il piano $(\alpha, \beta)$ ed è già molto utile nelle applicazioni.} dal \textbf{discriminante twistato}
				\begin{equation}
					\Delta^{B}_{\omega}(E) := (\omega \cdot \ch_{1}^{B}(E))^{2} - 2 \omega^{2} \ch_{0}^{B}(E) \ch_{2}^{B}(E) \,,
				\end{equation}
			come conseguenza della \textbf{disuguaglianza di Bogomolov},
			che afferma $\Delta^{B}_{\omega}(E) \ge 0$ 
			per ogni $E$ fascio $\mu_{\omega}$-semistabile. 
			Inaspettatamente, la disuguaglianza vale anche per oggetti $Z_{\omega, B}$-semistabili, 
			dopo aver svelato il legame tra la stabilità di Bridgeland e la $\mu_{\omega}$-stabilità:
			
			\begin{lemma}[\textbf{Large Volume Limit}]\label{LVL}
				Un oggetto $E \in \Coh^{\omega,B}(S)$ è $\sigma_{\alpha \cdot \omega, B}$-semistabile
				per ogni $\alpha \gge 0$ se e solo se $E$ è lo shift
				di un fascio $\mu_{\omega,B}$-semistabile. In altre parole: per $\alpha > 0$ tanto
				grande, la Bridgeland stabilità coincide con la slope-stability twistata.
				\begin{proof}[Idea]
					La funzione $Z_{\omega, B}$ definisce la pendenza
					\begin{equation}
						\nu_{\omega,B}(E) = \frac{\ch_{2}^{B}(E)}{\omega \cdot \ch_{1}^{B}(E)}
						- \frac{\omega^{2} \rk(E)}{2\omega \cdot \ch_{1}^{B}(E)}\,.
					\end{equation}
					Ora, per ogni $\alpha > 0$, un oggetto è $\nu_{\alpha \cdot \omega,B}$-(semi)stabile
					se e solo se è $\frac{2}{\alpha}\nu_{\alpha \cdot \omega,B}$-(semi)stabile.
					Ma allora prendere $\alpha \gge 0$ significa considerare il limite
					\begin{align*}
						\lim_{\alpha \to +\infty} \frac{2}{\alpha} \nu_{\alpha \omega,B}(E)
						&= \lim_{\alpha \to +\infty} \frac{2 \ch_{2}^{B}(E)}{ \alpha^{2} \,\omega \cdot \ch_{1}^{B}(E)}
						- \frac{2 \alpha^{2} \omega^{2} \rk(E)}{2 \alpha^{2} \, \omega \cdot \ch_{1}^{B}(E)} \\ &=
						- \frac{\omega^{2} \rk(E)}{\omega \cdot \ch_{1}^{B}(E)} = - \frac{1}{\mu_{\omega, B}(E)}\,,
					\end{align*}
				quindi le disuguaglianze della semistabilità sono essenzialmente le stesse;
				la parte difficile è che bisogna fare attenzione 
				alla nozione di \emph{sottofascio} in $\Coh(S)$ e
				$\Coh^{\alpha \omega, B}(S)$, in quanto sono diverse.
				\end{proof}
			\end{lemma}
			
			\begin{rmk}[per Arvid]
				Non so perché si chiami Large Volume Limit.
			\end{rmk}
			
			
			Siccome $\Delta^{B}_{\omega}$ è una forma quadratica che testimonia 
			la proprietà del supporto, segue che 
			$\sigma_{\omega,B} := (\Coh^{\omega,B}(S), Z_{\omega, B})$
			è una condizione di stabilità di Bridgeland su $\Db{S}$.
			
			Per poter concludere che la mappa
			\begin{equation*}
				\sigma : Amp(S)_{\R} \times \operatorname{NS}(S)_{\R} \longrightarrow Stab^{\dagger}(S)\,, \quad (\omega, B) \mapsto \sigma_{\omega,B}\,,
			\end{equation*}
			sia un embedding continuo, è necessario verificare che anche le
			classi reali $B \in \operatorname{NS}(S)_{\R}$ danno condizioni di stabilità.
			Per il \textbf{\Cref{thm:defo} di deformazione}, siamo sicuri che
			anche dopo una piccola deformazione, $Z_{\omega, B}$ 
			continua a dare una carica centrale per una condizione 
			$\sigma_{\omega,B} = (\Aa, Z_{\omega,B}) \in Stab^{\dagger}(S)$;
			la cosa sorprendente è che questa funzione di stabilità continua a essere
			valida su un cuore $\Aa$ della forma $\Coh^{\omega,B}(S)$: infatti, queste
			categorie ammettono la seguente caratterizzazione.
			\begin{lemma}\label{lemma:heart-characterization}
				Sia $\sigma = (\Aa, Z_{\omega, B})$ una condizione di stabilità 
				che soddisfa la proprietà del supporto. 
				Tutti i fasci grattacielo $\C(s)$ sono oggetti $\sigma$-stabili di fase $1$ 
				(i.e. per ogni $s \in S$, $\mu_{\sigma}(\C(s)) = +\infty$) se e solo se
				$\Aa = \Coh^{\omega,B}(S)$.
			\end{lemma}
			Dato che un fascio grattacielo ha vettore $\cat{v}(\C(s)) = (0,0,1)$,
			la formula \eqref{ZwB-explicit} mostra che $Z_{\omega,B}(\C(s)) = -1$,
			quindi il \textbf{\Cref{lemma:heart-characterization}} vale.
			
			
