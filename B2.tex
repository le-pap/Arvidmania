	
		\lecture[.]{2025-10-07}
		
		\section{Condizioni di stabilità di Bridgeland}
		
			La scorsa volta abbiamo motivato l'interesse fisico dello studio della
			$\mu$-stabilità, in quanto legato alle metriche Hermite-Einstein;
			passando alla teoria delle stringhe, si fa qualcosa di leggermente diverso
			che coinvolge le categorie derivate, alla maniera di Bridgeland.
			Per definire le sue condizioni di stabilità, si possono adottare
			due punti di vista:
			\begin{itemize}
				\item[i)] usando i cuori di $t$-strutture, cioè definizione su categorie abeliane;
				spesso questo approccio viene detto \textbf{condizioni di stabilità di King};
				\item[ii)] quello originale di Bridgeland, in cui vengono definite per 
				categorie triangolate qualsiasi tramite la nozione di \textbf{slicing};
				la definizione è lampante, ma il formalismo è più brutto.
			\end{itemize}
			
			\subsection{i) Stabilità di King}			
			
			Sia $\Aa$ una categoria abeliana, e denotiamo con $K_{0}(\Aa)$ il suo 
			\textbf{gruppo di Grothendieck}.
			\begin{df}\label{df:Z}
				Una \textbf{funzione di stabilità} su $\Aa$ è una funzione $Z : K_{0}(\Aa) \to \C$
				tale che:
				\begin{itemize}
					\item $Z$ è un omomorfismo di gruppi;
					\item per ogni $E \in \Aa$, si ha $\Im Z([E]) \ge 0$;
					\item se $E \ne 0$ ha $\Im Z([E]) = 0$, allora $\Re ([E]) < 0$.
				\end{itemize}
			\end{df}
			Il formalismo \emph{fa schifo} perché, a colpo d'occhio, non è affatto chiaro
			che relazione abbiano queste nozioni con quella di stabilità di fibrati vettoriali
			che abbiamo visto nelle motivazioni. Il seguente esempio fa luce su questo fatto:
			
			\begin{ex}\label{ex:curve-bridgeland}
				Sia $C$ una curva proiettiva liscia, $\Aa = \Coh(C)$, allora $K_{0}(\Aa) = K_{0}(C)$.
				Poniamo
				\begin{equation}
					Z : K_{0}(C) \longrightarrow \C\,, \quad Z([E]) := -\deg(E) + i \rk(E)\,.
				\end{equation}
				Per additività di grado e rango, $Z$ è un omomorfismo di gruppi ben definito;
				inoltre $\Im Z = \rk \ge 0$ e se $\Im Z([E]) = \rk(E) = 0$,
				allora $E$ è di torsione, quindi ha grado positivo $\deg(E) > 0$,
				da cui $\Re ([E]) < 0$. Questo mostra che $Z$ è una funzione di stabilità.
			\end{ex}
			
			Perché rango e grado abbiano senso in questa combinazione all'interno
			di $\C$ è ancora motivato dalla fisica: nei termini della teoria delle stringhe,
			la funzione $Z$, spesso chiamata \textbf{funzione di carica centrale},
			rappresenta la carica di particelle sulle $D$-brane. Boh.
			
			\begin{df}
				Data $Z$ una funzione di stabilità su $\Aa$ e $E \in \Aa$, poniamo
				\begin{equation}
					r_{Z}(E) := \Im Z([E])\,, \quad d_{Z}(E) := -\Re Z([E])\,,
					\quad \mu_{Z}(E) := \frac{d_{Z}(E)}{r_{Z}(E)}\,,
				\end{equation}
				chiamate, rispettivamente, \textbf{$Z$-rango}, \textbf{$Z$-grado} e 
				\textbf{$Z$-pendenza} di $E$.
			\end{df}
			
			\begin{rmk}
				La presenza del ``$-$'' nella definizione della funzione di carica centrale
				serve perché, se $r_{Z}(E)=0$, allora $d_{Z}(E) > 0$, e quindi
				un fascio con rango nullo ha pendenza infinita (positiva) $\mu_{Z}(E) = +\infty$.
				Inoltre, le condizioni di \textbf{\Cref{df:Z}} permettono di definire $Z$
				per ogni oggetto non-zero della categoria abeliana $\Aa$,
				e $0 \in \Aa$ è l'\textbf{unico} oggetto per cui $Z(0) = 0$.
			\end{rmk}
			
			\begin{df}\label{df:Z-ss}
				Un oggetto $E \in \Aa$ è detto \textbf{$Z$-semistabile} se,
				per ogni sottoggetto $0 \ne F \subsetneq E$, si ha
				\begin{equation*}
					\mu_{Z}(F) \le \mu_{Z}(E)\,.
				\end{equation*}
			\end{df}
			
			C'è un grande \textbf{problema} quando da $C$ curva andiamo a considerare
			$X$ una varietà proiettiva liscia di dimensione $n \ge 2$:
			preso $H \in Amp(X)$ e posti $A = \Coh(X)$ e $K_{0}(\Aa) = K_{0}(X)$,
			uno sarebbe tentato di considerare la funzione:
			\begin{equation}
				Z_{H} : K_{0}(X) \longrightarrow \C\,, \quad
				Z_{H}([E]) := -(c_{1}(E) \cdot H^{n-1}) + i \rk(E) H^{n}\,.
			\end{equation}
			Se questa fosse una funzione di stabilità, allora $\mu_{Z} = \mu_{H}$ sarebbe
			la solita pendenza rispetto a $H$, 
			ma purtroppo non soddisfa le condizioni in \textbf{\Cref{df:Z}}:
			infatti, se $E$ è un fascio supportato almeno in codimensione $2$, allora
			$Z_{H}(E) = 0$...
			Però potrei aver scelto male la mia funzione $Z$. 
			In realtà, c'è un problema più serio che non permette di ottenere mai
			funzioni di stabilità su $\Coh(X)$ in dimensione $n > 1$.
			\begin{df}
				Una funzione di stabilità $Z : K_{0}(X) \to \C$ è detta \textbf{numerica}
				se esiste un omomorfismo di gruppi $Z': H^{2*}(X;\Q) \to \C$ tale che
				$Z = Z'\circ \ch$, cioè il seguente triangolo commuti:
				\begin{equation*}
					\begin{tikzcd}
					K_{0}(X) \ar[dr, "Z"'] \ar[rr,"\ch"] && H^{2*}(X;\Q) \ar[dl, "Z'"] \\
					& \C & \,.
					\end{tikzcd}
				\end{equation*}
			\end{df}
			\begin{ex}
				Se $X=C$ è una curva, allora $Z = -\deg + i \rk$ è una condizione numerica,
				infatti $\ch: K_{0}(C) \to H^{2*}(C;\Q) = H^{0}(C; \Q) \oplus H^{2}(C; \Q)$ è
				proprio 
				$$\ch(E) = (\rk(E), \deg(E))\,,$$
				quindi basta porre $Z'(\alpha,\beta) := -\beta + i \alpha$.
			\end{ex}
			
			\begin{lemma}[\textbf{Toda}]\label{lemma:Toda-num}
				Se $\dim X \ge 2$ non esistono funzioni di stabilità numeriche su $\Coh(X)$.
			\end{lemma}
			Essenzialmente, la ragione per cui si ha questo fallimento è che, in dimensione maggiore
			di $1$, il carattere di Chern scontrato con $H$ \textbf{non} è più lineare.
			
			In tutto questo stiamo ancora parlando di \emph{funzioni}, 
			e non di \emph{condizioni di stabilità}, e abbiamo già capito che è difficile in generale
			fabbricare le prime.
			
			\begin{df}\label{df:stab}
				Una funzione di stabilità $Z$ su $\Aa$ è detta \textbf{condizione di stabilità}
				se, per ogni oggetto $E \in \Aa$, esiste una \textbf{filtrazione di Harder-Narasimhan}
				per $E$, i.e. una filtrazione
				\begin{equation*}
		 			0 = E_{0} \subsetneq E_{1} \subsetneq \dots \subsetneq E_{l-1} \subsetneq E_{l} = E
		 		\end{equation*}
		 	tale che, per ogni $0 < i \le l$, ogni fattore $E_{i}/E_{i-1}$ sia $\mu_{Z}$-semistabile e 
		 	si abbia pendenza decrescence $\mu(E_{i}/E_{i-1}) > \mu(E_{i+1}/E_{i})$.
			\end{df}
			
			Dimostrare l'esistenza della filtrazione di HN, in generale, non è un compito
			semplice, ma se si pongono condizioni buone sulla $Z$ che andiamo a costruire,
			allora abbiamo speranza di ottenere la filtrazione di HN in maniera automatica.
			Ad esempio, se so già che la categoria $\Aa$ ha oggetti che ammettono filtrazioni
			infinite, allora ho poche speranze di ottenere una condizione di stabilità da una $Z$
			qualsiasi.
			\begin{df}
				Una categoria abeliana $\Aa$ è detta \textbf{noetheriana} se, per ogni $E \in \Aa$
				e ogni catena ascendente
				\begin{equation*}
		 			0 = E_{0} \subset E_{1} \subset \dots \subset E_{l} \subset  \dots \subset E
		 		\end{equation*}
		 		ammette un $n_{0} \in \N$ tale che $E_{n} = E_{n_{0}}$, per ogni $n \ge n_{0}$.
			\end{df}
			
			\begin{thm}\label{thm:discrete-Z}
				Se $\Aa$ è abeliana e noetheriana e $Z : K_{0}(X) \to \C$ è una funzione
				di stabilità tale che $r_{Z} : K_{0}(\Aa) \to \R$ ha immagine discreta,
				allora $Z$ è una condizione di stabilità.
			\end{thm}
			
			Quindi, in fin dei conti, è sufficiente fabbricare funzioni di stabilità.
			
			\subsection{ii) Stabilità di Bridgeland originale}
			
			Sia $\Dd$ una categoria triangolata.
			\begin{df}\label{df:t-struttura}
				Una \textbf{$t$-struttura} su $\Dd$ è una coppia $(\Dd^{\le 0}, \Dd^{\ge 0})$,
				con $\Dd^{\le 0}, \Dd^{\ge 0}$ due sottocategorie piene di $\Dd$,
				non necessariamente triangolate, tali che:
				\begin{itemize}
					\item se $A \in \Dd^{\le 0}$ e  $B \in \Dd^{\ge 0}$,
					allora $A[1] \in \Dd^{\le 0}$ e  $B[-1] \in \Dd^{\ge 0}$;
					
					\item se $A \in \Dd^{\le 0}$ e  $B \in \Dd^{\ge 0}$, allora per ogni $i>j$
					si ha $\Hom_{\Dd}(A[i],B[j]) =0$;
					
					\item per ogni $E \in \Dd$ esiste un triangolo esatto
					\begin{equation*}
					A \longrightarrow E \longrightarrow B \longrightarrow A[1]
					\end{equation*}
					con $A \in \Dd^{\le 0}$ e  $B \in \Dd^{\ge 0}$.
				\end{itemize}
			\end{df}
			
			\begin{ex}
				Per ogni categoria abeliana $\Aa$, esiste una $t$-struttura canonica 
				$(\Dd^{\le 0}, \Dd^{\ge 0})$ su $\Db{\Aa}$ definita da
				\begin{align*}
					\Dd^{\le 0} := \Set{ E \in \Db{\Aa} \, | \, \forall_{j > 0} \, H^{j}(E) = 0 }\,, \\
					\Dd^{\ge 0} := \Set{ E \in \Db{\Aa} \, | \, \forall_{j \le 0} \, H^{j}(E) = 0 }\,.
				\end{align*}
				Chiameremo questa la \textbf{$t$-struttura standard}.
			\end{ex}
			
			\begin{df}
				Una $t$-struttura $(\Dd^{\le 0}, \Dd^{\ge 0})$ su $\Dd$
				è \textbf{limitata} se, per ogni $E \in \Dd$, esistono $m,n \in \Z$
				tale che $E[n] \in \Dd^{\le 0}$ e $E[m] \in \Dd^{\ge 0}$.
			\end{df}
			
			\begin{thm}
				Data una $t$-struttura $(\Dd^{\le 0}, \Dd^{\ge 0})$ su una categoria triangolata 
				$\Dd$, la sottocategoria $\Dd^{\heartsuit} := \Dd^{\le 0} \cap \Dd^{\ge 0}$
				è una categoria abeliana, detta \textbf{cuore} della $t$-struttura.
			\end{thm}
			
			\begin{ex}
				Se $(\Dd^{\le 0}, \Dd^{\ge 0})$ è la $t$-struttura standard in $\Db{\Aa}$
				dell'esempio precedente, allora $\Dd^{\heartsuit}$ è equivalente a $\Aa$.
			\end{ex}
			
			\begin{thm}[\textbf{Bridgeland}]\label{thm:heart}
				Sia $\Dd$ una categoria triangolata, una sottocategoria $\Aa \subset \Dd$ abeliana
				è il cuore di una $t$-struttura se e solo se valgono le seguenti tre condizioni:
				\begin{itemize}
					\item[i)] se $A, B \in \Aa$, per ogni $i>j$
					si ha $\Hom_{\Dd}(A[i],B[j]) =0$;
					
					\item[ii)] per ogni $E \in \Dd$,
					 esistono interi $h_{1}, \dots, h_{m} \in \Z$,
					 oggetti $E_{1}, \dots, E_{m} \in \Dd$ e $A_{1}, \dots, A_{m} \in \Aa$
					 tali che si abbia il diagramma di triangoli esatti
					 \begin{equation*}
					 	\begin{tikzcd}
0 = E_{0} \arrow[r] & E_{1} \arrow[r] \arrow[d]               & E_{2} \arrow[r] \arrow[d]               & \dots \arrow[r] & E_{m-1} \arrow[r] \arrow[d]                 & E_{m} = E \arrow[d]                     \\
                    & {A_{1}[h_{1}]} \arrow[lu, "+1", dashed] & {A_{2}[h_{2}]} \arrow[lu, "+1", dashed] &                 & {A_{m-1}[h_{m-1}]} \arrow[lu, "+1", dashed] & {A_{m}[h_{m}]} \arrow[lu, "+1", dashed]
					 	\end{tikzcd}
					 \end{equation*}
					 che prende il ruolo della \textbf{filtrazione di Harder-Narasimhan}
					 di $E$ in $\Dd$.
				\end{itemize}
			\end{thm}
			
			Sia $\Dd$ una categoria triangolata. Diamo ora una nozione molto simile
			a quella appena enunciata nel \textbf{\Cref{thm:heart}}.
			\begin{df}\label{df:slicing}
				Uno \textbf{slicing} di $\Dd$ è una famiglia $\Pp = \Set{\Pp(\phi) | \phi \in \R}$ 
				di sottocategorie piene di $\Dd$ tali che:
				\begin{itemize}
					\item[i)] per ogni $\phi \in \R$ vale $\Pp(\phi)[1] = \Pp(\phi + 1)$;
					
					\item[ii)] per ogni coppia di numeri reali $\phi_{1} > \phi_{2}$
					e per ogni $A \in  \Pp(\phi_{1})$ e $B \in  \Pp(\phi_{2})$,
					si ha $\Hom_{\Dd}(A,B) = 0$;
					
					\item[iii)] per ogni $E \in \Dd$, 
					esistono numeri reali $\phi_{1} > \dots > \phi_{m}$ e
					 oggetti $E_{1}, \dots, E_{m} \in \Dd$ e $A_{1}, \dots, A_{m} \in \Dd$,
					 con ogni $A_{i} \in \Pp(\phi_{i})$
					 tali che si abbia il diagramma di triangoli esatti
					 \begin{equation*}
					 	\begin{tikzcd}
0 = E_{0} \arrow[r] & E_{1} \arrow[r] \arrow[d]               & E_{2} \arrow[r] \arrow[d]               & \dots \arrow[r] & E_{m-1} \arrow[r] \arrow[d]                 & E_{m} = E \arrow[d]                     \\
                    & {A_{1}} \arrow[lu, "+1", dashed] & {A_{2}} \arrow[lu, "+1", dashed] &                 & {A_{m-1}} \arrow[lu, "+1", dashed] & {A_{m}} \arrow[lu, "+1", dashed]
					 	\end{tikzcd}
					 \end{equation*}
					 detta \textbf{filtrazione di Harder-Narasimhan} di $E$ per $\Pp$.
				\end{itemize}
				Gli oggetti in $\Pp(\phi)$ sono detti di \textbf{fase $\phi$}.
			\end{df}
			
			\begin{ex}
				Sia $C$ una curva proiettiva liscia e $\Dd = \Db{C}$.
				Per ogni $\phi \in \R$, poniamo
				\begin{equation*}
					\Pp(\phi) := \langle E \text{ fascio } \mu\text{-semistabile, con } 
					\mu(E) = \phi \rangle\,.
				\end{equation*}
				Allora la collezione di queste $\Pp(\phi)$ è uno slicing di $\Db{C}$.
			\end{ex}
			
			Data $X$ una varietà proiettiva liscia, fissiamo ora $\Dd = \Db{X}$
			e un reticolo $\Lambda$ finitamente generato, con un omomorfismo di gruppi
			suriettivo
			\begin{equation*}
				\nu : K_{0}(X) \twoheadrightarrow \Lambda\,.
			\end{equation*}
			Noi siamo interessati al caso $\Lambda = K_{num}(X) := K_{0}(X)/T$,
			dove 
			$$T := \Set{ a \in K_{0}(X) \, | \, \forall_{b} \, \chi(a,b) = 0}$$
			e $K_{num}(X)$ è dotato del pairing $(a,b) := - \chi(a,b)$;
			il ``$-$'' è messo lì per questione
			di compatibilità con il pairing tra vettori di Mukai.
			
			\begin{ex}
				Se $X$ è una superficie K3, allora il vettore di Mukai
				\begin{equation}\label{k3:mukai-vector}
					\cat{v} : K_{num}(X) \longrightarrow H^{2*}(X;\Z)\,,
					\quad \cat{v}([E]) = (\rk(E), c_{1}(E), \rk(E) + \ch_{2}(E))
				\end{equation}
				è un'isometria.
			\end{ex}
			
			\begin{df}
				Una \textbf{condizione di stabilità di Bridgeland } su $X$ 
				è una coppia $\sigma = (\Pp,Z)$, dove $\Pp$ è uno slicing di $\Db{X}$
				e $Z:\Lambda \to \C$ tali che:
				\begin{itemize}
					\item[i)] $Z$ è un omorfismo di gruppi;
					\item[ii)] se $E \in \Pp(\phi)$, allora $Z(\nu(E)) = \lambda e^{i\pi\phi}$,
					con $\lambda > 0$ dipendente da $E$;
					\item[iii)] vale la \textbf{proprietà di supporto}:
					\begin{equation}\label{df:support}
						C_{\sigma} := \inf \Set{\frac{\lvert Z(\nu(E)) \rvert}{\lVert \nu(V) \rVert} \, |\,
						0 \ne E \in \Pp(\phi), \, \phi \in \R } > 0 \,.	
					\end{equation}					 
				\end{itemize}
				Gli oggetti di $\Pp(\phi)$ vengono detti $\sigma$-semistabili.
			\end{df}
			
			\begin{rmk*}
				La proprietà di supporto è stata introdotta più avanti
				da \textbf{Kontsevich} e \textbf{Soibelman}, 
				motivata dal fatto che, il numeretto $C_{\sigma}$ definito in \eqref{df:support}
				è in qualche modo legato al \textbf{discriminante} del fascio $E$.
				Richiedendo questa condizione, si garantisce la validità
				di una disuguaglianza del tipo di \textbf{Bogomolov},
				disuguaglianza soddisfatta dai fasci semistabili e che quindi
				è auspicabile che abbiano tutti i fasci \emph{stabili} in un qualche senso.
			\end{rmk*}
			
			\begin{thm}\label{thm:equivalent-df-stab}
				Sia $X$ una varietà proiettiva liscia e $\Dd = \Db{X}$.
				Dare una condizione di stabilità di Bridgeland $(\Pp,Z)$ su $\Db{X}$ equivale
				a dare una coppia $(\Aa, Z)$, con $\Aa$ abeliana e $Z$ condizione di stabilità
				su $\Aa$ nel senso di \textbf{\Cref{df:stab}} 
				e tale che la \textbf{proprietà di supporto}
				è verificata:
				\begin{equation}\label{supp-abeliana}
						\inf \Set{\frac{\lvert Z(\nu(E)) \rvert}{\lVert \nu(E) \rVert} \, |\,
						0 \ne E \in \Aa \text{ è $Z$-semistabile} } > 0 \,.	
					\end{equation}	
				\begin{proof}[Idea della costruzione]
					Data $(\Pp,Z)$, si pone $\Dd^{\le 0} := \bigcup_{\phi > 0} \Pp(\phi)$ e 
					$\Dd^{\ge 0} := \bigcup_{\phi \le 1} \Pp(\phi)$. Allora la coppia
					$(\Dd^{\le 0}, \Dd^{\ge 0})$ definiscono una $t$-struttura limitata su $\Dd$,
					il cui cuore $\Aa := \Dd^{\heartsuit} = \Pp((0,1])$ ha la filtrazione
					di Harder-Narasimhan rispetto a $Z$, dove
					\begin{equation*}
						Z(E) := \lambda \cos(\pi \phi) + i \lambda \sin(\pi \phi)\,. \qedhere
					\end{equation*}
				\end{proof}
			\end{thm}	