\documentclass[a4paper, 10pt, oneside, DIV=9, chapterprefix=true, numbers=enddot,bibliography=totoc]{scrbook}

\RedeclareSectionCommand[tocdynnumwidth]{chapter}
\RedeclareSectionCommands[tocdynindent]{section,subsection}
\usepackage{style}
\usepackage{shortcuts}
\usepackage[normalem]{ulem}
\usepackage[outline]{contour}
\contourlength{2.25pt}

\newcommand{\embrace}[1]{\textup{(}#1\textup{)}}
\newlength{\LETTERheight}
\AtBeginDocument{\settoheight{\LETTERheight}{I}}
\newcommand*{\longrightsquigarrow}[1]{\ \raisebox{0.24\LETTERheight}{\tikz \draw [-to,
		line join=round, line cap=round,
		decorate, decoration={
			zigzag,
			segment length=4,
			amplitude=.9,
			post=lineto,
			post length=0.42ex
		}] (0,0) -- (#1,0);}\ }
	
\newlength{\HeightOfTextstyleOne}
\settoheight{\HeightOfTextstyleOne}{$\mathbf{1}$}
\newlength{\HeightOfScriptstyleOne}
\settoheight{\HeightOfScriptstyleOne}{$\scriptstyle\mathbf{1}$}
\newlength{\HeightOfScriptscriptstyleOne}
\settoheight{\HeightOfScriptscriptstyleOne}{$\scriptscriptstyle\mathbf{1}$}
\newcommand{\FancyOne}[1]{{\tikz[line cap=round,line join=round,line width=0.35*#1/\HeightOfTextstyleOne,scale=#1/\HeightOfTextstyleOne]{\draw (-0.0225,0.205) to (-0.0225,0.02) to[out=270,in=0] (-0.0425,0) to (-0.071,0) to (0.071,0) to (0.0425,0) to[out=180,in=270] (0.0225,0.02) to (0.0225,0.235) to (0.0175,0.235) to[out=210,in=0] (-0.075,0.201);}}}
\newcommand{\IOne}{\mathchoice%
	{\FancyOne{\HeightOfTextstyleOne}}%
	{\FancyOne{\HeightOfTextstyleOne}}%
	{\FancyOne{\HeightOfScriptstyleOne}}%
	{\FancyOne{\HeightOfScriptscriptstyleOne}}%
}
\newcommand{\IDigamma}{\tikz[line cap=round,line join=round,line width=0.35]{\draw (0.06,0.2286) to[out=0,in=90] (0.1353,0.172) to (0.1353,0.2286) to (-0.0525,0.2286) to (-0.0415,0.2286) to[out=0,in=90] (-0.0215,0.2086) to (-0.0215,0.02) to[out=270,in=0] (-0.0415,0) to (-0.0525,0) to (0.0605,0) to (0.0415,0) to[out=180,in=270] (0.0215,0.02) to (0.0215,0.2086) to[out=90,in=180] (0.0415,0.2286);\draw (0.025,0.1335) to[out=0,in=90] (0.0968,0.0769) to (0.0968,0.1335) to cycle;}\hspace{0.1ex}\vphantom{\IF}}


\DeclareFontFamily{U}{min}{}
\DeclareFontShape{U}{min}{m}{n}{<-> udmj30}{}

	
\makeatletter
\renewcommand{\@pnumwidth}{2em} 
\renewcommand{\@tocrmarg}{3em}
\makeatother
%\RedeclareSectionCommand[tocindent+=0.5em]{section}
%\RedeclareSectionCommand[tocindent+=0.5em]{subsection}






%%%% FRONTMATTER %%%%

\subject{Appunti su}
\title{Spazi di Moduli}
\author{{\normalsize Docente}\\
	Prof. Arvid Perego}
\date{{\normalsize Scritti da}\\
	Filippo Papallo}
\publishers{A.A. 2025/26\\
	Università di Genova}

\usepackage{bookmark}



%%%% bibliography %%%%
\usepackage[backend=biber,style=alphabetic]{biblatex}
%\setcounter{biburlnumpenalty}{7000}
%\setcounter{biburllcpenalty}{7000}
%\setcounter{biburlucpenalty}{8000}
\addbibresource{res.bib}
\DeclareSourcemap{
	\maps[datatype=bibtex]{
		\map[overwrite=true]{
			\step[fieldsource=fjournal]
			\step[fieldset=journal, origfieldval]
		}
	}
}








%%%% DOCUMENT %%%%

\begin{document}

\setlength{\parindent}{0pt}
\setlength{\parskip}{4pt}

\frontmatter
\KOMAoption{chapterprefix}{false}
\renewcommand{\thedummy}{\arabic{dummy}}
\maketitle


Questo documento ha lo scopo di raccogliere vari appunti tratti dagli incontri con il mio relatore,
Prof. \textbf{Arvid Perego}, durante il mio ultimo anno di dottorato. Nel corso dell'anno accademico 2025/26, l'obiettivo è quello di studiare più in profondità la teoria degli spazi di moduli, sia
dal punto di vista classico, sia dal linguaggio moderno introdotto da Bridgeland in \cite{B-stab-k3}.

\hrulefill

Ultimo aggiornamento: \today
	
	
	
\tableofcontents
\listoftoc{lol}
\setcounter{llecture}{0}
\mainmatter\KOMAoption{chapterprefix}{true}
\renewcommand{\thedummy}{\thechapter.\arabic{dummy}}
\renewcommand{\thechapter}{\arabic{chapter}}

	\chapter{Spazi di moduli classici}

    	
	\lecture[Introduzione alla teoria dei moduli.]{2025-09-17}
	
	INRI
    	
    	
	\lecture[Moduli di oggetti (semi)stabili come soluzioni di un problema GIT.]{2025-09-24}
		
		Abbiamo quindi visto che aggiungendo condizioni sul tipo di
		fasci che vogliamo parametrizzare, si ha più speranza nell'esistenza di buoni spazi di moduli.
		In particolare, una proprietà che funziona bene è la \textbf{stabilità},
		di cui ne esistono diverse nozioni, ad esempio:
		\begin{enumerate}
			\item \textbf{stabilità di Takemoto-Mumford} o \textbf{$\mu$-stabilità}:
			dato $F \in \Coh(X)$ fascio \textbf{torsion-free}, definiamo la sua \textbf{pendenza}
			rispetto a un fibrato lineare ampio $H$ come
			\begin{equation}
				\mu(F) := \mu_H(F) = \frac{c_{1}(F) \cdot H}{\rk(F)}\,,
			\end{equation}			 
			dove la prima classe di Chern $c_{1}(F)$ si definisce come $c_{1}(\det F\vert_{U})$,
			dove $U \subset X$ è un aperto di Zariski su cui $F$ è localmente libero.
			Un fascio $F$ si dice \textbf{$\mu$-stabile} 
			(resp. \textbf{$\mu$-semistabile}) se, per ogni sottofascio proprio 
			non banale $0 \neq F'\subsetneq F$ con $0 < \rk(F') < \rk(F)$, la pendenza cresce:
			\begin{equation*}
				\mu(F') < \mu(F) \quad (\text{resp. } \mu(F') \le \mu(F))\,.
			\end{equation*}
			Se nella definizione di $\xM_{P}(X,H)$ vi si aggiunge la condizione 
			``\emph{$F_{s}$ è $\mu_{H}$-stabile, per ogni $s \in S$}'', allora
			il funtore diventa \textbf{pro-rappresentabile}!
			
			\item \textbf{stabilità di Gieseker-Maruyama} o \textbf{$H$-stabilità}:
			per ogni fascio coerente $F \in \Coh(X)$ \textbf{puro}\footnote{Dato $G \in \Coh(X)$, si definisce la \textbf{dimensione di $G$}come la dimensione del supporto di $G$, cioè il sottoschema chiuso $Supp(E) = \Set{x \in X \,|\, G_{x} \ne 0}$. Un fascio $G$ si dice \textbf{puro} di dimensione $d$ se ogni sottofascio $0 \ne G' \subset G$ ha $\dim(G) = d$.}, di $\dim(F) = d$, 
			andiamo a considerarne il polinomio di Hilbert rispetto a un fibrato ampio $H$
			\begin{equation*}
				P_{H}(F) := \chi(F \otimes \Oo_{X}(nH)) = \alpha_{d} n^{d} + altri \, termini...
				\quad \text{per } n \gge 0\,,
			\end{equation*}
			e lo rendiamo monico, introducendo così il \textbf{polinomio di Hilbert ridotto}
			$p_{H}(F) := P_{H}(F)/\alpha_{d}$. Diremo che $F$ è \textbf{$H$-stabile} 
			(resp. \textbf{$H$-semistabile}) se, per ogni sottofascio $0 \neq F'\subsetneq F$
			si ha
			\begin{equation}\label{Gieseker-stable}
				p_{H}(F') < p_{H}(F) \quad (\text{resp. } p_{H}(F') \le p_{H}(F) )\,,
			\end{equation}
			dove l'ordine è da intendersi come ordinamento lessicografico. Anche
			in questo caso, gli oggetti $H$-stabili danno spazi di moduli belli e proiettivi!
		\end{enumerate}
		
		Ma come mai queste nozioni danno spazi di moduli?
		Grothendieck aveva scoperto che una condizione necessaria per la rappresentabilità
		è la \textbf{limitatezza}:
		\begin{df}
			Una famiglia di fasci $\Ff \subset \Coh(X)$ si dice \textbf{limitata}
			se esiste $S$ uno schema su $\C$ di tipo finito 
			e un fascio $\Oo_{S \times X}$-coerente $G$ tale che
			$\Ff = \Set{F\vert_{s \times X} \,|\, s \in U \text{ punto chiuso}}$,
			per un determinato \emph{sottoinsieme} $U \subset S$.
		\end{df}
		Sottolineiamo che $U$ nella definizione è solo un \emph{insieme}
		che a priori non ha nessuna struttura geometrica. 
		Questa proprietà è più facile da verificare rispetto alla rappresentabilità,
		grazie a criteri elaborati dallo stesso Grothendieck; uno di questi
		è legato incredibilmente alla \textbf{regolarità di Castelnuovo-Mumford}.
		
		\begin{prop}
			Dato $H$ un fibrato lineare ampio su $X$, se esiste $m \in \N$ tale che 
			ogni $F \in \Ff$ sia $m$-regolare rispetto a $H$, allora $\Ff$ è limitata.
		\end{prop}
		
		\begin{prop}
			Dato $H$ un fibrato lineare ampio su $X$ e $P \in \Q[x]$, la famiglia
			\begin{equation}
				\Ff = \Set{F \in \Coh(X) \, | \, F \text{ è } \ast\text{-semistabile e } P_{H}(F) = P}\,,
			\end{equation}
			con $\ast = H$ oppure $\ast = \mu$, è limitata.
		\end{prop}
		
		In particolare, questo implica che esiste un $m \in \N$ tale che, per ogni $F \in \Ff$,
		il fascio $F \otimes \Oo_{X}(mH)$ e globalmente generato, cioè la valutazione è surgettiva:
		\begin{equation*}
			H^{0}(X, F \otimes \Oo_{X}(nH)) \otimes \Oo_{X}(-mH) \twoheadrightarrow F\,.
		\end{equation*}
		Notiamo inoltre che la dimensione dell'$H^{0}$ non dipende da $F$, infatti per la 
		scelta di $m$ si ha
		\begin{align*}
			P_{H}(F)(m) = \chi(F \otimes \Oo_{X}(mH))
			= \sum_{i=0}^{\dim(F)} (-1)^{-1} h^{i}(F \otimes \Oo_{X}(mH)) = h^{0}(F \otimes \Oo_{X}(mH)\,,
		\end{align*}
		quindi $H^{0}(X, F \otimes \Oo_{X}(nH)) \otimes \Oo_{X}(-mH)$ è sempre lo stesso fascio,
		indipendentemente da $F$,
		che denoteremo $\Hh_{m}$. Il vantaggio, ora, è che possiamo vedere un $F \in \Ff$ qualsiasi
		come un quoziente $\Hh_{m} \twoheadrightarrow F$, ma quindi $F$ è un punto di uno schema Quot:
		\begin{equation*}
			[F] \in \cat{Quot}_{X}(\Hh_m, P)\,.
		\end{equation*}
		
		\begin{df}
			Sia $\xP$ una proprietà di varietà su $\C$. Diremo che $\xP$ è \textbf{aperta}
			se, per ogni famiglia piatta $f : \xX \to B$, l'insieme
			\begin{equation*}
				\Set{b \in B \,|\, \xX_{b} := f^{-1}(b) \text{ verifica } \xP}
			\end{equation*}
			è un aperto di $B$. Analogamente, una una proprietà $\xP$ di fasci coerenti su una varietà $X$ è detta \textbf{aperta} se, per ogni famiglia piatta $f : \xX \to B$ e $\Ff \in \Coh(\xX)$,
			 l'insieme
			\begin{equation*}
				\Set{b \in B \,|\, \Ff_{b} := \Ff\vert_{f^{-1}(b)} \text{ verifica } \xP}
			\end{equation*}
			è un aperto di $B$.
		\end{df}
		
		\begin{ex}
			Essere K\"ahleriana è una proprietà aperta di varietà; essere proiettiva
			non lo è: si pensi ad esempio allo spazio dei moduli delle K3, dove le K3 proiettive vivono in codimensione $1$.
		\end{ex}
		
		\begin{ex}
			Essere $\ast$-(semi)stabili, con $\ast=H,\mu$, è una condizione aperta per fasci!
		\end{ex}
		
		Segue da questo fatto che $R = \Set{[F] \in \cat{Quot}_{X}(\Hh_{m},P) \, | \, F \text{ è stabile}}$ è un aperto dello schema $\cat{Quot}$. Tuttavia, quest'ultimo \textbf{non} è lo spazio
		di moduli che stavamo cercando... in effetti, la notazione $[F]$ per un punto di $R$
		è un abuso di notazione: infatti, quello che tale schema parametrizza sono
		classi di equivalenza di
		 \textbf{morfismi surgettivi} $[\Hh_{m} \twoheadrightarrow F]$,
		 che in generale sono molti di più dei fasci stabili che vogliamo classificare.
		 Posto $V := H^{0}(X, \Oo_{X}(mH))$, 
		 si noti che $\GL(V)$ agisce per cambio di base su $V \otimes \Oo_{X}(-mH)$. 
		 L'azione $\GL(V) \curvearrowright V \otimes \Oo_{X}(-mH)$ 
		 si solleva a un'azione $\GL(V) \curvearrowright \Hh_{m}$;
		 in realtà, dato che gli scalari non cambiano la classe di isomorfismo del fibrato,
		 il gruppo che vi agisce è $\PGL(V)$ e si ottiene
		 \begin{align*}
		 	M_{H}^{ss}(X,P) &= R^{ss}/ \PGL(V) \quad \text{buon quoziente,}\\
		 	M_{H}^{s}(X,P) &= R^{s}// \PGL(V) \quad \text{quoziente geometrico!}
		 \end{align*}
    	
    		
	\lecture[.]{2025-10-09}
	

	Come sempre, sia $k$ un campo algebricamente chiuso $k$, di caratteristica $0$.
	Consideriamo $X$ uno schema proiettivo connesso su $k$,
	fissiamo $H$ un fibrato lineare ampio e $P \in \Q[t]$ un polinomio.
	\begin{thm}
		Lo spazio dei moduli dei fasci $H$-semistabili $M_{P}(X,H)$ esiste ed
		è uno schema \textbf{proiettivo}.
	\end{thm}
	
	Questo risultato è generalissimo. La sua debolezza è che sappiamo solamente che $M_{P}(X,H)$ è
	uno schema, ma comunque siamo felici di avere uno schema proiettivo! 
	Questo fatto è una conseguenza della costruzione GIT. 
	Inoltre, le nozioni di \textbf{stabilità} coincidono, 
	sia nel senso di Gieseker, sia nel senso GIT (ovvero la condizione sulle orbite).
	
	Ma cerchiamo ora di capire cosa rappresentano i punti di questi schemi.
	
	\section{Filtrazioni}
	
		\subsection{Harder-Narasimhan}
		
		Sia $X$ come sopra. 
		\begin{thm}[\textbf{Harder-Narasimhan}, 1975]
		 	Per ogni fascio $F \in \Coh(X)$, esiste un'unica filtrazione
		 	\begin{equation*}
		 		0 = F_{0} \subsetneq F_{1} \subsetneq \dots \subsetneq F_{l-1} \subsetneq F_{l} = F
		 	\end{equation*}
		 	tale che, per ogni $0 < i \le l$, ogni fattore $E_{i}/E_{i-1}$ sia $H$-semistabile e 
		 	si abbia pendenza decrescente $p_{H}(F_{i}/F_{i-1}) > p_{H}(F_{i+1}/F_{i})$.
		 \end{thm}
		 Lo stesso risultato vale per la $\mu$-stabilità:
		 nell'enunciato del teorema possiamo rimpiazzare `\emph{$H$-semistabili}' 
		 con `\emph{$\mu$-semistabili}'
		 e il polinomio di Hilbert ridotto $p_{H}$ con la pendenza $\mu$.
		 Nel seguito useremo solo la stabilità di Gieseker, ma va ricordato che
		 risultati analoghi valgono anche per la slope stability.
		 
		 Quindi, per capire i fasci su uno schema proiettivo connesso $X$ è sufficiente
		 classificare i suoi fasci semistabili (in un qualsiasi senso).
		 Ma possiamo decomporre ulteriormente i fasci semistabili in \emph{fattori stabili}:
		  \begin{thm}[\textbf{Filtrazione di Jordan-H\"older}]\label{JH-filtration}
		 	Dato $F \in \cat{Coh}(X)$ un fascio $H$-semistabile, esiste una filtrazione (non unica!)
		 	\begin{equation*}
		 		0 = F_{0} \subsetneq F_{1} \subsetneq \dots \subsetneq F_{l-1} \subsetneq F_{l} = F
		 	\end{equation*}
		 	tale che, per ogni $0 < i \le l$, ogni fattore $F_{i}/F_{i-1}$ sia $H$-stabile e 
		 	tutti i termini abbiano lo stesso polinomio 
		 	$p_{H}(F) = p_{H}(F_{i}/F_{i-1}) = p_{H}(F_{i+1}/F_{i})$.
		 \end{thm}
		 
		 Quindi è sufficiente classificare solo i \textbf{fasci stabili}? Non proprio, perché abbiamo
		 appena detto che questa filtrazione \textbf{non} è unica... Quindi dobbiamo considerarla
		 a meno di una certa relazione di equivalenza:
		 si dimostra che, se
		\begin{equation*}
		 		0 = F'_{0} \subsetneq F'_{1} \subsetneq \dots \subsetneq F'_{s-1} \subsetneq F'_{s} = F
		 	\end{equation*}
		 	è un'altra filtrazione di JH per $F$, allora $l=s$ e, per ogni $1 \le i \le l$, 
		 	esiste un unico $1 \le j \le l$ tale che
		 	\begin{equation}
		 		F_{i}/F_{i-1} \simeq F'_{j}/F'_{j-1}\,.
		 	\end{equation}
		 	Quindi a essere unico è il \textbf{graduato di Jordan-H\"older} associato a $F$,
		 	cioè il fascio di moduli graduati
		 	\begin{equation*}
		 		gr_{JH}(F) =  \bigoplus_{i=1}^{l} gr^{i}_{JH}(F) := \bigoplus_{i=1}^{l} F_{i}/F_{i-1}\,.
		 	\end{equation*}
		 	
		 	\begin{df}
		 		Due fasci $H$-semistabili $F$ e $F'$ sono \textbf{S-equivalenti} se e solo se
		 		$$gr_{JH}(F) \simeq gr_{JH}(F')\,.$$
		 	\end{df}
		 	Si deduce  che i punti chiusi di $M_{P}(X,H)$ parametrizzano le \textbf{classi di S-equivalenza}
		 	dei fasci $H$-semistabili di polinomio di Hilbert fissato $P$, mentre
		 	$M^{S}_{P}(X,H)$ rappresenta le classi di S-equivalenza di fasci stabili,
		 	che però coincidono con le classi di \textbf{isomorfismo} di fasci stabili 
		 	di polinomio fissato $P$.
		 	
		 	Allora adesso possiamo concludere che $M^{S}_{P}(X,H)$ è uno schema che rappresenta
		 	il funtore dei fasci stabili? Purtroppo no...		 	
		 	Cerchiamo di capire cosa impedisce la rappresentabilità del funtore $\xM_{H}(X,P)$.
		 	\begin{prop*}
		 		Per ogni $F \in \Coh(X)$, esiste una successione esatta corta
				\begin{equation*}
					0 \longrightarrow E \longrightarrow F \longrightarrow G \longrightarrow 0
				\end{equation*}						 		
		 		 con $E,G$ fasci $H$-stabili su $X$ e $p_{H}(E) = p_{H}(G)$, 
		 		 e esiste $\Ff \in \Coh(X \times \AA^{1})$
		 		 una famiglia di fasci $\AA^{1}$-piatta tale che 
		 		 \begin{equation*}
		 		 	\Ff\vert_{0} \simeq E \oplus G\,, \quad \Ff\vert_{t} \simeq F\,, \text{ per ogni } t \ne 0\,.
		 		 \end{equation*}
		 	\end{prop*}
		 	
		 	\begin{cor*}
		 		Il funtore $\xM_{H}(X,P)$ non è rappresentabile.
		 		\begin{proof}
		 			Supponiamo che $\xM_{H}(X,P)$ sia rappresentato da uno schema $M$.
		 			L'inclusione $\iota: \AA^{1} \setminus \{0\} \subset \AA^{1}$
		 			induce una funzione $\xM_{H}(X,P)(\AA^{1} \setminus \{0\}) \to \xM_{H}(X,P)(\AA^{1})$,
		 			che per rappresentabilità corrisponde alla precomposizione con l'incusione
		 			\begin{equation*}
		 				- \circ \iota : \Hom_{\Sch}\left(\AA^{1} \setminus \{0\},M \right) \longrightarrow \Hom_{\Sch}\left(\AA^{1},M \right)\,.
		 			\end{equation*}
		 			Ora, notiamo che l'immagine di
		 			$\Ff\vert_{\AA^{1}\setminus \{0\}} \in \xM_{H}(X,P)(\AA^{1} \setminus \{0\})$
		 			attraverso questa funzione è $\Ff \in \xM_{H}(X,P)(\AA^{1})$.
		 			Tuttavia, $\Ff\vert_{\AA^{1}\setminus \{0\}}$ è la famiglia che vale costantemente $F$
		 			sull'aperto $\AA^{1}\setminus \{0\}$, quindi per continuità si estende alla famiglia
		 			costante $\underline{F}$ su tutta la retta $\AA^{1}$, ma questo contraddice
		 			il fatto che l'estensione sia $\Ff$, che non è costante.
		 		\end{proof}
		 	\end{cor*}
		 	
		 	Tuttavia ci sono dei casi in cui possiamo concludere che 
		 	il funtore dei fasci stabili $\xM^{s}_{H}(X,P)$ è rappresentabile.
		 	Per enunciarlo, abbiamo bisogno di parlare di un altro fatto.
		 	Solitamente, leggendo gli articoli, in generale non si fissa $P \in \Q[t]$,
		 	ma è consuetudine fissare degli \textbf{invarianti numerici}.
		 	Adesso assumiamo che $X$ sia uno schema proiettivo \textbf{liscio}\footnote{Per il caso singolare dobbiamo capire quale sia il setting giusto; forse ce la possiamo cavare con il gruppo di Grothendieck dei \textbf{complessi perfetti}.}.
		 	In questo caso, possiamo considerare il \textbf{gruppo di Grothendieck numerico} di $X$
		 	\begin{equation}
		 		K_{num}(X) := K_{0}(X)/T\,,
		 	\end{equation}
		 	ottenuto quozientando il gruppo di Grothendieck $K_{0}(X) = K(\Coh(X))$
		 	per il sottogruppo
		 	\begin{equation*}
		 		T := \Set{[E] \in K_{0}(X) \,|\, \forall_{a \in K_{0}(X)}\, \chi([E],a) = 0 }\,;
		 	\end{equation*}
		 	in questo modo, la caratteristica di Eulero $\chi$ definisce un pairing su $K_{num}(X)$. 
		 	Dato $E \in \Coh(X)$, gli associamo $c(E) \in K_{num}(X)$.
		 	Osserviamo che, per definizione di $P_{H}(F)$, questo dipende dalla polarizzazione
		 	$X$ e dal carattere di Chern: quindi fasci con stesso $\ch$ danno lo stesso polinomio.
		 	Notiamo che, se $[E] \in T$, allora per ogni fascio $G \in \Coh(X)$
		 	Hirzebruch-Riemann-Roch ci dà
			\begin{equation*}
				\chi(E,G) = -\int_{X} \ch(E)\ch(G)^{\vee} 
			\end{equation*}			 	
				 	
		 	\begin{equation}
		 		\ch : K_{num}(X)
		 	\end{equation}
    	
    \chapter{Condizioni di stabilità di Bridgeland}
    
    	
	\lecture[Inquadramento storico e motivazione che ha portato alla formulazione delle condizioni di stabilità su categorie triangolate da parte di T. Bridgeland.]{2025-09-30}
	
	\section{Motivazioni}
	
		Come sostiene sempre Arvid, per poter capire a fondo alcuni concetti matematici
		e il perché di alcune definizioni, è necessario capire il contesto storico in cui sono nate
		e le motivazioni che vi stanno dietro. Come molte teorie della matematica moderna,
		le ragioni dietro le condizioni di stabilità di Bridgeland vanno ricercate nella
		\emph{fisica}, dove già era importante il concetto di \textbf{stabilità di fibrati vettoriali}.
		
		Data $M$ una varietà $C^{\infty}$ e $E$ un fibrato vettoriale $C^{\infty}$ complesso su $M$,
		si può definire su esso una \textbf{connessione} $D$, cioè una mappa $C^{\infty}(M)$-lineare
		\begin{equation*}
			D : A^{0}(E) \longrightarrow A^{1}(E)
		\end{equation*}
		che può a sua volta essere estesa a una $D : A^{1}(E) \to A^{2}(E)$ grazie alla quale
		si definisce l'\textbf{opertatore di curvatura} $R_{D} := D \circ D$.
		
		\begin{fact*}
			Questo operatore può essere interpretato come $R_{D} \in A^{2}(End(E))$.
		\end{fact*}
		
		In geometria complessa siamo interessati a varietà \emph{più ricche}: se $X$ è una varietà
		olomorfa e $E$ un fibrato vettoriale $C^{\infty}$ complesso su $X$, 
		allora abbiamo la decomposizione di Hodge
		\begin{equation*}
			A^{1}(E) = A^{1,0}(E) \oplus A^{0,1}(E)\,,
		\end{equation*}
		da cui si deduce una decomposizione $D = D^{1,0} + D^{0,1}$, 
		per una qualsiasi connessione $D$ su $X$.
		
		\begin{df*}
			Una connessione $D$ si dice \textbf{compatibile} con la struttura complessa su $X$
			se $D^{0,1} = \overline{\partial}$.
		\end{df*}
		
		\begin{fact*}
			Nel caso $D$ sia compatibile , l'operatore di curvatura  $R_{D} \in A^{1,1}(End(E))$.
		\end{fact*}
		
		Quando studiamo un fibrato vettoriale complesso $E$, possiamo studiare delle
		\textbf{metriche hermitiane} $h$ su di esso; anche in questo caso è 
		si può studiare l'interazione di una connessione rispetto a una metrica:
		
		\begin{df*}
			Se $h$ è una matrica hermitiana su $E$, una connessione $D$ 
			si dice \textbf{compatibile con h} se, per ogni $a,b \in A^{0}(E)$ vale
			\begin{equation*}
			 	D(h(a,b)) = h(Da,b) + h(a,Db)\,.
			\end{equation*}
		\end{df*}
		
		\begin{thm*}
			Dato un fibrato vettoriale complesso $E$ su una varietà olomorfa $X$ e $h$ una metrica hermitiana su $E$, esiste un'unica connessione $D_{h}$, detta \textbf{connessione di Chern},
			compatibile sia con la struttura complessa di $X$, sia con $h$.
		\end{thm*}
		
		D'altra parte, possiamo considerare una metrica hermitiana $g$ sul tangente complessificato
		$T_{M} \otimes \C$ e con questa definire l'\textbf{operatore di Hodge} $\ast_{g}$
		e l'\textbf{operatore di Lefschetz} $L_{g}$. Grazie all'aggiunto
		\begin{equation*}
			\Lambda_{g} := \ast_{g}^{-1} \circ L_{g} \circ \ast_{g} : 
			A^{p,q}(E) \longrightarrow A^{p-1,q-1}
		\end{equation*}
		viene definita la sezione $\Lambda_{g}(R_{h}) \in A^{0}(End(E))$.
		
		\begin{df*}
			La metrica $h$ si dice \textbf{$g$-Hermite-Einstein} se esiste  $\lambda \in \R$
			tale che $$\Lambda_{g}(R_{h}) = \lambda \id_{E} \,.$$
		\end{df*}
		
		La costante $\lambda$ nella definizione viene talvolta chiamata \textbf{costante cosmologica}
		e ha un'importante interpretazione in fisica: infatti, le connessioni Hermite-Einstein
		corrispondono alle soluzioni delle \textbf{equazioni di Yang-Mills} che descrivono le
		particelle del modello standard. 
		Inoltre, queste connessioni hanno un inaspettato legame con la geometria algebrica:
		infatti, data $h$ una metrica hermitiana su $E$, 
		possiamo considerare la \textbf{prima classe di Chern} 
		$$c_{1}(E,h) \in H^{1,1}(X) \cap H^{2}(X;\Z)\,,$$
		che indicheremo solo con $c_{1}(E)$ poiché, nonostante la $2$-forma dipenda da $h$,
		la sua classe in coomologia è sempre la stessa. 
		Grazie ad essa, definiamo la \textbf{$g$-slope}
		del fibrato $E$ come la quantità
		\begin{equation*}
			\mu_{g}(E) := \frac{c_{1}(E) \cdot \omega_{X}^{n-1}}{\rk(E)} \in \R\,, 
			\quad \text{dove } n = \dim X\,.
		\end{equation*}
		
		\begin{fact*}
			Per $h$ metrica $g$-Hermite-Einstein si dimostra che $\mu_{g}(E)$ è essenzialmente
			la costante cosmologica $\lambda$ nella definizione, i.e. il rapporto $\mu_{g}/\lambda$
			è una costante che non dipende da $E$.
		\end{fact*}
		
		\begin{df*}
			Un fibrato vettoriale olomorfo $E$ su $X$ si dice \textbf{$\mu_{g}$-stabile} se,
			per ogni sottofascio coerente $0 \ne F \subsetneq E$ vale
				$\mu_{g}(F) < \mu_{g}(E)$. 
			Diremo che $E$ è \textbf{$\mu_{g}$-polistabile} 
			se $E \simeq E_{1} \oplus \dots \oplus E_{l}$, dove ogni $E_{i}$ è $\mu_{g}$-stabile
			e $\mu_{g}(E_{1}) = \dots = \mu_{g}(E_{l})$.
		\end{df*}
		
		\begin{thm*}[\textbf{Corrispondenza di Kobayashi-Hitchin}]
			Sia $X$ una varietà K\"ahler compatta, $E$ un fibrato vettoriale olomorfo su $X$ e $g$
			metrica K\"ahleriana su $E$. Allora
			\begin{equation*}
				E \text{ è } g\text{-Hermite-Einstein} \quad \iff \quad 
				E \text{ è } \mu_{g}\text{-polistabile}\,.
			\end{equation*}
		\end{thm*}
		Questo risultato è difficilissimo! 
		
		Altre motivazioni dietro la ricerca delle condizioni di stabilità vanno ricercate:
		 nella \textbf{teoria delle stringhe}: a seguito della \emph{homological mirror symmetry} 
		 cogetturata da Kontsevich in \cite{K-hms}, si è cominciato a pensare che le particelle
		 descritte dalle equazioni di Yang-Mills dovrebbero corrispondere a \emph{qualcosa di stabile}
		 nella categoria derivata dei \textbf{fasci coerenti di una varietà}: ecco di nuovo i fasci
		 stabili! In \cite{D-Dbranes}, Douglas formalizza 
		 il concetto di $\pi$-stabilità e sviluppa il collegamento con la mirror symmetry.
		 Tuttavia, i fisici si ritrovano confusi dal seguente problema: cosa significa
		 prendere un sottofascio di un complesso di fasci? 
		 La categoria derivata di una categoria abeliana non è quasi mai abeliana, 
		 quindi non ha senso parlare di sottoggetti...
		 
		 Nel 2002, Tom Bridgeland formalizza per bene in 
		 il concetto di \textbf{condizioni di stabilità} 
		 per una categoria triangolata in una preprint di arXiv (pubbicata poi come \cite{B}), 
		 e in \cite{B-stab-k3} dà una teoria fatta e finita
		 per superfici K3.
		 
		 \begin{quote}
		 	\emph{Imitiamo quello che succede per le curve, di cui conosciamo bene la teoria!}
		 \end{quote}
		 Tra gli strumenti principali ad aver avuto impatto in questa teoria
		 ricordiamo la \textbf{filtrazione di Harder-Narasimhan} e gli \textbf{spazi di moduli}.
		 
		 \begin{question*}[\textbf{di Arvid}]
		 	Nello studio della stabilità ``classica'' interviene lo studio delle metriche sui fibrati
		 	che crea un collegamento tra la stabilità in senso \emph{differenziale} e quello \emph{algebrico}.
		 	Passando alle condizioni di Bridgeland, questo pezzo viene a mancare?
		 	Esiste una qualche nozione di ``metrica hermitiana'' per complessi di fibrati
		 	che permetta di costruire un ponte tra il mondo differenziale e quello algebrico?
		 \end{question*}
		 
		 
		 
		 
		 \section{Stabilità su curve}
		 
		 Sia $C$ una curva proiettiva liscia su $\C$. Un problema classico in geometria
		 è la classificazione di tutti i fibrati vettoriali olomorfi $E \to \C$.
		 Nel caso $C = \PP{1}{}$ la classificazione è dovuta a \textbf{Grothendieck}:
		 \begin{thm}
		 	Dato $E$ è un fibrato olomorfo su $\PP{1}{}$, esiste un'unica sequenza di interi
		 	$a_{1} > a_{2} > \dots > a_{n}$ e di spazi vettoriali $V_{1}, \dots, V_{n}$ tali che
		 	\begin{equation}
		 		E \simeq (V_{1} \otimes \Oo(a_{1})) \oplus \dots \oplus (V_{n} \otimes \Oo(a_{n}))\,.
		 	\end{equation}
		 \end{thm}
		 Se la dimostrazione di questo fatto si riconduce, essenzialmente, 
		 a un problema di algebra lineare, la classificazione per $C$ di genere positivo è stata
		 risolta utilizzando tecniche di \emph{stabilità}.
		 
		 \begin{df}
		 	Sia $C$ una curva e fissiamo $H$ un fibrato lineare ampio su di essa.
		 	Dato $E$ un fibrato olomorfo su $C$, definiamo la sua \textbf{pendenza} 
		 	(dall'inglese ``\textbf{slope}'') come
		 	\begin{equation*}
		 		\mu(E) := \frac{\deg(E)}{\rk(E)}\,,
		 	\end{equation*}
		 	dove $\deg(E) := \deg(\det E)$ è il grado rispetto alla polarizzazione $H$.
		 \end{df}
		 
		 \begin{fact}
		 	Se $F$ è un fascio coerente su una curva $C$, allora in $\cat{Coh}(C)$ esiste una
		 	successione esatta corta canonica
		 	\begin{equation*}
		 		0 \to \Tt_{F} \to F \to \Ff_{F} \to 0\,,
		 	\end{equation*}
		 	dove $\Ff_{F}$ è un fascio localmente libero e $\Tt_{F}$ è un fascio di torsione.
		 	Grazie a questa decomposizione, possiamo definire
		 	\begin{equation*}
		 		\rk(F) := \rk(\Ff_{F})\,, \quad \deg(F) = \deg(\Ff_{F}) 
		 		+ \operatorname{length}(\Tt_{F})\,.
		 	\end{equation*}
		 \end{fact}
		 
		Grazie a questo fatto, possiamo quindi dare un senso alla pendenza $\mu(F)$ di qualsiasi
		fascio coerente usando la convenzione che, se $\rk(F)=0$, allora $\mu(F)=+\infty$.	
		Si noti in particolare che $\rk(F)=0$ implica $\deg(F) =  \operatorname{length}(\Tt_{F}) > 0$.	 
		 
		 \begin{df}
		 	Un fascio $F \in  \cat{Coh}(C)$ è detto \textbf{$\mu$-stabile} 
		 	(resp. \textbf{$\mu$-semistabile}) se, per ogni sottofascio $0 \ne F'\subsetneq F$ vale
		 	\begin{equation*}
		 		\mu(F') < \mu(F) \quad (\text{resp. } \mu(F') \le \mu(F))\,.
		 	\end{equation*}
		 \end{df}
		 
		 \begin{thm}[\textbf{Harder-Narasimhan}, 1975]\label{HN-filtration}
		 	Dato $F \in \cat{Coh}(C)$, esiste un'unica filtrazione
		 	\begin{equation*}
		 		0 = F_{0} \subsetneq F_{1} \subsetneq \dots \subsetneq F_{l-1} \subsetneq F_{l} = F
		 	\end{equation*}
		 	tale che, per ogni $0 < i \le l$, ogni fattore $E_{i}/E_{i-1}$ sia $\mu$-semistabile e 
		 	si abbia pendenza decrescence $\mu(F_{i}/F_{i-1}) > \mu(F_{i+1}/F_{i})$.
		 \end{thm}
		 
		 Alla luce di questo risultato, per classificare tutti i possibili fibrati su una curva
		 (ma in generale, su una varietà), è sufficiente conoscere tutti i possibili
		 fasci \textbf{$\mu$-semistabili}.
		 
		 \begin{ex}
		 	Se $C=\PP{1}{}$, un fibrato $E$ è semistabile se e solo se $E = V \otimes \Oo(a)$,
		 	per un qualche $a \in \Z$ e $V$ spazio vettoriale.
		 \end{ex}
		 
		 \begin{ex}
		 	Se $C$ è una curva ellittica, la classificazione dei fasci $\mu$-semistabili
		 	è stata risolta da Atiyah. Se $\gcd(\rk, \deg)=1$, allora valgono le equivalenze
		 	\begin{equation*}
		 		E \text{ è } \mu\text{-semistabile } \quad \iff \quad 
		 		E \text{ è } \mu\text{-stabile } \quad \iff \quad
		 		E \text{ è indecomponibile}\,,
		 	\end{equation*}
		 	e si sa che i fasci indecomponibili sono della forma $E \simeq L \otimes F$,
		 	dove $L \in \Pic^{0}(C)$ e $F$ si ottiene iterando un numero finito di volte
		 	estensioni di $\Oo_{C}$. Se $\gcd(\rk, \deg) > 1$, la classificazione è
		 	più complicata, ma esiste.
		 \end{ex}
		 
		 Per genere $g(C) \ge 2$, utilizziamo invece un altro tipo di tecnica:
		 dati $r \in \N$ e $d \in \Z$, definiamo 
		 \begin{equation*}
		 	\xM_{C}(r,d) : \cat{Sch}/\C \longrightarrow \cat{Set}\,,
		 \end{equation*}
		 il funtore che su uno schema $S$ vale
		 \begin{equation*}
		 	\xM_{C}(r,d)(S) := \left\{
		 	E \text{ fibrato su } C \times S  \, \middle| \,
		 	\begin{aligned}
		 		\text{per ogni }& s \in S\,,   \\
		 		E_{s} := E\vert_{\operatorname{pr}_{S}^{-1}(s)} &\text{ è } \mu\text{-semistabile }\\
		 		\text{ e } \rk(E_{s}) =& r\,, \, \deg(E_{s}) = d
		 	\end{aligned}
		 	\right\}\,.
		 \end{equation*}
		 
		 \begin{thm}\label{thm:moduli-curve}
		 	Il funtore $\xM_{C}(r,d)$ è \textbf{pro-rappresentabile} da uno schema
		 	$M_{C}(r,d)$, chiamato \textbf{spazio dei moduli} di fasci semistabili (di rango $r$ e grado $d$). Inoltre, $M_{C}(r,d)$ è una varietà proiettiva, integrale, normale e localmente fattoriale 
		 	di dimensione $r^{2}(g-1) - 1$ e il suo gruppo di Picard è
		 	\begin{equation*}
		 		\Pic(M_{C}(r,d)) \simeq \Pic(\Pic^{d}(C)) \oplus \Z\,.
		 	\end{equation*}
		 \end{thm}
		 
		 \begin{rmk*}
		 	Si noti che il grado $d$ non entra nella dimensione, 
		 	conseguenza del fatto che i jacobiani di ogni grado sono tutti isomorfi.
		 \end{rmk*}
		 
		 All'interno di $M_{C}(r,d)$ troviamo lo spazio $M_{C}^{s}(r,d)$
		 che parametrizza i fasci $\mu$-stabili.
		 
		 \begin{fact*}
		 	Se $\gcd(r,d) = 1$, allora lo spazio dei moduli è liscio e $M_{C}(r,d) = M_{C}^{s}(r,d)$.
		 	Se invece $\gcd(r,d) > 1$, allora $M_{C}(r,d) \sm M_{C}^{s}(r,d)$ è il luogo
		 	singolare, che vive in codimensione maggiore o uguale a $2$.
		 \end{fact*}
		 
		 Invece di considerare il funtore $\xM_{C}(r,d)$, è interessante fissare $\Ll \in \Pic(C)$
		 e studiare il funtore $\xM_{C}(r,\Ll)$ i cui fasci $E \in \xM_{C}(r,\Ll)(S)$
		 hanno, per ogni $s \in S$, il determinante fissato $\det(E_{s}) = \Ll$;
		 anche in questo caso lo spazio dei moduli $M_{C}(r,\Ll)$ esiste come schema proiettivo
		 e ha dimensione $(r^{2}-1)(g(C)-1)$, esiste un divisore
		 ampio $\theta$ tale che $\Pic(M_{C}(r,\Ll)) \simeq \Z \theta$ e 
		 $K_{M_{C}(r,\Ll)} \simeq -2u\theta$, con $u=\gcd(r,d)$.
		 
		 Nel caso in cui, al posto di una curva $C$ andiamo a considerare una varietà di dimensione
		 maggiore, la situazione diventa più complicata. Noi saremo maggiormente interessati
		 al caso in cui $X$ è una superficie complessa, compatta (probabilmente anche proiettiva).
		 Anche in dimensione più alta, la filtrazione di Harder-Narasimhan~\ref{HN-filtration}
		 continua a esistere, ma abbiamo un'altra filtrazione molto utile,
		 che ci permette di ridurre il problema della classificazione dei
		 fasci $\mu$-semistabili solamente a quelli $\mu$-stabili.
		 
		 \begin{thm}[\textbf{Filtrazione di Jordan-H\"older}]
		 	Sia $F$ un fascio semistabile su $X$. Esiste una filtrazione (non unica!)
		 	\begin{equation*}
		 		0 = F_{0} \subsetneq F_{1} \subsetneq \dots \subsetneq F_{l-1} \subsetneq F_{l} = F
		 	\end{equation*}
		 	tale che le sue componenti graduate $gr_{i}(F) := F_{i}/F_{i-1}$ siano
		 	$\mu$-stabili, con stessa pendenza $\mu(gr_{i}(E)) = \mu(E)$, 
		 	per ogni $0 < i \le l$. Inoltre, il fascio graduato $gr(F) := \oplus_{i=1}^{l} gr_{i}(F)$
		 	non dipende dalla filtrazione scelta.
		 \end{thm}
		 
		 Parallelamente, si possono studiare altre nozioni di stabilità, diverse dalla $\mu$-stabilità,
		 come ad esempio la \textbf{stabilità di Gieseker}, definita in \eqref{Gieseker-stable}: 
		 anche in questo caso otteniamo
		 filtrazioni analoghe a quella di Harder-Narasimhan e Jordan-H\"older.
		 
		 \begin{rmk}
		 	Nel caso $X=C$ sia una curva, le nozioni di $\mu$-stabilità e Gieseker stabilità coincidono, 
		 	infatti, dato $F \in \cat{Coh}(C)$ un fascio, 
		 	possiamo calcolare il polinomio di Hilbert di $F$ rispetto a un fibrato $H$ di classe
		 	$h \in H^{2}(C;\Z) \simeq \Z$ fissato grazie a \textbf{Hirzebrich-Riemann-Roch} come
		 	\begin{align*}
		 		P_{H}(F)(n) &= \int_{X} \ch(F \otimes \Oo_{X}(nH)) \smile \td(X) \\
		 		&= \int_{X} \ch(F) \smile \ch(\Oo_{X}(nH)) \smile \td(X) \\
		 		&= \int_{X} (\rk(F), \deg(F)) \smile (1,hn) \smile (1,1-g) \\
		 		&= \int_{X} (\rk(F), \rk(F) hn + \deg(F)) \smile (1,1-g) \\
		 		&= \int_{X} (\rk(F), \rk(F) hn + \deg(F) + (1-g)\rk(F))  \\
		 		&=  h\rk(F) n  + \deg(F) + (1-g)\rk(F)\,,
		 	\end{align*}
		 \end{rmk}
		 e quindi il polinomio di Hilbert ridotto è
		 \begin{equation*}
		 	p_{H}(F)(n) =  n  + \frac{\deg(F)}{h\rk(F)} + \frac{(1-g)}{h} 
		 	= n + \frac{\mu(F)}{h} + costante\,.
		 \end{equation*}
		 
		 In dimensione più alta, però, la classe di Todd $\td(X)$ diventa più complicata
		 e, nonostante la pendenza $\mu(F)$ compaia all'interno di $p_{H}(F)$,
		 non possiamo dedurre un'equivalenza come in questo caso, ma le disuguaglianze strette
		 \begin{equation*}
		 	\mu\text{-stabilità} \implies H\text{-stabilità} 
		 	\implies H\text{-semistabilità} \implies \mu\text{-semistabilità}\,.
		 \end{equation*}
		 A questo punto, per studiare la stabilità in dimensione alta, sono state seguite
		 più strade:
		 \begin{itemize}
		 	\item studiare i \textbf{fasci twistati}, sempre importanti per la fisica
		 	per le loro relazioni con i \emph{B-fields};
		 	\item studiare la \textbf{stabilità twistata} di fasci, i.e. presa una
		 	classe $B \in H^{1,1}(X) \cap H^{2}(X;\R)$, si rifanno i conti della pendenza
		 	con il carattere di Chern twistato
		 	\begin{equation*}
		 		\ch^{B}(F) := \ch(F) \smile e^{-B}\,;
		 	\end{equation*}
		 	\item studiare le \textbf{condizioni di stabilità di Bridgeland},
		 	che offrono un punto di vista completamente nuovo
		 	e un formalismo non ancora studiato, in quanto vengono considerati
		 	non più fasci, ma complessi $E^{\bullet} \in \Db{X}$.
		 \end{itemize}
		 
		 
		 
		 
		 
		 
		 
		 
		 
		 
    	
    		
		\lecture[.]{2025-10-07}
		
		\section{Condizioni di stabilità di Bridgeland}
		
			La scorsa volta abbiamo motivato l'interesse fisico dello studio della
			$\mu$-stabilità, in quanto legato alle metriche Hermite-Einstein;
			passando alla teoria delle stringhe, si fa qualcosa di leggermente diverso
			che coinvolge le categorie derivate, alla maniera di Bridgeland.
			Per definire le sue condizioni di stabilità, si possono adottare
			due punti di vista:
			\begin{itemize}
				\item[i)] usando i cuori di $t$-strutture, cioè definizione su categorie abeliane;
				spesso questo approccio viene detto \textbf{condizioni di stabilità di King};
				\item[ii)] quello originale di Bridgeland, in cui vengono definite per 
				categorie triangolate qualsiasi tramite la nozione di \textbf{slicing};
				la definizione è lampante, ma il formalismo è più brutto.
			\end{itemize}
			
			\subsection{i) Stabilità di King}			
			
			Sia $\Aa$ una categoria abeliana, e denotiamo con $K_{0}(\Aa)$ il suo 
			\textbf{gruppo di Grothendieck}.
			\begin{df}\label{df:Z}
				Una \textbf{funzione di stabilità} su $\Aa$ è una funzione $Z : K_{0}(\Aa) \to \C$
				tale che:
				\begin{itemize}
					\item $Z$ è un omomorfismo di gruppi;
					\item per ogni $E \in \Aa$, si ha $\Im Z([E]) \ge 0$;
					\item se $E \ne 0$ ha $\Im Z([E]) = 0$, allora $\Re ([E]) < 0$.
				\end{itemize}
			\end{df}
			Il formalismo \emph{fa schifo} perché, a colpo d'occhio, non è affatto chiaro
			che relazione abbiano queste nozioni con quella di stabilità di fibrati vettoriali
			che abbiamo visto nelle motivazioni. Il seguente esempio fa luce su questo fatto:
			
			\begin{ex}\label{ex:curve-bridgeland}
				Sia $C$ una curva proiettiva liscia, $\Aa = \Coh(C)$, allora $K_{0}(\Aa) = K_{0}(C)$.
				Poniamo
				\begin{equation}
					Z : K_{0}(C) \longrightarrow \C\,, \quad Z([E]) := -\deg(E) + i \rk(E)\,.
				\end{equation}
				Per additività di grado e rango, $Z$ è un omomorfismo di gruppi ben definito;
				inoltre $\Im Z = \rk \ge 0$ e se $\Im Z([E]) = \rk(E) = 0$,
				allora $E$ è di torsione, quindi ha grado positivo $\deg(E) > 0$,
				da cui $\Re ([E]) < 0$. Questo mostra che $Z$ è una funzione di stabilità.
			\end{ex}
			
			Perché rango e grado abbiano senso in questa combinazione all'interno
			di $\C$ è ancora motivato dalla fisica: nei termini della teoria delle stringhe,
			la funzione $Z$, spesso chiamata \textbf{funzione di carica centrale},
			rappresenta la carica di particelle sulle $D$-brane. Boh.
			
			\begin{df}
				Data $Z$ una funzione di stabilità su $\Aa$ e $E \in \Aa$, poniamo
				\begin{equation}
					r_{Z}(E) := \Im Z([E])\,, \quad d_{Z}(E) := -\Re Z([E])\,,
					\quad \mu_{Z}(E) := \frac{d_{Z}(E)}{r_{Z}(E)}\,,
				\end{equation}
				chiamate, rispettivamente, \textbf{$Z$-rango}, \textbf{$Z$-grado} e 
				\textbf{$Z$-pendenza} di $E$.
			\end{df}
			
			\begin{rmk}
				La presenza del ``$-$'' nella definizione della funzione di carica centrale
				serve perché, se $r_{Z}(E)=0$, allora $d_{Z}(E) > 0$, e quindi
				un fascio con rango nullo ha pendenza infinita (positiva) $\mu_{Z}(E) = +\infty$.
				Inoltre, le condizioni di \textbf{\Cref{df:Z}} permettono di definire $Z$
				per ogni oggetto non-zero della categoria abeliana $\Aa$,
				e $0 \in \Aa$ è l'\textbf{unico} oggetto per cui $Z(0) = 0$.
			\end{rmk}
			
			\begin{df}\label{df:Z-ss}
				Un oggetto $E \in \Aa$ è detto \textbf{$Z$-semistabile} se,
				per ogni sottoggetto $0 \ne F \subsetneq E$, si ha
				\begin{equation*}
					\mu_{Z}(F) \le \mu_{Z}(E)\,.
				\end{equation*}
			\end{df}
			
			C'è un grande \textbf{problema} quando da $C$ curva andiamo a considerare
			$X$ una varietà proiettiva liscia di dimensione $n \ge 2$:
			preso $H \in Amp(X)$ e posti $A = \Coh(X)$ e $K_{0}(\Aa) = K_{0}(X)$,
			uno sarebbe tentato di considerare la funzione:
			\begin{equation}
				Z_{H} : K_{0}(X) \longrightarrow \C\,, \quad
				Z_{H}([E]) := -(c_{1}(E) \cdot H^{n-1}) + i \rk(E) H^{n}\,.
			\end{equation}
			Se questa fosse una funzione di stabilità, allora $\mu_{Z} = \mu_{H}$ sarebbe
			la solita pendenza rispetto a $H$, 
			ma purtroppo non soddisfa le condizioni in \textbf{\Cref{df:Z}}:
			infatti, se $E$ è un fascio supportato almeno in codimensione $2$, allora
			$Z_{H}(E) = 0$...
			Però potrei aver scelto male la mia funzione $Z$. 
			In realtà, c'è un problema più serio che non permette di ottenere mai
			funzioni di stabilità su $\Coh(X)$ in dimensione $n > 1$.
			\begin{df}
				Una funzione di stabilità $Z : K_{0}(X) \to \C$ è detta \textbf{numerica}
				se esiste un omomorfismo di gruppi $Z': H^{2*}(X;\Q) \to \C$ tale che
				$Z = Z'\circ \ch$, cioè il seguente triangolo commuti:
				\begin{equation*}
					\begin{tikzcd}
					K_{0}(X) \ar[dr, "Z"'] \ar[rr,"\ch"] && H^{2*}(X;\Q) \ar[dl, "Z'"] \\
					& \C & \,.
					\end{tikzcd}
				\end{equation*}
			\end{df}
			\begin{ex}
				Se $X=C$ è una curva, allora $Z = -\deg + i \rk$ è una condizione numerica,
				infatti $\ch: K_{0}(C) \to H^{2*}(C;\Q) = H^{0}(C; \Q) \oplus H^{2}(C; \Q)$ è
				proprio 
				$$\ch(E) = (\rk(E), \deg(E))\,,$$
				quindi basta porre $Z'(\alpha,\beta) := -\beta + i \alpha$.
			\end{ex}
			
			\begin{lemma}[\textbf{Toda}]\label{lemma:Toda-num}
				Se $\dim X \ge 2$ non esistono funzioni di stabilità numeriche su $\Coh(X)$.
			\end{lemma}
			Essenzialmente, la ragione per cui si ha questo fallimento è che, in dimensione maggiore
			di $1$, il carattere di Chern scontrato con $H$ \textbf{non} è più lineare.
			
			In tutto questo stiamo ancora parlando di \emph{funzioni}, 
			e non di \emph{condizioni di stabilità}, e abbiamo già capito che è difficile in generale
			fabbricare le prime.
			
			\begin{df}\label{df:stab}
				Una funzione di stabilità $Z$ su $\Aa$ è detta \textbf{condizione di stabilità}
				se, per ogni oggetto $E \in \Aa$, esiste una \textbf{filtrazione di Harder-Narasimhan}
				per $E$, i.e. una filtrazione
				\begin{equation*}
		 			0 = E_{0} \subsetneq E_{1} \subsetneq \dots \subsetneq E_{l-1} \subsetneq E_{l} = E
		 		\end{equation*}
		 	tale che, per ogni $0 < i \le l$, ogni fattore $E_{i}/E_{i-1}$ sia $\mu_{Z}$-semistabile e 
		 	si abbia pendenza decrescence $\mu(E_{i}/E_{i-1}) > \mu(E_{i+1}/E_{i})$.
			\end{df}
			
			Dimostrare l'esistenza della filtrazione di HN, in generale, non è un compito
			semplice, ma se si pongono condizioni buone sulla $Z$ che andiamo a costruire,
			allora abbiamo speranza di ottenere la filtrazione di HN in maniera automatica.
			Ad esempio, se so già che la categoria $\Aa$ ha oggetti che ammettono filtrazioni
			infinite, allora ho poche speranze di ottenere una condizione di stabilità da una $Z$
			qualsiasi.
			\begin{df}
				Una categoria abeliana $\Aa$ è detta \textbf{noetheriana} se, per ogni $E \in \Aa$
				e ogni catena ascendente
				\begin{equation*}
		 			0 = E_{0} \subset E_{1} \subset \dots \subset E_{l} \subset  \dots \subset E
		 		\end{equation*}
		 		ammette un $n_{0} \in \N$ tale che $E_{n} = E_{n_{0}}$, per ogni $n \ge n_{0}$.
			\end{df}
			
			\begin{thm}\label{thm:discrete-Z}
				Se $\Aa$ è abeliana e noetheriana e $Z : K_{0}(X) \to \C$ è una funzione
				di stabilità tale che $r_{Z} : K_{0}(\Aa) \to \R$ ha immagine discreta,
				allora $Z$ è una condizione di stabilità.
			\end{thm}
			
			Quindi, in fin dei conti, è sufficiente fabbricare funzioni di stabilità.
			
			\subsection{ii) Stabilità di Bridgeland originale}
			
			Sia $\Dd$ una categoria triangolata.
			\begin{df}\label{df:t-struttura}
				Una \textbf{$t$-struttura} su $\Dd$ è una coppia $(\Dd^{\le 0}, \Dd^{\ge 0})$,
				con $\Dd^{\le 0}, \Dd^{\ge 0}$ due sottocategorie piene di $\Dd$,
				non necessariamente triangolate, tali che:
				\begin{itemize}
					\item se $A \in \Dd^{\le 0}$ e  $B \in \Dd^{\ge 0}$,
					allora $A[1] \in \Dd^{\le 0}$ e  $B[-1] \in \Dd^{\ge 0}$;
					
					\item se $A \in \Dd^{\le 0}$ e  $B \in \Dd^{\ge 0}$, allora per ogni $i>j$
					si ha $\Hom_{\Dd}(A[i],B[j]) =0$;
					
					\item per ogni $E \in \Dd$ esiste un triangolo esatto
					\begin{equation*}
					A \longrightarrow E \longrightarrow B \longrightarrow A[1]
					\end{equation*}
					con $A \in \Dd^{\le 0}$ e  $B \in \Dd^{\ge 0}$.
				\end{itemize}
			\end{df}
			
			\begin{ex}
				Per ogni categoria abeliana $\Aa$, esiste una $t$-struttura canonica 
				$(\Dd^{\le 0}, \Dd^{\ge 0})$ su $\Db{\Aa}$ definita da
				\begin{align*}
					\Dd^{\le 0} := \Set{ E \in \Db{\Aa} \, | \, \forall_{j > 0} \, H^{j}(E) = 0 }\,, \\
					\Dd^{\ge 0} := \Set{ E \in \Db{\Aa} \, | \, \forall_{j \le 0} \, H^{j}(E) = 0 }\,.
				\end{align*}
				Chiameremo questa la \textbf{$t$-struttura standard}.
			\end{ex}
			
			\begin{df}
				Una $t$-struttura $(\Dd^{\le 0}, \Dd^{\ge 0})$ su $\Dd$
				è \textbf{limitata} se, per ogni $E \in \Dd$, esistono $m,n \in \Z$
				tale che $E[n] \in \Dd^{\le 0}$ e $E[m] \in \Dd^{\ge 0}$.
			\end{df}
			
			\begin{thm}
				Data una $t$-struttura $(\Dd^{\le 0}, \Dd^{\ge 0})$ su una categoria triangolata 
				$\Dd$, la sottocategoria $\Dd^{\heartsuit} := \Dd^{\le 0} \cap \Dd^{\ge 0}$
				è una categoria abeliana, detta \textbf{cuore} della $t$-struttura.
			\end{thm}
			
			\begin{ex}
				Se $(\Dd^{\le 0}, \Dd^{\ge 0})$ è la $t$-struttura standard in $\Db{\Aa}$
				dell'esempio precedente, allora $\Dd^{\heartsuit}$ è equivalente a $\Aa$.
			\end{ex}
			
			\begin{thm}[\textbf{Bridgeland}]\label{thm:heart}
				Sia $\Dd$ una categoria triangolata, una sottocategoria $\Aa \subset \Dd$ abeliana
				è il cuore di una $t$-struttura se e solo se valgono le seguenti tre condizioni:
				\begin{itemize}
					\item[i)] se $A, B \in \Aa$, per ogni $i>j$
					si ha $\Hom_{\Dd}(A[i],B[j]) =0$;
					
					\item[ii)] per ogni $E \in \Dd$,
					 esistono interi $h_{1}, \dots, h_{m} \in \Z$,
					 oggetti $E_{1}, \dots, E_{m} \in \Dd$ e $A_{1}, \dots, A_{m} \in \Aa$
					 tali che si abbia il diagramma di triangoli esatti
					 \begin{equation*}
					 	\begin{tikzcd}
0 = E_{0} \arrow[r] & E_{1} \arrow[r] \arrow[d]               & E_{2} \arrow[r] \arrow[d]               & \dots \arrow[r] & E_{m-1} \arrow[r] \arrow[d]                 & E_{m} = E \arrow[d]                     \\
                    & {A_{1}[h_{1}]} \arrow[lu, "+1", dashed] & {A_{2}[h_{2}]} \arrow[lu, "+1", dashed] &                 & {A_{m-1}[h_{m-1}]} \arrow[lu, "+1", dashed] & {A_{m}[h_{m}]} \arrow[lu, "+1", dashed]
					 	\end{tikzcd}
					 \end{equation*}
					 che prende il ruolo della \textbf{filtrazione di Harder-Narasimhan}
					 di $E$ in $\Dd$.
				\end{itemize}
			\end{thm}
			
			Sia $\Dd$ una categoria triangolata. Diamo ora una nozione molto simile
			a quella appena enunciata nel \textbf{\Cref{thm:heart}}.
			\begin{df}\label{df:slicing}
				Uno \textbf{slicing} di $\Dd$ è una famiglia $\Pp = \Set{\Pp(\phi) | \phi \in \R}$ 
				di sottocategorie piene di $\Dd$ tali che:
				\begin{itemize}
					\item[i)] per ogni $\phi \in \R$ vale $\Pp(\phi)[1] = \Pp(\phi + 1)$;
					
					\item[ii)] per ogni coppia di numeri reali $\phi_{1} > \phi_{2}$
					e per ogni $A \in  \Pp(\phi_{1})$ e $B \in  \Pp(\phi_{2})$,
					si ha $\Hom_{\Dd}(A,B) = 0$;
					
					\item[iii)] per ogni $E \in \Dd$, 
					esistono numeri reali $\phi_{1} > \dots > \phi_{m}$ e
					 oggetti $E_{1}, \dots, E_{m} \in \Dd$ e $A_{1}, \dots, A_{m} \in \Dd$,
					 con ogni $A_{i} \in \Pp(\phi_{i})$
					 tali che si abbia il diagramma di triangoli esatti
					 \begin{equation*}
					 	\begin{tikzcd}
0 = E_{0} \arrow[r] & E_{1} \arrow[r] \arrow[d]               & E_{2} \arrow[r] \arrow[d]               & \dots \arrow[r] & E_{m-1} \arrow[r] \arrow[d]                 & E_{m} = E \arrow[d]                     \\
                    & {A_{1}} \arrow[lu, "+1", dashed] & {A_{2}} \arrow[lu, "+1", dashed] &                 & {A_{m-1}} \arrow[lu, "+1", dashed] & {A_{m}} \arrow[lu, "+1", dashed]
					 	\end{tikzcd}
					 \end{equation*}
					 detta \textbf{filtrazione di Harder-Narasimhan} di $E$ per $\Pp$.
				\end{itemize}
				Gli oggetti in $\Pp(\phi)$ sono detti di \textbf{fase $\phi$}.
			\end{df}
			
			\begin{ex}
				Sia $C$ una curva proiettiva liscia e $\Dd = \Db{C}$.
				Per ogni $\phi \in \R$, poniamo
				\begin{equation*}
					\Pp(\phi) := \langle E \text{ fascio } \mu\text{-semistabile, con } 
					\mu(E) = \phi \rangle\,.
				\end{equation*}
				Allora la collezione di queste $\Pp(\phi)$ è uno slicing di $\Db{C}$.
			\end{ex}
			
			Data $X$ una varietà proiettiva liscia, fissiamo ora $\Dd = \Db{X}$
			e un reticolo $\Lambda$ finitamente generato, con un omomorfismo di gruppi
			suriettivo
			\begin{equation*}
				\nu : K_{0}(X) \twoheadrightarrow \Lambda\,.
			\end{equation*}
			Noi siamo interessati al caso $\Lambda = K_{num}(X) := K_{0}(X)/T$,
			dove 
			$$T := \Set{ a \in K_{0}(X) \, | \, \forall_{b} \, \chi(a,b) = 0}$$
			e $K_{num}(X)$ è dotato del pairing $(a,b) := - \chi(a,b)$;
			il ``$-$'' è messo lì per questione
			di compatibilità con il pairing tra vettori di Mukai.
			
			\begin{ex}
				Se $X$ è una superficie K3, allora il vettore di Mukai
				\begin{equation}\label{k3:mukai-vector}
					\cat{v} : K_{num}(X) \longrightarrow H^{2*}(X;\Z)\,,
					\quad \cat{v}([E]) = (\rk(E), c_{1}(E), \rk(E) + \ch_{2}(E))
				\end{equation}
				è un'isometria.
			\end{ex}
			
			\begin{df}
				Una \textbf{condizione di stabilità di Bridgeland } su $X$ 
				è una coppia $\sigma = (\Pp,Z)$, dove $\Pp$ è uno slicing di $\Db{X}$
				e $Z:\Lambda \to \C$ tali che:
				\begin{itemize}
					\item[i)] $Z$ è un omorfismo di gruppi;
					\item[ii)] se $E \in \Pp(\phi)$, allora $Z(\nu(E)) = \lambda e^{i\pi\phi}$,
					con $\lambda > 0$ dipendente da $E$;
					\item[iii)] vale la \textbf{proprietà di supporto}:
					\begin{equation}\label{df:support}
						C_{\sigma} := \inf \Set{\frac{\lvert Z(\nu(E)) \rvert}{\lVert \nu(V) \rVert} \, |\,
						0 \ne E \in \Pp(\phi), \, \phi \in \R } > 0 \,.	
					\end{equation}					 
				\end{itemize}
				Gli oggetti di $\Pp(\phi)$ vengono detti $\sigma$-semistabili.
			\end{df}
			
			\begin{rmk*}
				La proprietà di supporto è stata introdotta più avanti
				da \textbf{Kontsevich} e \textbf{Soibelman}, 
				motivata dal fatto che, il numeretto $C_{\sigma}$ definito in \eqref{df:support}
				è in qualche modo legato al \textbf{discriminante} del fascio $E$.
				Richiedendo questa condizione, si garantisce la validità
				di una disuguaglianza del tipo di \textbf{Bogomolov},
				disuguaglianza soddisfatta dai fasci semistabili e che quindi
				è auspicabile che abbiano tutti i fasci \emph{stabili} in un qualche senso.
			\end{rmk*}
			
			\begin{thm}\label{thm:equivalent-df-stab}
				Sia $X$ una varietà proiettiva liscia e $\Dd = \Db{X}$.
				Dare una condizione di stabilità di Bridgeland $(\Pp,Z)$ su $\Db{X}$ equivale
				a dare una coppia $(\Aa, Z)$, con $\Aa$ abeliana e $Z$ condizione di stabilità
				su $\Aa$ nel senso di \textbf{\Cref{df:stab}} 
				e tale che la \textbf{proprietà di supporto}
				è verificata:
				\begin{equation}\label{supp-abeliana}
						\inf \Set{\frac{\lvert Z(\nu(E)) \rvert}{\lVert \nu(E) \rVert} \, |\,
						0 \ne E \in \Aa \text{ è $Z$-semistabile} } > 0 \,.	
					\end{equation}	
				\begin{proof}[Idea della costruzione]
					Data $(\Pp,Z)$, si pone $\Dd^{\le 0} := \bigcup_{\phi > 0} \Pp(\phi)$ e 
					$\Dd^{\ge 0} := \bigcup_{\phi \le 1} \Pp(\phi)$. Allora la coppia
					$(\Dd^{\le 0}, \Dd^{\ge 0})$ definiscono una $t$-struttura limitata su $\Dd$,
					il cui cuore $\Aa := \Dd^{\heartsuit} = \Pp((0,1])$ ha la filtrazione
					di Harder-Narasimhan rispetto a $Z$, dove
					\begin{equation*}
						Z(E) := \lambda \cos(\pi \phi) + i \lambda \sin(\pi \phi)\,. \qedhere
					\end{equation*}
				\end{proof}
			\end{thm}	
    	
    	
	\lecture[Costruzione dello spazio delle condizioni di stabilità di Bridgeland.
	Contestualizzazione dei problemi di rappresentabilità. Cenni di wall-crossing.]{2025-10-14}
		
		\section{Lo spazio delle condizioni di stabilità}
		
		Sia $X$ una varietà proiettiva liscia di dimensione $\dim X = n$,
		$\Lambda$ un reticolo di rango finito fissato e $\nu : K_{0}(X) \to \Lambda$ un morfismo
		surgettivo di gruppi. La scorsa volta abbiamo definito le 
		\textbf{condizioni di stabilità di Bridgeland} rispetto a $\Lambda$ e $\nu$ 
		in due modi equivalenti, vedi \textbf{\Cref{thm:equivalent-df-stab}}.
%		vedi \textbf{\Cref{df:stab}} e \textbf{\Cref{df:slicing}}.
		
		\begin{oss}
			Per una curva proiettiva liscia $C$ c'è sempre almeno una condizione di stabilità
			$\sigma_{C} = (\Coh(C), Z)$, dove $Z([E]) := -\deg(E) + i \rk(E)$.
			La categoria dei fasci $\Coh(C)$ è il cuore della $t$-struttura standard di $\Db{C}$
			e $Z$ è una condizione di stabilità su $\Coh(C)$. Preso $\Lambda = K_{num}(C)$,
			tensorizzando con $\R$ possiamo vedere che il carattere di Chern dà un isomorfismo
			\begin{equation*}
				\Lambda \otimes_{\Z} \R \longrightarrow H^{0}(C;\R) \oplus H^{2}(C; \R) \simeq \R^{2}\,,
				\quad \ch([E]) = (\rk(E), \deg(E))\,.
			\end{equation*}
			Se dotiamo $\Lambda \otimes_{\Z} \R$ della norma euclidea, allora notiamo che 
			per ogni $0 \ne E \in \Coh(C)$ vale
			\begin{equation*}
				\frac{\lvert Z(E) \rvert}{\lVert \nu(E) \rVert} = 1 \quad \implies \quad
				\inf \Set{\frac{\lvert Z(E) \rvert}{\lVert \nu(E) \rVert} \, | \, 0 \ne E \in \Coh(C)} = 1
			\end{equation*}
			e quindi la proprietà del supporto in \eqref{supp-abeliana} è verificata, 
			concludendo la dimostrazione che $\sigma_{C}$ è una condizione di stabilità di Bridgeland su $C$.
		\end{oss}
		
		Quindi saremmo tentati di dire che le condizioni di stabilità di Bridgeland 
		sono generalizzazioni della stabilità... Ma non è proprio così:
		\emph{le condizioni di stabilità di Bridgeland 
		sono generalizzazioni della stabilità sulle \textbf{curve}}.
		
		\begin{notation*}
			Se $\sigma = (\Pp, Z)$ è una condizione di stabilità nel senso 
			della \textbf{\Cref{df:slicing}}, per ogni $E \in \Dd$, 
					esistono numeri reali $\phi_{1} > \dots > \phi_{m}$, detti \textbf{fasi},
					univocamente determinati dalla filtrazione di HN in iii) 
					della \textbf{\Cref{df:slicing}}; noi saremo interessati
					principalmente alla fase massima e quella minima, che denotiamo 
					rispettivamente come
					\begin{equation*}
						\phi^{+}_{\sigma} := \phi_{1}\,, \quad \phi^{-}_{\sigma} := \phi_{m}\,.
					\end{equation*}
		\end{notation*}
		
		\begin{notation*}
			Se $\sigma = (\Pp, Z)$, allora andiamo a dotare $Z \in \Hom(\Lambda, \C)$
			della \textbf{norma operatoriale} ereditata da $\Hom(\Lambda \otimes_{\Z} \R, \C)$,
			cioè
			\begin{equation*}
				\lVert Z \rVert := \sup \Set{\frac{\lvert Z(\nu(E)) \rvert}{\lVert \nu(E) \rVert} \, | \, \phi \in \R \text{ e } 0 \ne E \in \Pp(\phi)}\,.
			\end{equation*}
			Notiamo che la formula nella definizione è la stessa che 
			definisce la costante $C_{\sigma}$ in \eqref{df:support}. 
%			garantita dalla proprietà di supporto
%			\begin{equation}
%				C_{\sigma} := \inf \Set{\frac{\lvert Z(\nu(E)) \rvert}{\lVert \nu(E) \rVert} \, | \, \phi \in \R \text{ e } 0 \ne E \in \Pp(\phi)} > 0\,.
%			\end{equation}
			Richiedere che $C_{\sigma} > 0$ è equivalente all'esistenza di 
			una forma bilineare simmetrica 
				\begin{equation*}
					Q : (\Lambda \otimes_{\Z} \R) \times (\Lambda \otimes_{\Z} \R) \longrightarrow \R
				\end{equation*}
			tale che:
			\begin{enumerate}
				\item se $E$ è $\sigma$-semistabile, allora $Q(\nu(E), \nu(E)) \ge 0$;
				\item se $\alpha \in \Lambda$ è tale che $Z(\alpha)=0$, allora $Q(\alpha, \alpha)<0$.
			\end{enumerate}
		\end{notation*}
		
		Data $X$ una varietà proiettiva liscia e $\Lambda$ un reticolo di rango finito
		con $\nu : K_{0}(X) \to \Lambda$ surgettiva, poniamo
		\begin{equation*}
			Stab_{\Lambda, \nu}(X) := \Set{ \sigma = (\Pp, Z) \, | \, \sigma \text{ è una condizione di stabilità di Bridgeland } }\,.
		\end{equation*}	
		Per esperienza con il caso della stabilità classica,
		si sospetta che $Stab$ non debba essere solo un insieme, 
		ma in realtà abbia una
		struttura in più.
		Supponiamo (anche se non è così) che la stabilità di Bridgeland
		generalizzi la $\mu$-stabilità vista nella prima lezione: 
		data $X$ una superficie, per la $\mu$-stabilità
		serve una \textbf{classe di K\"ahler} $\omega$, con cui si definisce
		\begin{equation*}
			\mu_{\omega}(E) := \frac{c_{1}(E) \cdot \omega^{n-1}}{\rk(E)}\,.
		\end{equation*}
		Ma allora possiamo identificare ``$Stab_{\mu}(X) = Kah(X)$'' con il cono di K\"ahler
		della varietà, che è una varietà topologica! Quindi, se stiamo davvero generalizzando
		questa idea, vorremmo che le condizioni di stabilità formino una varietà.
		In realtà, la topologia su $Stab$ è anche indotta da una \textbf{distanza generalizzata}
		\begin{equation*}
			d_{\Lambda,\nu} : Stab_{\Lambda, \nu}(X) \times Stab_{\Lambda, \nu}(X) \longrightarrow
			\R \cup \{ + \infty \}\,,
		\end{equation*}
		dove la distanza tra due condizioni $\sigma_{1}$ e $\sigma_{2}$ è data da
		\begin{equation}\label{df:dist}
			d_{\Lambda,\nu}(\sigma_{1}, \sigma_{2}) = 
			\sup \Set{ \lvert \phi_{1}^{+}(E) - \phi_{2}^{+}(E) \rvert, \, 
			\lvert \phi_{1}^{-}(E) - \phi_{2}^{-}(E) \rvert,\,
			\lVert Z_{1}(E) - Z_{2}(E) \rVert}\,.
		\end{equation}
		
		\begin{thm}[\textbf{di deformazione di Bridgeland}]\label{thm:defo}
			Se $Stab_{\Lambda, \nu}(X)$ è non vuoto 
			ed è dotata della topologia indotta da $d_{\Lambda,\nu}$,
			allora la mappa
			\begin{equation}\label{local-homeo}
				\Zz : Stab_{\Lambda, \nu}(X) \longrightarrow \Hom(\Lambda \otimes \R, \C)\,,
				\quad \sigma = (\Pp, Z) \longmapsto Z 
			\end{equation}
			è un omeomorfismo locale. In particolare, $Stab_{\Lambda, \nu}(X)$ è
			una varietà complessa di dimensione $\rk(\Lambda)$.
		\end{thm}
		
		Il fatto che lo spazio delle condizioni di stabilità di Bridgeland (su una qualsiasi
		categoria triangolata) sia \textbf{non vuoto} è un problema non banale,
		e ci sono articoli che studiano quando è vuoto o no; altre proprietà
		topologiche di questo spazio non sono note in generale.
		
		\begin{oss}
			Ci sono due azioni di gruppo su $Stab_{\Lambda, \nu}(X)$:
			\begin{enumerate}[label=\roman*)]
				\item il gruppo 
				$\Aut_{\Lambda,\nu}(\Db{X}) := \Set{F : \Db{X} \to \Db{X} \, | \,
					\nu \circ F^{K} = \nu }$ delle autoequivalenze che preservano
					$\nu$, cioè tali che commuti il triangolo
				\begin{equation*}
					 \begin{tikzcd}
						{K_{0}(X)} \ar[rr, "F^{K}"] \ar[dr, "\nu"'] & & {K_{0}(X)} \ar[dl, "\nu"] \\
											&{\Lambda}& \,, 
					\end{tikzcd}
				\end{equation*}
				agisce a sinistra tramite
				\begin{align*}
					\lambda : \Aut_{\Lambda,\nu}(\Db{X}) \times Stab_{\Lambda, \nu}(X) \longrightarrow
					Stab_{\Lambda, \nu}(X)\,, \\
					\quad \lambda(F, \sigma = (\Pp, Z)) = F(\sigma) := (F(\Pp), F(Z)) \,,
				\end{align*}
				dove $F(\Pp)$ è lo slicing dato da $F(\Pp)(\phi) := F(\Pp(\phi))$, 
				per ogni $\phi \in \R$, mentre la carica centrale si ottiene per precomposizione
				$F(Z) := Z \circ F^{K}$.
				Si verifica facilmente che la filtrazione di HN esiste, 
				in quanto la filtrazione in $F(\sigma)$ di un oggetto $E$ è
				data dall'immagine attraverso $F$ della filtrazione in $\sigma$ di $F^{-1}(E)$;
				
				\item il rivestimento universale $\widetilde{\GL}_{2}^{+}(\R)$ di $\GL_{2}^{+}(\R)$
				ha una descrizione esplicita come
				\begin{equation}
					 \widetilde{\GL}_{2}^{+}(\R) \simeq 
					 \Set{ (T,f) \, | \substack{\, T \in \GL_{2}^{+}(\R), \, f:\R \to \R \text{ crescente tale che } \\ f(\phi +1) = f(\phi) + 1 
					 \text{ e } T\vert_{S^{1}} = f\vert_{\R/\Z} } 
					 }\,.
				\end{equation}
				Questo gruppo agisce a destra su $\sigma = (\Pp,Z) \in Stab_{\Lambda,\nu}(X)$ 
				tramite la seguente formula: dato $a = (T,f) \in \widetilde{\GL}_{2}^{+}(\R)$,
				definiamo $\sigma \cdot a = \sigma_{a} = (\Pp_{a},Z_{a})$ come la coppia
				\begin{align*}
					\Pp_{a} := \Set{\Pp(f(\phi)) \, | \, \phi \in \R }\,, \quad Z_{a} := T^{-1} \circ Z\,.
				\end{align*}
			\end{enumerate}
		\end{oss}
		
		\begin{prop}
			Se $X = C$ è una curva proiettiva liscia, allora $Stab_{\Lambda, \nu}(C)$
			è noto:
			\begin{itemize}
				\item se $C = \PP^{1}_{\C}$, allora $Stab_{\Lambda, \nu}(\PP^{1}_{\C}) \simeq \C^{2}$,
				dimostrato da \textbf{Okada} in \cite{OKADA};
				\item se $C$ ha genere positivo, 
				allora $Stab_{\Lambda, \nu}(C) = \sigma_{C} \cdot \widetilde{\GL}_{2}^{+}(\R)
				\simeq \HH \times \C$.
			\end{itemize}
		\end{prop}
		Quindi stiamo sì generalizzando le nozioni di stabilità sulle curve... ma mica tanto!
		Se $\dim X \ge 2$, allora il problema di caratterizzare $Stab_{\Lambda, \nu}(X)$ 
		è ancora aperto e si hanno solo risultati
		parziali nel caso delle superfici K3, delle abeliane e poco altro.
		
		
		
		\subsection{Spazi di moduli}
			
			Data $X$ una varietà proiettiva liscia e $B$ uno schema localmente di tipo finito su $\C$.
			\begin{df}
				Un oggetto $E \in D(\cat{QCoh}(X \times B))$ è \textbf{$B$-perfetto} se,
				per ogni $b \in B$, esiste un intorno aperto $U \subset B$  di $b$  tale che
				$E\vert_{\operatorname{pr}^{-1}_{B}(U)}$ è isomorfo a un complesso finito
				di fasci $U$-piatti.
			\end{df}
			
			\begin{oss}
				Quando $B = \Spec \C$, un oggetto perfetto è quasi-isomorfo
				a un complesso limitato di fibrati vettoriali.			
			\end{oss}
			
			Questa nozione permette di definire il funtore
			\begin{equation*}
				\xM : \underline{Sch} \longrightarrow \cat{Grp}\,, \quad 
				B \longmapsto \Set{E \in D(\cat{QCoh}(X \times B)) \, | \, \substack{E 
				\text{ è } B\text{-perfetto tale che } \\
				\forall_{b \in B} \, \forall_{i<0}\, \Ee xt^{i}(E_{b},E_{b}) =0} }
			\end{equation*}
			
			\begin{thm}[\textbf{Lieblich}]
				$\xM$ è uno stack di Artin localmente di tipo finito,
				localmente quasi separato e a diagonale separata.
			\end{thm}
			Per chi non ha idea di cosa sia uno stack, questo enunciato ha poco senso,
			ma il nocciolo è che $\xM$ si comporta bene, quasi come uno schema.
			Dentro $\xM$ possiamo andare a considerare il funtore che va a considerare
			i complessi \textbf{semplici}
			\begin{equation*}
				\xM_{spl} : \underline{Sch} \longrightarrow \cat{Grp}\,, \quad 
				B \longmapsto \Set{E \in \xM(B) \, | \, \forall_{b \in B} \, \End(E_{b}) \simeq \C }
			\end{equation*}
			che si verifica essere un sottostack aperto di $\xM$.
			Se si mette una relazione d'equivalenza $\sim$ a $\xM$ visto come insieme, 
			allora si \emph{fascifica lo stack} eliminandone gli automorfismi, 
			e si ottiene così un funtore
			\begin{equation*}
				\underline{M_{spl}} : \underline{Sch} \longrightarrow \cat{Set}\,, \quad 
				B \longmapsto \xM_{spl}(B)/\sim\,,
			\end{equation*}
			dove $E \sim E'$ se e solo se esiste $\Ll \in \Pic(B)$ 
			tale che $E \simeq E'\otimes \operatorname{pr}_{B}^{*}\Ll$.
			
			\begin{thm}[\textbf{Inaba}]
				Il funtore $\underline{M_{spl}}$ è rappresentabile
				da uno spazio algebrico (non esattamente uno schema) 
				localmente di tipo finito $M_{spl}$.
			\end{thm}
			
			Questo risultato è la base tecnica per poter parlare di spazi di moduli su $X$.
			Sia $\sigma = (\Pp,Z) \in Stab_{\Lambda, \nu}(X)$ e fissiamo $\alpha \in \Lambda$.
			Vorremmo parametrizzare gli oggetti $\sigma$-semistabili di fase $\phi$,
			e quindi che il nostro spazio dei moduli fosse
			\begin{equation*}
				\widehat{M}_{\sigma}(\alpha, \phi) := \Set{ E \in \Pp(\phi) \, | \, \nu(E) = \alpha}\,,
			\end{equation*}
			e analogamente denotiamo con $\widehat{M}_{\sigma}^{s}(\alpha, \phi)$ quello
			dei fasci $\sigma$-stabili.
			È lo spazio dei moduli che cerchiamo?
			Più precisamente, stiamo considerando i funtori $\xM^{s}_{\sigma}(\alpha, \phi) \subset \xM_{\sigma}(\alpha, \phi) \subset \xM_{spl}$, con
			\begin{equation*}
				\xM_{\sigma}(\alpha, \phi)(B) := \Set{ E \in \xM_{spl}(B) \, | \, \forall_{b \in B} \, E_{b} \in \widehat{M}_{\sigma}(\alpha, \phi) }\,,
			\end{equation*}
			analogamente $\xM_{\sigma}^{s}(\alpha, \phi)(B)$ considera quelli $\sigma$-stabili,
			e ci chiediamo se siano \textbf{aperti} di $\xM_{spl}$. 
			In generale, la \textbf{risposta non si sa}: in casi particolari,
			la risposta è affermativa grazie al lavoro di \textbf{Toda}.
			Se invece andassimo a considerare i funtori a valori in $\cat{Set}$
			\begin{equation*}
				\underline{M}^{s}_{\sigma}(\alpha, \phi) \subset \underline{M}_{\sigma}(\alpha, \phi) \subset \underline{M}_{spl}\,,
			\end{equation*}
			ci chiediamo: sono rappresentabili? In caso affermativo, 
			se sappiamo che gli stack associati sono aperti, allora abbiamo anche
			delle inclusioni aperte di schemi
			\begin{equation*}
				{M}^{s}_{\sigma}(\alpha, \phi) \subset {M}_{\sigma}(\alpha, \phi) \subset {M}_{spl}\,.
			\end{equation*}
			
			\begin{question*}
				Cosa sono punti di questi spazi di moduli $\widehat{M}_{\sigma}(\alpha, \phi)$?
				Parametrizzano classi di $S$-equivalenza?
				\begin{proof}[Risposta parziale]
					Negli esempi noti, la risposta è sempre sì:
					\begin{itemize}
						\item se $\dim X = 1$, allora abbiamo
						che ${M}_{\sigma}(\alpha, \phi)$ (risp. ${M}^{s}_{\sigma}(\alpha, \phi)$)
						è isomorfo allo spazio dei moduli di fasci 
						$\mu$-semistabili (risp. $\mu$-stabili) su $C$,
						quindi la risposta è affermativa;
						\item se $\dim X = 2$, allora si hanno dei risultati sulla rappresentabilità
						nei casi in cui $X$ sia $\PP^{2}, \PP^{1} \times \PP^{1}, Bl_{p}\PP^{2}$
						e in alcuni casi in cui $X$ è una superficie K3, Enriques o abeliana.
						In tutti questi casi, la risposta è affermativa.
					\end{itemize}
				\end{proof}
			\end{question*}
			
			\subsection{Wall-crossing}
				Dato $H$ un fibrato lineare ampio su $X$, 
				se $E$ è un fascio $\mu_{H}$-stabile, per ogni $0 \ne F \subsetneq E$, 
				possiamo riscrivere la disuguaglianza $\mu_{H}(F) < \mu_{H}(E)$ come
				\begin{align}\label{mu-walls}
					(\rk(E) c_{1}(F) - \rk(F) c_{1}(E)) \cdot H < 0\,. 
				\end{align}
				Quindi, per ogni sottofascio $F$ come sopra, possiamo definire il divisore su $X$
				\begin{align*}
					D_{E,F} := \rk(E) c_{1}(F) - \rk(F) c_{1}(E)\,,
				\end{align*}
				che per Hodge Index Theorem ha quadrato negativo; in particolare,
				si può dimostrare che esiste $\alpha > 0$ tale che $-\alpha \le D_{E,F}^{2} < 0$
				per ogni $F \subsetneq E$. Viceversa, 
				se esistesse un divisore $D$ tale che $D \cdot H = 0$, potrebbe esserci la possibilità
				che $D = D_{E,F}$ per un qualche fascio che destabilizza $E$: vado quindi a togliere
				dal cono ampio queste polarizzazioni, cioè vado a considerare il complementare dei 
				$D_{E,F}^{\perp}$ in $Amp(X)$, al variare di $F$. Questo arrangiamento
				di iperpiani suddivide il cono ampio in \textbf{camere} in cui il segno del pairing
				con i $D_{E,F}$ rimane costante: più precisamente, presa una camera 
				$\Cc \subset Amp(X) \setminus \bigcup_{0 \ne F \subsetneq E} D_{E,F}^{\perp}$,
				per ogni $H,H'\in \Cc$, allora
				\begin{align*}
					D_{E,F} \cdot H < 0 \quad \iff \quad D_{E,F} \cdot H' < 0\,.
				\end{align*}
				
				Un'idea analoga vale anche per lo spazio delle condizioni di stabilità di Bridgeland:
				infatti, ricordando che per una carica centrale $Z$
				 la $\Im Z$ rappresenta il `\emph{rango}', 
				 mentre la $-\Re Z$ fa le veci del `\emph{grado}',
				 possiamo rimpiazzare l'equazione $D_{E,F} \cdot H =0$ con
				 \begin{equation}
				 	\Im Z(v_{0}) \Re Z(\nu(E)) - \Im Z(\nu(E)) \Re Z(v_{0}) = 0\,, \quad \text{con }
				 	v_{0} \in \Lambda\,,
				 \end{equation}
				definendo in questo modo una sottovarietà reale di codimensione $1$ in $Stab_{\Lambda,\nu}(X)$. Si può dimostrare che l'insieme di queste sottovarietà, detti \textbf{muri numerici}, è localmente finita e divide $Stab_{\Lambda,\nu}(X)$ in camere, proprio come succedeva per il cono ampio.
				In particolare, la nozione di stabilità rimane costante in ogni camera:
				più precisamente, se $\Cc \subset Stab_{\Lambda,\nu}(X)$ è una camera e $\sigma, \sigma'\in \Cc$, allora
				\begin{align*}
					M_{\sigma}(X, \nu(E)) \simeq M_{\sigma'}(X, \nu(E))\,.
				\end{align*}

    		
	\lecture[Costruzione di condizioni di stabilità di Bridgeland per superfici, prendendo come esempio il caso K3. Tecnica di tilting per cuori di $t$-strutture per la costruzione di opportuni cuori per le condizioni di stabilità.]{2025-10-21}
	
		\section{Condizioni di stabilità su superfici}
			
		\begin{warn*}
			Questa lezione è stata tenuta da \textbf{Filippo Papallo} (cioè da me).
		\end{warn*}
		
		Lo scopo di oggi è quello di costruire condizioni di stabilità di Bridgeland
		per una varietà $X$ di $\dim X > 1$. Non andremo molto lontano, in quanto
		le costruzioni note riguardano solo $\dim X = 2$ e, congetturalmente, $\dim X =3$;
		vedremo oggi il caso delle superfici.
		
		Nella scorsa lezione, Arvid ci ha parlato della varietà $Stab_{\Lambda,\nu}(C)$
		delle condizioni di stabilità nel caso in cui $C$ sia una curva proiettiva liscia:
		\begin{enumerate}[label=\roman*)]
			\item se $C = \PP^{1}_{\C}$, \textbf{So Okada} ha dimostrato che 
			$Stab_{\Lambda,\nu}(\PP^{1}_{\C}) \simeq \C^{2}$ e i muri sono
			determinati dalle equazioni
			\begin{align*}
				Z(\Oo(k)) = Z(\Oo(k-1)[1])\,, \quad \text{ con } k \in  \Z;
			\end{align*}
			quindi, a meno di autoequivalenze, abbiamo il solo muro $Z(\Oo) = Z(\Oo(-1)[1])$
			che separa la \emph{camera di Gieseker}, data da $Z(\Oo) < Z(\Oo(-1)[1])$,
			con la camera $Z(\Oo) > Z(\Oo(-1)[1])$ in cui
			gli unici oggetti semistabili sono $\Oo^{\oplus m}$ e $\Oo(-1)^{\oplus m'}[1]$.
			\item se $C$ è una curva di genere positivo, 
			allora $Stab_{\Lambda,\nu}(C) = \sigma_{C} \cdot \widetilde{\GL}_{2}^{+}(\R)$,
			dove $\sigma_{C} = (\Coh(C), -\rk + i \deg)$ è la condizione standard
			vista in \textbf{\Cref{ex:curve-bridgeland}}; in altri termini,
			su una curva tutte le condizioni di stabilità di Bridgeland sono
			equivalenti alla nozione di $\mu$-stabilità in senso classico.
		\end{enumerate}
		
		
		Abbiamo visto infatti che in queste situazioni $\Coh(C)$ 
		è un cuore accettabile per una possibile $\sigma \in Stab_{\Lambda,\nu}(C)$, 
		ma non appena $\dim X \ge 2$, per il \textbf{\Cref{lemma:Toda-num}} questo fallisce.
		Quindi, il problema non è che non abbiamo trovato la giusta carica centrale $Z$:
		il fatto è che dobbiamo cambiare categoria!
		
		\begin{oss}
			Questo non è vero nel caso di varietà non-proiettive:
			infatti, se $X$ è una superficie K3 con $\Pic(X) = 0$,
			allora $\Coh(X)$ può essere il cuore per una condizione di stabilità:
			infatti, basta prendere $z:=\alpha + i\beta$ con $\beta > 0$
			e per $\cat{v}(E) = (r,0,a)$ definire $Z(E) = -rz-a$.
			In questa situazione, \textbf{Huybrechts, Macrì} e \textbf{Stellari}
			hanno dimostrato che $Stab(X)$ è connessa e semplicemente connessa.
		\end{oss}
		
			I primi esempi di costruzione di condizioni su una superficie $S$ sono stati 
		trovati da Bridgeland stesso per le superfici K3 proiettive e abeliane, 
		nell'articolo \cite{B-stab-k3}, e poi generalizzato da \textbf{Arcara} e \textbf{Bertram}
		per superfici \emph{K-triviali}\footnote{...anche se Macrì afferma che la costruzione valga per \emph{tutte} le superfici proiettive lisce.}. Data $S$ una superficie K3 proiettiva,
		il suo scopo principale nell'articolo è quello
		di studiare il gruppo delle autoequivalenze $\Aut(\Db{S})$ e, per farlo,
		caratterizza esplicitamente le condizioni di stabilità $Stab^{\dagger}(S)$
		di una speciale componente connessa di $Stab(S)$ e come si comportano
		i vari pezzi della sua frontiera sotto l'azione di $\Aut(\Db{S})$.
		
		\begin{df}
			Una \textbf{superficie K3} è una superficie complessa $S$ compatta, 
			semplicemente connessa, con canonico banale $\Omega_{S}^{2} \simeq \Oo_{S}$.
			Più in generale, si può definire una \textbf{superficie K3 algebrica} su un campo $k$
			come una varietà completa, non singolare, di $\dim S = 2$ tale che
			\begin{align*}
				\Omega_{S|k}^{2} \simeq \Oo_{S} \,, \quad H^{1}(S;\Z) = 0\,.
			\end{align*}
		\end{df}
		
		Prima di entrare nei dettagli tecnici della costruzione, seguiamo l'idea di Bridgeland,
		come spiegata nell'introduzione del suo articolo.
		Data $S$ una K3, vogliamo studiarne le condizioni di stabilità \emph{numeriche},
		quindi consideriamo il suo \textbf{reticolo di Mukai algebrico}
		$$\Lambda := H^{*}_{alg}(S;\Z) = H^{0}(S;\Z) \oplus \operatorname{NS}(S) \oplus H^{4}(S;\Z) \,,$$
		che è isometrico al gruppo di Grothendieck numerico di $S$ 
		tramite il \textbf{carattere di Chern}
		\begin{align*}
			\cat{v} : K_{num}(S) &\longrightarrow H^{*}_{alg}(S;\Z)\,, \\
			\cat{v}([E]) := \ch(E) \smile \sqrt{\td(S)} &= (\rk(E), c_{1}(E), \rk(E) + \ch_{2}(E))\,,
		\end{align*}
		dato che $\sqrt{\td(S)} = (1,0,1)$. 
		Supponiamo ora che $Stab(S) = Stab_{(H^{*}_{alg}, \cat{v})}(S)$ sia non vuoto.
		Siccome siamo in spazi finito-dimensionali con prodotto scalare,
		possiamo riscrivere l'omeomorfismo locale $\Zz$ in \eqref{local-homeo} come
			\begin{equation*}
				\Zz : Stab(S) \longrightarrow (H^{*}_{alg}(S;\Z) \otimes \C)^{\vee} \,,
				\quad \Zz(\sigma) = \langle \cat{v}(\sigma), - \rangle\,, 
			\end{equation*}
		per un unico vettore 
		$\cat{v}(\sigma) \in \C \oplus (\operatorname{NS}(S) \otimes \C) \oplus \C$.
		All'interno di questo spazio vettoriale, andiamo a considerare solo i vettori
		\begin{equation}
			\Pp(S) := \Set{ \cat{w} \in \Lambda \otimes \C \, 
			| \, \Span_{\R}\{ \Re \cat{w}, \Im \cat{w} \} \text{ è un piano definito positivo in } \Lambda \otimes \R }\,.
		\end{equation}
		
		\begin{ex*}
			Per ogni $\omega \in Amp(S)_{\R}$ la classe 
			$e^{i\omega} = \left(1, i\omega, \frac{- \omega^{2}}{2}\right)$
			appartiene in $\Pp(S)$ poiché 
			\begin{align*}
				\langle \Re e^{i\omega}, \Im e^{i\omega} \rangle
				= \langle \left(1, 0, \frac{- \omega^{2}}{2}\right), 
				\left(0, \omega, 0 \right) \rangle = 0\,, \quad \text{ e } \quad
				(\Re e^{i\omega})^{2} = (\Im e^{i\omega})^{2} = \omega^{2} > 0\,.
			\end{align*}
			Inoltre, un conto veloce mostra che, per ogni $B \in \operatorname{NS}(S) \otimes \R$,
			la moltiplicazione $e^{-B} \smile -$ è una 
			trasformazione ortogonale di $H^{*}_{alg}(S;\Z) \otimes \R$, quindi tutti i vettori
			della forma
			$$e^{-B + i\omega} = \left(1, -B + i\omega, \frac{B^{2} - \omega^{2}}{2} - iB \cdot \omega \right)$$
			appartengono a $\Pp(S)$.
		\end{ex*}
		
		Per avere una descrizione più geometrica,
		notiamo che anche $\GL_{2}^{+}(\R)$ agisce liberamente su $\Pp(S)$,
		quindi $\Pp(S)$ può essere visto come un $\GL_{2}^{+}(\R)$-bundle sul quoziente.
		Una possibile sezione di questo fibrato è data da
		\begin{align*}
			\Qq(S) := \Set{ \cat{w} = (w_{0}, w_{2}, w_{4}) \in \Lambda \otimes \C \, 
			| \, \langle \cat{w}, \cat{w} \rangle = 0\,, \,
			\langle \cat{w}, \overline{\cat{w}} \rangle > 0\,, \, w_{0} = 1 }\,,
		\end{align*}
		che ci descrive l'insieme delle condizioni di stabilità 
		\emph{a meno dell'azione di $\GL_{2}^{+}(\R)$}. 
		Questo insieme $\Qq(X)$ si identifica con il dominio tubolare 
		$\Set{-B + i\omega \in \operatorname{NS}(S) \otimes \C \,|\, \omega^{2} > 0}$
		tramite la mappa esponenziale. Detta $\Pp^{+}(S)$ la componente connessa
		di $\Pp(S)$ che contiene le classi $e^{-B + i\omega}$, 
		con $\omega$ ampio come nell'\textbf{Esempio}, questa ci dà una descrizione
		esplicita di un pezzo di $Stab(S)$.
		
		\begin{thm}[\textbf{Bridgeland}]
			Sia $\Delta := \Set{\delta \in H^{*}_{alg}(S;\Z) \, | \, \delta^{2} = -2}$ l'insieme
			delle \emph{classi sferiche} e poniamo
			\begin{equation*}
				\Pp^{+}_{0}(S) := \Pp^{+} \setminus \bigcup_{\delta \in \Delta} \delta^{\perp}\,,
				\quad Stab^{\dagger}(S) := \Zz^{-1}(\Pp^{+}_{0}(S))\,.
			\end{equation*}
			La mappa $\Zz : Stab^{\dagger}(S) \to \Pp^{+}_{0}(S)$ è un rivestimento.
		\end{thm}
		
%		Le classi sferiche hanno un ruolo importante nella geometria delle K3
%		e indotte dalle classi di \textbf{(-2)-curve}.
%		La loro importanza, a livello derivato, è che inducono particolari
%		autoequivalenze, dette \textbf{twist sferici}, la cui composizione
%		può essere invisibile a livello coomologico (ma non a livello derivato).
%		

		In sintesi, questa linea di pensiero giustifica l'aspettativa che,
		al variare di $\omega \in Amp(S)_{\R}$ e $B \in \operatorname{NS}(S)_{\R}$,
		le funzioni della forma
		\begin{equation*}
			Z_{\omega, B} = \langle e^{-B + i \omega}, - \rangle
		\end{equation*}
		siano cariche centrali di una qualche condizione $\sigma_{\omega, B} \in Stab^{\dagger}(S)$.
		In effetti sarà così, ma la cosa complicata da capire, a priori, 
		è come costruire il cuore della condizione $\sigma_{\omega, B}$.
		
		
		\section{Coppie di torsione}
		
			Gli appassionati di algebra non-commutativa e categorie triangolate
			che ci sono a Padova, Verona, Praga e Murcia, conoscono
			un metodo, chiamato \textbf{tilting}, 
			per produrre $t$-strutture nuove da una già conosciuta.
			Questo metodo prevede di sfruttare la relazione tra $t$-strutture 
			e \textbf{teoria di torsione} che esistono per un cuore $\Aa$.
			
			\begin{df}
				Sia $\Aa$ una categoria abeliana. Una \textbf{teoria di torsione} $(\Ff, \Tt)$
				(o \textbf{coppia di torsione}) è una coppia di sottocategorie additive piene
				$\Ff, \Tt \subset \Aa$ tali che $\Hom(\Tt,\Ff) = 0$ e,
				per ogni oggetto $E \in \Aa$, esiste una (unica) sequenza esatta corta
				\begin{equation*}
					0 \longrightarrow T_{E} \longrightarrow E \longrightarrow F_{E} \longrightarrow 0\,,
					\quad \text{ con } T_{E} \in \Tt\,, \, F_{E} \in \Ff\,.
				\end{equation*}
			\end{df}
			
			\begin{ex}
				Data $X$ una varietà proiettiva liscia, 
				la coppia $\Tt = \Set{\text{fasci su $X$ di torsione}}$
				e $\Ff = \Set{\text{fasci su $X$ senza torsione}}$ definisce una teoria di torsione
				in $\Coh(X)$.
			\end{ex}
			
			
			\begin{thm}[\textbf{Tilting}]
				Sia $X$ una varietà proiettiva liscia. 
				Dato $\Aa \subset \Db{X}$ il cuore di una $t$-struttura,
				con coomologico $H_{\Aa}$,
				e $(\Ff, \Tt)$ una teoria di torsione su $\Aa$, allora la
				sottocategoria piena $\Aa^{\#}$, i cui oggetti sono i complessi $E \in \Db{X}$
				tali che
				\begin{equation}
					H^{\Aa}_{0}(E) \in \Tt\,, 
					\quad H_{\Aa}^{-1}(E) \in \Ff
					\quad \text{ e }  H_{\Aa}^{p}(E) = 0 \text{ per ogni } p \notin \{0,-1\}\,,
				\end{equation}
				è il cuore di una $t$-struttura. 
				Denoteremo il \textbf{cuore tiltato} $\Aa^{\#} = \langle \Ff[1], \Tt \rangle$.
%				\begin{proof}[Idea]
%					Infatti, ogni teoria di torsione definisce una $t$-struttura
%					\begin{align*}
%					\Dd^{\le 0} := \Set{ E \in \Db{\Aa} \, | \, \forall_{j > 0} \, H^{j}(E) = 0 
%					\text{ e } H^{0}(E) \in \Tt}\,, \\
%					\Dd^{\ge 0} := \Set{ E \in \Db{\Aa} \, | \, \forall_{j < -1} \, H^{j}(E) = 0 
%					\text{ e } H^{-1}(E) \in \Ff}\,.
%				\end{align*}
%				\end{proof}
			\end{thm}
			
			\begin{oss}
				Essenzialmente, in $\Aa$ ogni oggetto $E$ siede in un triangolo esatto
				$T \to E \to F$, con $(F,T) \in (\Ff, \Tt)$, e quindi rappresenta un elemento 
				di $\Ext_{\Aa}^{1}(T,F)$. Quando tiltiamo, 
				stiamo considerando estensioni nel senso opposto, cioè $E' \in \Db{X}$ che
				siedono in mezzo a un triangolo della forma $F_{E'}[1] \to E'\to T_{E'}$,
				con $F_{E'} \in \Ff$ e $T_{E'} \in \Tt$,
				quindi rappresentano degli elementi in $\Ext^{1}_{\Aa}(T_{E'},F_{E'}[1])
				= \Ext^{2}_{\Aa}(T_{E'},F_{E'})$.
			\end{oss}
			
			Tornando al caso di $X = S$ una superficie (qualsiasi),
			costruiamo una famiglia di teorie di torsione che dipendono da
			$\omega \in Amp_{\R}$ e $B \in \operatorname{NS}(X) \otimes \R$.
			Per prima cosa, twistando il carattere di Chern $\ch^{B} := e^{-B}\ch$,
			possiamo definire una versione \emph{twistata} della classica \textbf{slope-stability}
			considerando la pendenza
			\begin{equation*}
				\mu_{\omega,B}(E) := \frac{\omega \cdot \ch^{B}_{1}(E)}{\omega^{2} \rk(E)} =
				\frac{\omega \cdot c_{1}(E)}{\omega^{2} \rk(E)} - \frac{\omega \cdot B}{\omega^{2}}\,;
			\end{equation*}
			siccome il secondo termine non dipende da $E$, è chiaro che la 
			$\mu_{\omega,B}$-(semi)stabilità coincide con la classica $\mu_{\omega}$-(semi)stabilità.
			L'utilità di $B$ risiede nel poter definire diverse teorie di torsione:
			infatti, definiamo una coppia $(\Ff_{\omega,B}, \Tt_{\omega,B})$ `\emph{troncando}'
			la filtrazione di HN rispetto all'equazione $\mu_{\omega,B} = 0$,
			o più precisamente definiamo
			\begin{align*}
				\Tt_{\omega, B} :=& \Set{ E \in \Coh(X) \,| \, \forall_{q} \: \mu_{\omega}(\operatorname{HN}_{q}(E)) > \omega \cdot B } \\
				\Ff_{\omega, B} :=& \Set{E \in \Coh(X) \,| \, \forall_{q} \: \mu_{\omega}(\operatorname{HN}_{q}(E)) \le \omega \cdot B}\,.
			\end{align*}
			Allora $(\Ff_{\omega,B}, \Tt_{\omega,B})$ è una teoria di torsione
			per quanto sappiamo su $\mu_{\omega}$, e quindi possiamo tiltare 
			$\Coh(X)$ rispetto a questa coppia per ottenere il cuore
			\begin{equation}\label{Coh-wB}
				\Coh^{\omega,B}(S) := \langle \Ff_{\omega,B}[1], \Tt_{\omega,B} \rangle\,.
			\end{equation}
			
			Adesso che abbiamo un cuore e una possibile carica centrale,
			vogliamo convincerci che la coppia
			 $\sigma_{\omega,B} = (\Coh^{\omega,B}(S), Z_{\omega, B})$
			sia una condizione di stabilità di Bridgeland,
			usando il \textbf{\Cref{thm:equivalent-df-stab}}.
			Si dimostra che, per una \textbf{classe razionale} 
			$B \in \operatorname{NS}(S) \otimes \Q$,
			la categoria $\Coh^{\omega,B}(S)$ è noetheriana;
			quindi, alla luce del \textbf{\Cref{thm:discrete-Z}}, per costruire
			una condizione di stabilità nel senso della \textbf{\Cref{df:stab}}
			è sufficiente trovare una funzione
			di stabilità $Z_{\omega,B}$ con $\Im Z_{\omega,B}$ discreto in $\R$.
			Dall'introduzione, sappiamo che deve essere della forma 
			$Z_{\omega,B}([E]) = \langle \cat{v}(\omega,B), \cat{v}(E) \rangle$
			e un candidato appetibile è
			\begin{equation*}
				Z_{\omega,B}([E]) = \langle e^{-B+i\omega}, \cat{v}(E) \rangle
				= - \int_{X} e^{i\omega} \smile \ch^{B}(E)\,,
			\end{equation*}
			che esplicitamente, per $\cat{v} = (r, c, a) \in \Lambda$ appare come 
			\begin{equation}\label{ZwB-explicit}
				Z_{\omega,B}(\cat{v}) = \left( - a + c \cdot B + r\frac{\omega^{2} - B^{2}}{2} \right) + i \omega \cdot (c - rB)\,.
			\end{equation}
			
			\begin{prop}
				Per ogni $\omega \in Amp_{\R}$ e $B \in \operatorname{NS}(S) \otimes \Q$,
				la $Z_{\omega, B}$ è una funzione di stabilità su $\Coh^{\omega,B}(S)$.
%				\begin{proof}
%					Idea...
%				\end{proof}
			\end{prop}
			
			Per avere una condizione di stabilità di Bridgeland è necessario verificare, infine,
			la \textbf{proprietà del supporto}: equivalentemente, 
			è sufficiente trovare una forma quadratica 
			definita positiva sugli oggetti semistabili, ma questa è data\footnote{Non è proprio questo il discriminante che va bene per tutti gli $\omega,B$,
			 ma è quello che funziona per il piano $(\alpha, \beta)$ ed è già molto utile nelle applicazioni.} dal \textbf{discriminante twistato}
				\begin{equation}
					\Delta^{B}_{\omega}(E) := (\omega \cdot \ch_{1}^{B}(E))^{2} - 2 \omega^{2} \ch_{0}^{B}(E) \ch_{2}^{B}(E) \,,
				\end{equation}
			come conseguenza della \textbf{disuguaglianza di Bogomolov},
			che afferma $\Delta^{B}_{\omega}(E) \ge 0$ 
			per ogni $E$ fascio $\mu_{\omega}$-semistabile. 
			Inaspettatamente, la disuguaglianza vale anche per oggetti $Z_{\omega, B}$-semistabili, 
			dopo aver svelato il legame tra la stabilità di Bridgeland e la $\mu_{\omega}$-stabilità:
			
			\begin{lemma}[\textbf{Large Volume Limit}]\label{LVL}
				Un oggetto $E \in \Coh^{\omega,B}(S)$ è $\sigma_{\alpha \cdot \omega, B}$-semistabile
				per ogni $\alpha \gge 0$ se e solo se $E$ è lo shift
				di un fascio $\mu_{\omega,B}$-semistabile. In altre parole: per $\alpha > 0$ tanto
				grande, la Bridgeland stabilità coincide con la slope-stability twistata.
				\begin{proof}[Idea]
					La funzione $Z_{\omega, B}$ definisce la pendenza
					\begin{equation}
						\nu_{\omega,B}(E) = \frac{\ch_{2}^{B}(E)}{\omega \cdot \ch_{1}^{B}(E)}
						- \frac{\omega^{2} \rk(E)}{2\omega \cdot \ch_{1}^{B}(E)}\,.
					\end{equation}
					Ora, per ogni $\alpha > 0$, un oggetto è $\nu_{\alpha \cdot \omega,B}$-(semi)stabile
					se e solo se è $\frac{2}{\alpha}\nu_{\alpha \cdot \omega,B}$-(semi)stabile.
					Ma allora prendere $\alpha \gge 0$ significa considerare il limite
					\begin{align*}
						\lim_{\alpha \to +\infty} \frac{2}{\alpha} \nu_{\alpha \omega,B}(E)
						&= \lim_{\alpha \to +\infty} \frac{2 \ch_{2}^{B}(E)}{ \alpha^{2} \,\omega \cdot \ch_{1}^{B}(E)}
						- \frac{2 \alpha^{2} \omega^{2} \rk(E)}{2 \alpha^{2} \, \omega \cdot \ch_{1}^{B}(E)} \\ &=
						- \frac{\omega^{2} \rk(E)}{\omega \cdot \ch_{1}^{B}(E)} = - \frac{1}{\mu_{\omega, B}(E)}\,,
					\end{align*}
				quindi le disuguaglianze della semistabilità sono essenzialmente le stesse;
				la parte difficile è che bisogna fare attenzione 
				alla nozione di \emph{sottofascio} in $\Coh(S)$ e
				$\Coh^{\alpha \omega, B}(S)$, in quanto sono diverse.
				\end{proof}
			\end{lemma}
			
			\begin{rmk}[per Arvid]
				Non so perché si chiami Large Volume Limit.
			\end{rmk}
			
			
			Siccome $\Delta^{B}_{\omega}$ è una forma quadratica che testimonia 
			la proprietà del supporto, segue che 
			$\sigma_{\omega,B} := (\Coh^{\omega,B}(S), Z_{\omega, B})$
			è una condizione di stabilità di Bridgeland su $\Db{S}$.
			
			Per poter concludere che la mappa
			\begin{equation*}
				\sigma : Amp(S)_{\R} \times \operatorname{NS}(S)_{\R} \longrightarrow Stab^{\dagger}(S)\,, \quad (\omega, B) \mapsto \sigma_{\omega,B}\,,
			\end{equation*}
			sia un embedding continuo, è necessario verificare che anche le
			classi reali $B \in \operatorname{NS}(S)_{\R}$ danno condizioni di stabilità.
			Per il \textbf{\Cref{thm:defo} di deformazione}, siamo sicuri che
			anche dopo una piccola deformazione, $Z_{\omega, B}$ 
			continua a dare una carica centrale per una condizione 
			$\sigma_{\omega,B} = (\Aa, Z_{\omega,B}) \in Stab^{\dagger}(S)$;
			la cosa sorprendente è che questa funzione di stabilità continua a essere
			valida su un cuore $\Aa$ della forma $\Coh^{\omega,B}(S)$: infatti, queste
			categorie ammettono la seguente caratterizzazione.
			\begin{lemma}\label{lemma:heart-characterization}
				Sia $\sigma = (\Aa, Z_{\omega, B})$ una condizione di stabilità 
				che soddisfa la proprietà del supporto. 
				Tutti i fasci grattacielo $\C(s)$ sono oggetti $\sigma$-stabili di fase $1$ 
				(i.e. per ogni $s \in S$, $\mu_{\sigma}(\C(s)) = +\infty$) se e solo se
				$\Aa = \Coh^{\omega,B}(S)$.
			\end{lemma}
			Dato che un fascio grattacielo ha vettore $\cat{v}(\C(s)) = (0,0,1)$,
			la formula \eqref{ZwB-explicit} mostra che $Z_{\omega,B}(\C(s)) = -1$,
			quindi il \textbf{\Cref{lemma:heart-characterization}} vale.
			
			


    		
	\lecture[Applicazione della teoria di Bridgeland per lo studio del problema di Weak Brill-Noether.]{2025-10-04}
		
	\begin{warn*}
			Continuiamo a camminare sui cocci di vetro perché anche questa 
			lezione è stata tenuta da \textbf{Filippo Papallo} (cioè da me).
	\end{warn*}		
		
		Nella lezione di oggi vedremo insieme un'applicazione delle 
		condizioni di stabilità di Bridgeland
		allo studio di un problema classico, quello di \emph{weak Brill-Noether}.
		
		\section{Wall-crossing per stabilità di Bridgeland}
		
			Sia $S$ una superficie complessa proiettiva 
			e fissiamo $H$ un figrato lineare ampio su di essa.
			La volta scorsa abbiamo visto che esiste un embedding continuo
			\begin{equation*}
				\sigma : \operatorname{NS}(S)_{\R} \times Amp(S)_{\R} \longrightarrow Stab^{\dagger}(S)\,, \quad (B, \omega) \mapsto \sigma_{\omega,B}\,,
			\end{equation*}
			che, in generale, non sarà surgettivo. Ma questo non è davvero un problema
			per studiare gli spazi di moduli: infatti, 
			abbiamo accennato alla divisione di $Stab^{\dagger}(S)$
			in \emph{muri} e \emph{camere} e abbiamo accennato al fatto
			che la nozione di stabilità rimane costante in ciascuna camera, 
			analogamente a quanto accade
			alla classica stabilità con la suddivisione del \emph{cono ampio}.
			Quindi, possiamo accontentarci di studiare una fettina \textbf{bidimensionale}
			di $Stab^{\dagger}(S)$, che da Macrì viene chiamata \textbf{piano $(\alpha,\beta)$},
			ma io chiamerò \textbf{piano $(s,t)$}:
				\begin{equation}
					\sigma : \R \times (0,+\infty) \longrightarrow Stab^{\dagger}(S)\,,\quad
					\sigma_{(s,t)} := \sigma_{tH, sH}\,.
				\end{equation}
				
				\begin{warn*}
					Nelle loro note, \textbf{Macr\`i} e \textbf{Schmidt} scrivono che questa $\sigma$
					è un embedding, ma in realtà non è vero che ogni coppia $(s,t)$ definisce
					una condizione di stabilità: questo è espresso chiaramente da Bridgeland
					nel suo articolo \cite{B-stab-k3} sulle K3, e in \cite{CNY-K3} vengono
					descritte due regioni $U_{+}$ e $U_{-}$ 
					dove si è certi di avere condizioni di stabilità.
				\end{warn*}
			
				Preso $E \in \Db{S}$, per ogni $\cat{w} \in K_{num}(S)$ i \textbf{muri numerici} 
				per $E$	sono definiti dalle equazioni
				\begin{equation}\label{formula:num-wall}
					\Im Z_{t H, s H}(\cat{v}(E)) \Re Z_{tH, sH}(\cat{w})
					= \Im  Z_{t H, sH}(\cat{w}) \Re Z_{tH, sH}(\cat{v}(E))\,,
				\end{equation}
				che, scritte esplicitamente, descrivono una quadrica in $s$ e $t$ facile da
				studiare. A \textbf{Maciocia} è dovuta la caratterizzazione completa
				della struttura dei muri per qualsiasi superficie $S$.
				
				\begin{thm}[\textbf{Struttura dei Muri per Superfici}]
					Sia $\cat{v} \in K_{num}(S)$ fissato.
					\begin{enumerate}[label=\roman*)]
						\item Tutti i muri numerici per $\cat{v}$ sono semicerchi
						con centro sull'asse $s$, oppure sono semirette verticali;
						\item muri distinti non si intersecano;
						\item se $\ch_{0}(\cat{v}) \ne 0$, allora il muro
						verticale è unico ed è definito dall'equazione
						\begin{equation}
							s = \frac{H \cdot \ch_{1}(\cat{v})}{H^{2}\ch_{0}(\cat{v})}\,, \quad t > 0\,;
						\end{equation}
						In questo caso, tutti i muri semicircolari sono inscatolati, e giacciono da un solo lato del muro.
						\item se $\ch_{0}(\cat{v}) = 0$, allora ci sono solo muri circolari
						inscatolati.
						\item se un muro numerico ha un punto in cui è un muro effettivo
						(i.e. esiste un oggetto $E \in \Aa_{\sigma}$ che diventa strettamente semistabile), allora il muro numerico è un muro effettivo.
					\end{enumerate}
				\end{thm}
				
			Vale la pena sottolineare che non tutti i muri numerici ``dividono'' condizioni
			di stabilità diverse: possono esistere muri \emph{superflui},
			spesso chiamati \emph{fake walls}, e muri che, una volta attraversati,
			trasformano gli spazi di moduli secondo fenomeni detti di \emph{wall-crossing}.
			Stupefacentemente, in alcuni casi è possibile classificare esplicitamente la natura
			di tutti i muri e le trasformazioni birazionali ad essi legate.
			
			\begin{df}
				Fissato $\cat{v} \in K_{num}(S)$, sia $\Ww \subset Stab^{\dagger}(S)$
				un muro per $\cat{v}$. Indichiamo con $\sigma_{0} \in \Ww$ una condizione sul muro,
				mentre con $\sigma_{+}$ e $\sigma_{-}$ due condizioni che stanno in camere distinte,
				separate da $\Ww$. Diremo che $\Ww$ è:
				\begin{itemize}
					\item un \textbf{fake wall} se in 
					$M_{\sigma_{+}}(\cat{v})$ e $M_{\sigma_{-}}(\cat{v})$ non esistono curve
					di oggetti S-equivalenti tra loro rispetto alla stabilità indotta 
					da $\sigma_{0}$;
					\item un \textbf{muro totalmente semistabile} 
					se $M_{\sigma_{0}}(\cat{v}) = \emptyset$;
					\item un \textbf{flopping wall} se esiste una \emph{flopping contration}
					$M_{\sigma_{+}}(\cat{v}) \dashrightarrow M_{\sigma_{-}}(\cat{v})$;
					\item un \textbf{muro divisoriale} se esiste una contrazione divisoriale
					$\pi_{\pm} : M_{\sigma_{\pm}}(\cat{v}) \to \overline{M}_{\pm}$
					verso una varietà proiettiva irriducibile e normale.
				\end{itemize}
			\end{df}
			
			Non mi soffermo sul significato delle definizioni per due motivi: il primo è che non 
			conosco queste trasformazioni, note in geometria birazionale; il secondo è che
			saremo interessati solamente ai \textbf{muri totalmente semistabili}.
			
			\begin{ex}\label{K3-walls}
				Sia $S$ una superficie K3 proiettiva. 
				Esistono molti risultati di classificazione degli spazi $M_{S,H}(\cat{v})$
				in base al vettore di Mukai $\cat{v} = (r,c,a)$ fissato, e in particolare
				\textbf{Bayer} e \textbf{Macrì} (tra gli altri) hanno portato
				a conclusione lo studio degli spazi dei moduli di oggetti Bridgeland semistabili:
				
				\begin{prop}
					Sia $\cat{v} = m \cat{v}_{0}$ con $m > 0$ e $\cat{v}_{0}$ una classe primitiva.
					Supponiamo che $\sigma$ sia \textbf{$\cat{v}$-generica}, ovvero non giaccia
					su alcun muro numerico per $\cat{v}$. Se $\cat{v}_{0}^{2} \ge -2$,
					allora $M_{\sigma}(\cat{v}) \ne \emptyset$,  ha dimensione $\cat{v}^{2}+2$ 
					e $M_{\sigma}^{s}(\cat{v}) \ne \emptyset$ 
					(con l'eccezione $\cat{v}_{0}^{2} \le 0$ e $m > 1$). Inoltre,
					se $\cat{v}_{0}^{2} > 0$, allora $M_{\sigma}(\cat{v})$ è
					una varietà normale, proiettiva e irriducibile.
				\end{prop}
				
				Ora che sappiamo quando i moduli esistono, concentriamoci sui muri.
				Il \textbf{reticolo di Mukai} di $S$ è definito come
				\begin{equation*}
					H_{alg}^{*}(S; \Z) 
					:= H^{0}(S; \Z) \oplus \operatorname{NS}(S) \oplus H^{4}(S; \Z) \,,
				\end{equation*}
				e il \textbf{vettore di Mukai}
				\begin{equation*}
					\cat{v} : K_{num}(S) \longrightarrow H_{alg}^{*}(S; \Z)\,, \quad
					\cat{v}(E) := \ch(E) \smile \sqrt{\td(S)} 
					= (\rk(E), c_{1}(E), \rk(E) + \ch_{2}(E))
				\end{equation*}
				induce un'isometria, dove il prodotto scalare è dato dal \textbf{pairing di Mukai}
				\begin{equation*}
					\langle (r,c,a), (r_{1},c_{1},a_{1}) \rangle := c \cdot c_{1}- ra_{1} - r_{1}a\,,
				\end{equation*}
				che dunque per $E,F \in \Coh(S)$ coincide con il prodotto di Eulero
				\begin{equation*}
					\langle \cat{v}(E), \cat{v}(F) \rangle 
					= - \chi(E,F) = \sum_{p=0}^{2} (-1)^{p+1} \operatorname{ext}^{p}(E,F)\,.
				\end{equation*}								
				Nell'$(s,t)$-plane, la formula esplicita \eqref{ZwB-explicit} della carica centrale
				diventa
				\begin{equation*}
				Z_{(s,t)}(\cat{v})  = \langle e^{sH + itH}, (r,c,a,) \rangle 
				= \left( s(c \cdot H) - a - r H^{2} \frac{s^{2} - t^{2}}{2} \right) + i t H \cdot (c - rsH)\,,
				\end{equation*}
				pertanto, se $\cat{v}_{1} \in H^{*}_{alg}(S; \Z)$ è un altro vettore,
				l'equazione \eqref{formula:num-wall} del muro per $\cat{v}$ definita da $\cat{v}_{1}$
				si può riscrivere come:
				\begin{equation}
					 \frac{s^{2}+t^{2}}{2} [r_{1}(c \cdot H) - r (c_{1} \cdot H)] 
					 - s(r_{1}a-ra_{1})
					 = \frac{1}{H^{2}} [a_{1}(c \cdot H) - a (c_{1} \cdot H)]\,.
				\end{equation}
				Possiamo semplificare l'espressione come 
				$\frac{\epsilon}{2}(s^{2}+t^{2}) - \gamma s = \frac{\delta}{H^{2}}$
				ponendo
				\begin{equation}\label{not:edg}
					\epsilon := r_{1}(c \cdot H) - r (c_{1} \cdot H) \,, \quad
					\delta := a_{1}(c \cdot H) - a (c_{1} \cdot H) \,, \quad
					\gamma := r_{1}a-ra_{1} \,;
				\end{equation}
				si noti che $\epsilon = 0$ significa che $\cat{v}$ e $\cat{v}_{1}$ hanno la
				stessa pendenza $\mu$ e, in tal caso, il muro identificato da $\cat{v}_{1}$
				è verticale. Altrimenti, abbiamo una semicirconferenza centrata sull'asse $s$.
				La seguente classificazione è dovuta a \textbf{Bayer} e \textbf{Macrì} \cite[\textbf{Theorem~{5.7}}]{BM-MMP}:
				\begin{thm}
					Supponiamo $\cat{v}^{2}>0$. Un muro numerico $\Ww \subset Stab^{\dagger}(S)$ dato dall'equazione \eqref{formula:num-wall} è totalmente semistabile se e solo se $\cat{w}$ soddisfa una delle due condizioni: 
					\begin{itemize}
						\item $\cat{w}^{2} = 0$ e $\langle \cat{v}, \cat{w} \rangle = 1$;
						\item $\cat{w}^{2} = -2$, i.e. è una classe sferica effettiva, 
						e $\langle \cat{v}, \cat{w} \rangle < 0$.
					\end{itemize}
				\end{thm}
			\end{ex}
			
			\begin{ex}\label{Enriques-walls}
				Sia $S$ una superficie di Enriques proiettiva. Ci concentreremo sul caso di una
				generica Enriques \textbf{non-nodale}, i.e. $S$ non continene $(-2)$-curve;
				in questo caso, il numero di Picard è $10$ e quindi il diamante di Hodge
				appare così:
				\begin{equation*}
					\begin{tikzcd}[row sep = tiny, column sep = tiny]
						& & 1 &  & \\
						& 0 & & 0  & \\
						0 & & 10 &  & 0 \\
						& 0 & & 0  & \\
						& & 1 &  & \,.
					\end{tikzcd}
				\end{equation*}
				Per una tale $S$, consideriamo il reticolo
				\begin{equation*}
					\Lambda = H^{*}_{alg}(S; \Z) := \Set{ \left(r,c,\frac{a}{2} \right) \, | \, 
					r, a \in \Z \text{ tali che } r \equiv_{2} a, \, c \in \operatorname{NS}(S)/\langle K_{X} \rangle }
					\subset H^{*}(S; \Q)\,.
				\end{equation*}
				Ricordando che $\td(S) = (1,0,\frac{1}{2})$, notiamo che è ben definito
				il vettore di Mukai
				\begin{equation*}
					\cat{v} : K_{num}(S) \longrightarrow H_{alg}^{*}(S; \Z)\,, \quad
					\cat{v}(E) :=  
					= (\rk(E), c_{1}(E), \frac{\rk(E)}{2} + \ch_{2}(E))\,,
				\end{equation*}
				compatibile con il pairing di Mukai
				\begin{equation*}
					\langle \left(r,c,\frac{a}{2}\right),\left(r_{1},c_{1},
					\frac{a_{1}}{2}\right) \rangle := c \cdot c_{1}- \frac{ra_{1}}{2} - \frac{r_{1}a}{2}\,,
				\end{equation*}
				quindi è tutto formalmente simile al caso delle K3 e, in effetti, è compatibile:
				detto $\pi : \widetilde{S} \to S$ il rivestimento universale $2 : 1$ dell'Enriques,
				esso induce un embedding di reticoli:
				\begin{equation*}
					\pi^{*} : H_{alg}^{*}(S; \Z) \hookrightarrow H_{alg}^{*}(\widetilde{S}; \Z)\,,
					\quad \text{ tale che } \langle \pi^{*} \cat{v}, \pi^{*} \cat{w} \rangle 
					= 2 \langle \cat{v}, \cat{w} \rangle\,.
				\end{equation*}
				Fissato $\cat{v} \in  H_{alg}^{*}(S; \Z)$, usando
				la notazione \eqref{not:edg}, l'equazione del muro individuato
				da $\cat{v}_{1} = (r_{1}, c_{1}, \frac{a_{1}}{2})$ è
					\begin{equation*}
							\epsilon(s^{2}+t^{2}) - \gamma s = \frac{\delta}{H^{2}}\,.
					\end{equation*}									
				Lo studio degli spazi di moduli $M_{\sigma}(\cat{v})$ per le Enriques
				è stato effettuato nel dettaglio da \textbf{Yoshioka}; non riporto qui
				i criteri di esistenza, ma basti sapere che per $\cat{v}^{2} \ge 4$ e $\sigma$
				condizione $\cat{v}$-generica, allora $M_{\sigma}(\cat{v})$ è non vuoto,
				irriducibile e normale. La classificazione completa dei muri in $Stab^{\dagger}(S)$
				è dovuta a \textbf{Nuer} e \textbf{Yoshioka} in \cite{NY-MMP}.
				\begin{thm}\label{thm:K3-tss-walls}
					Sia $S$ una generica Enriques non-nodale e $\cat{v} \in H^{*}_{alg}(S; \Z)$
					tale che $\cat{v}^{2} \ge 4$. Allora un muro numerico 
					$\Ww \subset Stab^{\dagger}(S)$ dato dall'equazione \eqref{formula:num-wall} 
					è totalmente semistabile se e solo se 
					$\cat{w}=(r_{1},c_{1},\frac{a_{1}}{2})$ soddisfa una delle seguenti
					condizioni: 
					\begin{itemize}
						\item $\cat{w}^{2} = -1$ e $\langle \cat{v}, \cat{w} \rangle < 0$;
						\item $\cat{w}^{2} = 0$, $\ell(\cat{w}):= \gcd(r,c,a) = 2$ 
						e $\langle \cat{v}, \cat{w} \rangle = 1$.
					\end{itemize}
				\end{thm}
			\end{ex}
			
			
			
	
		
		\section{La teoria di Brill-Noether}
		
			Come ogni problema in geometria, il primo passo per formularlo è per le curve.
			Sia $C$ una curva complessa di genere $g$. Con Arvid abbiamo parlato
			dell'esistenza degli spazi di moduli $M_{C}(r,d)$ di fibrati vettoriali
			di rango $r$ e grado $d$, e in particolare nel \textbf{\Cref{thm:moduli-curve}}
			abbiamo annunciato alcune sue proprietà. Il passo successivo è
			quello di capirne meglio la geometria e alcuni sottoluoghi speciali,
			andando a studire, ad esempio, il comportamento coomologico del generico
			fibrato in $M_{C}(r,d)$. 
			
			Nel caso $r=1$, sappiamo che i fibrati lineari
			di grado $d$ sono parametrizzati da $\Pic^{d}(C)$, che è una varietà abeliana,
			e il \textbf{Teorema di Riemann-Roch} mostra che $\chi(\Ll) = d - g +1$,
			quindi la coomologia del generico $\Ll \in \Pic^{d}(C)$ è essenzialmente determinato
			dal grado:
			\begin{prop}\label{prop:WBN-curve}
				Sia $C$ curva di genere $g$ e $\Ll \in \Pic^{d}(C)$ generico, allora
				\begin{equation*}
					h^{0}(C,\Ll) = \max\{0,d-g+1\}\,, \quad h^{1}(C, \Ll) = \max\{0, g-d-1\}\,.
				\end{equation*}
				In particolare, un generico $\Ll$ con $\chi(\Ll) \ge 0$ ha $h^{1}(\Ll) = 0$.
			\end{prop}
			
			\begin{df}\label{df:WBN}
				Un fascio coerente $\Ff$ su una varietà proiettiva (complessa) $X$ soddisfa
				la condizione di \textbf{weak Brill-Noether} (in breve \textbf{WBN}) 
				se $\Ff$ ha \emph{al più} un gruppo di
				coomologia non nullo. Chiamiamo $\Ff$ \textbf{non-speciale} se
					$H^{i}(X, \Ff) = 0$ per ogni  $i > 0$,
				e \textbf{speciale} altrimenti.
			\end{df}
			
			La classica Teoria di Brill-Noether si concentra sullo studio dei luoghi
			in $\Pic^{d}(C)$ di fibrati \emph{speciali}, 
			quindi con un comportamento omologico inaspettato: solitamente, a $h > 0$ fissato,
			si studiano i cosiddetti
			\begin{align*}
				W^{h}_{d} := \Set{ \Ll \in \Pic^{d}(C) \,|\, h^{0} \ge h + 1}\,.
			\end{align*}
			Per capirne la natura, è sufficiente il \textbf{numero di Brill-Noether}
			\begin{equation}
				\rho(g,h,d) := g - (r+1)(g-d+r)\,.
			\end{equation}
			
			\begin{thm}
				Se $C$ è una \emph{generica} curva di genere $g$, allora
				\begin{enumerate}[label=\roman*)]
					\item $W^{h}_{d} \ne 0$ se e solo se $\rho(g,h,d) \ge 0$;
					\item Se $\rho(g,h,d) \ge 0$, allora $W^{h}_{d}$ è normale, Cohen-Macaulay
					di dimensione $\rho(g,h,d)$, liscio al di fuori di $W^{h+1}_{d}$;
					inoltre, se $\rho(g,h,d) > 0$, lo spazio $W^{h}_{d}$ è irriducibile.
				\end{enumerate}
			\end{thm}
			
			Ci sono anche generalizzazioni 
			per fibrati vettoriali \textbf{stabili} di rango $r \ge 2$ sulle curve, 
			e per studiarli confronta la pendenza $\mu$ con le quantità descritte sopra.
			Non entreremo nei dettagli, ma ci concentreremo sul problema per le superfici.
			Se $S$ è una superficie complessa, emergono le seguenti difficoltà:
			\begin{itemize}
				\item la coomologia di un fibrato $E$ su $S$ è concentrato in ben tre gradi, i.e.
				$H^{p}(S, E) = 0$ per $p<0$ e $p>2$, mentre su una curva avevamo solo due possibili
				coomologie non nulle; in particolare, \textbf{\Cref{prop:WBN-curve}}
				mostra che tutte le curve soddisfano WBN, ma su $S$ possono esserci più coomologie non nulle, nonostante qualche vanishing (e.g. se $S$ è una K3, $h^{1}(\Oo_{X}) = 0$ ma $h^{0}(\Oo_{X}) = h^{2}(\Oo_{X}) = 1$);
				\item a differenza del caso delle curve, 
				lo stack dei fasci coerenti con invarianti fissati su $S$ è quasi sempre riducibile;
				per ottenere proprietà migliori è necessario considerare i \textbf{fasci stabili}
				rispetto a un fibrato ampio $H$. Nonostante tutto, esistono casi in cui
				$M_{S,H}(\cat{v})$ sia ridotto o disconnesso, e quindi non ha senso parlare
				di un `\emph{elemento generico}'.
			\end{itemize}
			
			\begin{problem*}[\textbf{Weak Brill-Noether}]
				Data una componente irriducibile di $M_{S,H}(\cat{v})$, calcolare la 
				coomologia dell'elemento generico $E$ in tale componente.
			\end{problem*}
			
			Esistono già risultati dettagliati nel caso di superfici razionali
			e nel caso di alcune superfici $K$-triviali, come le superfici K3, 
			abeliane e biellittiche; un resoconto dettagliato sull'argomento è \cite{CHN-WBN-survey}.
			Oggi vedremo come approcciare il problema per le superfici K3,
			usando la strategia di \textbf{Coskun, Nuer} e \textbf{Yoshioka};
			la stessa strategia può essere usata per altre superfici $K$-triviali,
			come ad esempio le Enriques.
			
	\section{Il problema di Weak Brill-Noether via Bridgeland}

		Le superfici $K$-triviali $S$ hanno una ricca struttura derivata e
		il \textbf{Teorema di ricostruzione di Bondal-Orlov} fallisce per queste varietà.
		In particolare, esistono molte autoequivalenze di Fourier-Mukai
		\begin{equation*}
			\Phi^{\Ee} : \Db{S} \longrightarrow \Db{S}\,, \quad 
			\Phi^{\Ee}(E) := (\operatorname{pr}_{2})_{*}(\operatorname{pr}_{1}^{*}(E) \otimes \Ee)\,,
		\end{equation*}
		con $\Ee \in \Db{S \times S}$, che permettono (sotto opportune ipotesi) 
		di passare tra spazi di moduli diversi. In particolare, utilizziamo una particolare
		trasformata di Fourier-Mukai per cambiare punto di vista al problema di WBN.
		
		Sia $S$ una superficie K3 proiettiva con $\Pic(S) \simeq \Z H$, con $H$ ampio.
		Se $E$ è un generico fascio stabile con vettore $\cat{v}(E) = (r,dH,a)$, dove $d > 0$,
		allora sappiamo già che 
		\begin{equation*}
			H^{2}(S, E) \simeq \Ext^{2}(\Oo_{S}, E) \simeq \Hom(E, \Oo_{S})^{\vee} = 0\,,
		\end{equation*}
		dove l'annullamento è dovuto al fatto che $\mu(E) > 0 = \mu(\Oo_{S})$.
		Quindi il problema di WBN si riduce al calcolo dell'$h^{1}(E)$.
		
		\begin{lemma}\label{lemma:FM-WBN}
			Sia $\Delta \subset S \times S$ la diagonale e indichiamo con $\Ii_{\Delta}$
			l'ideale della diagonale. Sia $E \in \Coh(S)$ senza torsione in dimensione zero.
			Se $\Phi^{\Ii_{\Delta}}(E)^{\vee}$ è un fascio (in grado zero),
			allora $E$ soddisfa WBN non-speciale (e inoltre è genericamente globalmente generato).
			\begin{proof}
				Dalla successione esatta corta di fasci
				\begin{equation*}
					0 \longrightarrow \Ii_{\Delta} \longrightarrow \Oo_{S \times S}
					\longrightarrow \Oo_{\Delta} \longrightarrow 0\,,
				\end{equation*}
				si ottiene un triangolo esatto di trasformate di Fourier-Mukai applicate a $E$,
				la cui successione esatta lunga in coomologia è
				\begin{equation*}
        \begin{tikzcd}[row sep=small]
            0 \ar[r] 
            & \Hh^{0}(\Phi^{\Ii_{\Delta}}(E)) \ar[r]
            & H^{0}(S,E) \otimes \Oo_{S} \ar[r, "ev_{E}"]
            & E  \\
            {} \ar[r] 
            & \Hh^{1}(\Phi^{\Ii_{\Delta}}(E)) \ar[r, "\alpha_{1}"]
            & H^{1}(S,E) \otimes \Oo_{S}\ar[r]
            & 0   \\
            {} \ar[r] 
            & \Hh^{2}(\Phi^{\Ii_{\Delta}}(E)) \ar[r, "\alpha_{2}"]
            & H^{2}(S,E) \otimes \Oo_{S}) \ar[r]
            & 0\,. 
        \end{tikzcd}
    \end{equation*}
    		Riscrivendo le coomologie come
    		\begin{equation*}
    			\Hh^{p}(\Phi^{\Ii_{\Delta}}(E)) = 
    			\Hh^{p}(R\Hh om (\Phi^{\Ii_{\Delta}}(E)^{\vee}, \Oo_{S}))
    			= \Ee xt^{p}(\Phi^{\Ii_{\Delta}}(E)^{\vee}, \Oo_{S})\,,
    		\end{equation*}
    		l'ipotesi che $\Phi^{\Ii_{\Delta}}(E)^{\vee}$ sia un fascio
    		implica che il fascio $\Ee xt^{p}$ sia di torsione per ogni $p > 0$, 
    		da cui $\alpha_{1} = \alpha_{2} = 0$ poiché i codomini sono fasci
    		liberi da torsione; la tesi segue.
			\end{proof}
		\end{lemma}	
		
		Usando i criteri di Bondal e Orlov, si può anche dimostrare il seguente
		\begin{lemma}\label{lemma:autoequiv}
			Il funtore $\Phi : E \mapsto \Phi^{\Ii_{\Delta}}(E)^{\vee}$ induce
			un'autoequivalenza di $\Db{S}$.
		\end{lemma}
		
		Quindi, per studiare il problema di weak Brill-Noether per $E$, ci siamo 
		ricondotti a studiare quando la trasformata $\Phi^{\Ii_{\Delta}}(E)^{\vee}$ è un fascio
		(localmente libero). Per capirlo, useremo proprio le tecniche della teoria di Bridgeland!
		Infatti, all'interno di $Stab^{\dagger}(S)$ esiste una camera molto speciale:
		dato un qualsiasi punto $x \in S$, si verifica che 
		il fascio $\Ii_{\Delta}\vert_{x \times S}^{\vee}$
		è un oggetto $\sigma_{(s,t)}$-stabile, per ogni 
		$0 < s \lle 1$ e $t > 0$ in una determinata regione.
		Sia $\Ww$ il muro per $E$ determinato da 
		$\cat{w} = \cat{v}(\Ii_{\Delta}\vert_{x \times S}^{\vee}[1])=(-1,0,0)$,
		cioè la semicirconferenza
		\begin{equation*}
			\Ww_{0} : t^{2} + s \left( s - \frac{2a}{dH^{2}} \right) = 0\,.
		\end{equation*}
		Sia $\Cc_{0}$ la camera adiacente a $\Ww_{0}$ `\emph{da sopra}', 
		in cui vale la disuguaglianza
		$\mu_{Z_{(s,t)}}(\cat{w}) > \mu_{Z_{(s,t)}}(\cat{v}(E))$.
		
		\begin{prop}
			Per ogni $(s,t) \in \Cc_{0}$, la trasformata $\Phi$ del \textbf{\Cref{lemma:autoequiv}}
			induce un isomorfismo tra gli spazi di moduli
			\begin{equation*}
				\Phi : M_{\sigma_{(s,t)}}(r,dH,a) \overset{\sim}{\longrightarrow} M_{S,H}(a,dH,r)\,.
			\end{equation*}
		\end{prop}
		
		D'altra parte, a causa del \textbf{Large Volume Limit~\ref{LVL}} sappiamo che
		per $t \gge 1$ cadiamo nella camera di Gieseker $\Gg$, i.e. esiste un isomorfismo
		di spazi di moduli $M_{\sigma_{(s,t)}}(r,dH,a) \simeq  M_{S,H}(r,dH,a)$.
		Mettendo tutto insieme, ricapitoliamo quindi: se esiste un percorso
		in $Stab^{\dagger}(S)$ che parte da $\Gg$, arriva in $\Cc_{0}$
		e \emph{non attraversa nessun muro totalmente semistabile}, allora
		il generico elemento $E \in M_{S,H}(r,dH,a)$,
		che è $\sigma_{(s,t)}$-semistabile per $(s,t) \in \Cc_{0}$,
		viene trasformato da $\Phi$ in un \emph{fascio} $\mu$-semistabile 
		$\Phi^{\Ii_{\Delta}}(E)^{\vee} \in M_{S,H}(a,dH,r)$ e quindi
		deduciamo che $E$ non ha coomologie in gradi positivi
		per via del \textbf{\Cref{lemma:FM-WBN}}. 
		
		Ora, come fare a capire se esistono muri totalmente semistabili?
		Grazie alla classificazione presentata nell'\textbf{Esempio~\ref{K3-walls}},
		il problema si riduce a studiare delle disequazioni molto elementari:
		infatti, osservando che $\Ww_{0}$ passa dall'origine dell'$(s,t)$-plane,
		per capire se esistono dei muri più in alto di questo, è sufficiente studiare
		quando l'asse $s=0$ interseca muri di equazione \eqref{formula:num-wall}:
		quindi, una condizione necessaria che si aggiunge a quelle del
		\textbf{\Cref{thm:K3-tss-walls}} è che $\epsilon/\delta > 0$.
		In realtà, queste condizioni sono anche sufficienti, e riassumiamo tutto
		quanto detto nel seguente risultato:
		\begin{thm}
			Sia $S$ una generica K3 proiettiva, con $\Pic(S) \simeq \Z H$,
			e fissiamo un vettore di Mukai $\cat{v} = (r,dH,a)$, 
			con 
				\begin{equation*}
					r \ge 0\,, \quad  d > 0\,, \quad  r+a \ge 0 \quad  \text{e} \quad \cat{v}^{2} > 0\,.
				\end{equation*}							
			Se non esistono vettori $\cat{w} = (r_{1}, d_{1}H, a_{1}) \in H_{alg}^{*}(S; \Z)$
			tali che
			\begin{equation}\label{num-conditions}
				\cat{w}^{2} \in \Set{0,-1}\,,
				\quad \langle \cat{v}, \cat{w} \rangle < \cat{w}^{2} + 2\,,
				\quad d \ge d_{1} > 0 \quad \text{e} \quad
				\frac{ad_{1}-a_{1}d}{rd_{1}-r_{1}d} >0\,,
			\end{equation}
			allora $E$ soddisfa la proprietà di weak Brill-Noether (non-speciale).
		\end{thm}
		
		\begin{rmk}
			La condizione $d \ge d_{1} > 0$ in \eqref{num-conditions}
			deriva dalla definizione dei cuori tiltati: infatti, supponiamo
			che $\cat{w}$ sia il vettore di Mukai di un fascio $F$ destabilizzante
			tale che $F[1] \in \Ff_{(s,t)}$, i.e.
			 $\Im Z_{(s,t)}(\cat{w}) = (d_{1}-r_{1}s)tH^{2}\ge 0$.
		\end{rmk}		
		
		\begin{thm}[\textbf{Coskun-Nuer-Yoshioka}]
			Sia $S$ una generica K3 proiettiva, con $\Pic(S) \simeq \Z H$ e $H^{2} = 2h$. 
			Consideriamo un vettore di Mukai $\cat{v} = (r,dH,a)$, 
			con 
				\begin{equation*}
					r \ge 2\,, \quad  d > 0 \quad  \text{e} \quad \cat{v}^{2} \ge -2\,.
				\end{equation*}							
			Allora:
			\begin{enumerate}
				\item fissato $r \ge 2$, esiste un numero finito di tuple $(r,d,h,a) \in \Z^{4}$
				per cui il vettore $\cat{v} = (r,dH,a)$ \textbf{non} soddisfi la 
				proprietà di weak Brill-Noether;
				\item se $h \ge r$, allora $\cat{v}$ soddisfa weak Brill-Noether;
				\item se $a \le 1$, allora $\cat{v}$ soddisfa weak Brill-Noether.
			\end{enumerate}
		\end{thm}
		
		Questo criterio offre quindi un modo sistematico per capire quando un (generico) fascio
		non ha coomologie superiori solamente studiandone il vettore di Mukai!
		I teoremi enunciati sopra possono offrire anche
		un modo per andare a individuare possibili controesempi al problema di WBN;
		tuttavia, bisogna fare attenzione, poiché non è sempre detto che l'esistenza di un $\cat{w}$
		che soddisfi le condizioni \eqref{num-conditions} implichi che il generico
		elemento $E \in M_{S,H}(\cat{v})$ \emph{non} soddisfi WBN. In queste situazioni,
		è necessario raffinare i criteri e studiare più nel dettaglio cosa accade 
		a $E$ sul muro, ad esempio capendo la sua filtrazione di HN $E \in M_{\sigma_{0}}(\cat{v})$.
		
		
		
	
    	
    		
	\lecture[.]{2025-11-20}
	
		\section{Dettagli e applicazioni}
		
		\begin{warn*}
			L'incontro di oggi è presentato da \textbf{Alessandro Frassineti}.
		\end{warn*}		
		
		Lo scopo di oggi è di presentare alcuni ragionamenti sulla teoria di Bridgeland
		e l'elaborazione di alcuni dettagli che Alessandro ha svolto per poter capire
		bene le condizioni di stabilità; faremo molti conti, utili a capire questi argomenti.
		
		Come le scorse volte, denotiamo con $\omega, B \in NS(X)_{\R}$ due classi reali,
		con $\omega$ ampio. Fissato un fibrato lineare ampio $H$, considereremo $\omega = \alpha H$,
		con $\alpha$ reale positivo. Grazie a queste due classi, possiamo definire le seguenti
		quantità:
		\begin{align}
			\ch_{0}^{B} = \ch_{0} \dots
		\end{align}
		
		\section{Dettagli scritti bene}
		
		La scorsa volta abbiamo utilizzato l'$(\alpha,\beta)$-plane per studiare varie nozioni
		di stabilità per fasci e complessi di fasci. Oggi vogliamo capire come
		i muri in $Stab(X)$ cambiano la proprietà del supporto rispetto al
		discriminante $\overline{\Delta}_{\omega}^{B}$.
		
		\begin{lemma}\label{lemma:raggi-estremali}
			Sia $Q$ una forma quadratica su uno spazio vettoriale reale $V$
			e $Z : V \to \C$ mappa $\R$-lineare tale che $\ker Z$ sia semidefinito negativo.
			Sia $\rho$ un raggio in $\C$ che parte dall'origine e poniamo
			\begin{equation}
				\Cc^{+}_{\rho} = Z^{-1}(\rho) \cap \Set{Q \ge 0}\,.
			\end{equation}
			\begin{enumerate}
				\item Se $\omega_{1}, \omega_{2} \in \Cc^{+}_{\rho}$, allora $Q(\omega_{1},\omega_{2}) \ge 0$.
				\begin{proof}
					Dato che  esiste $\lambda > 0$ tale che $Z(\omega_{1} - \lambda \omega_{2}) = 0$,
					allora
					\begin{equation}
						Q(\omega_{1} - \lambda \omega_{2}) \le 0 = Q(\omega_{1}) + \lambda^{2} Q(\omega_{2}) - 2 \lambda Q(\omega_{1}, \omega_{2})\,,
					\end{equation}
					da cui la tesi.
				\end{proof}
				
				\item $\Cc^{+}_{\rho}$ è un cono convesso.
				\item Se $\omega_{1}, \omega_{2} \in \Cc^{+}_{\rho}$, allora
				$0 \le Q(\omega_{1}) + Q(\omega_{2}) \le  Q(\omega_{1} + \omega_{2})$;
				inoltre, se $Q(\omega_{1}) = Q(\omega_{2})$, allora è tutto zero.
				\begin{proof}
					...
				\end{proof}
				
				\item Se $\ker Z$ è definito negativo, allora ogni $\omega \in \Cc^{+}_{\rho}$
				tale che $Q(\omega) = 0$ è in un raggio estremale del cono.
				\begin{proof}
					...
				\end{proof}
			\end{enumerate}
		\end{lemma}
		
		\begin{cor}
			Sia $Q$ una forma quadratica sul reticolo $\Lambda$, 
			con $v : K_{0}(X) \twoheadrightarrow \Lambda$ $E \in \Db{X}$. 
			Sia $E \in \Db{X}$ e $\sigma \in Stab(X)$.
			Se $E$ è $\sigma$-stabile e $Q(E) = 0$, allora $E$ è $\sigma'$-stabile
			per ogni altra condizione $\sigma'$ nella stessa componente connessa contenente $\sigma$.
			\begin{proof}
				Sappiamo già che $E$ è $\sigma'$-stabile per ogni condizione $\sigma'$ nella
				stessa camera di $\sigma$, quindi cerchiamo di capire
				se consideriamo $\sigma'$ su un muro $\Ww$ per $v(E)$ che destabilizza $E$.
				Sappiamo che $\ker Z_{\sigma'}$ è definito negativo perché vale la proprietà del supporto,
				quindi per il \textbf{\Cref{lemma:raggi-estremali}}, sappiamo
				che $v(E)$ è un raggio estremale di $\Cc^{+}_{\rho} \subset \Lambda_{\R}$.
				L'oggetto $E$ è strettamente semistabile 
				e ammette la $\sigma'$-filtrazione di JH con fattori
				$E_{1}, \dots, E_{n}$; siccome tutti i fattori hanno la stessa fase,
				in particolare le loro cariche centrali giacciono sullo stesso raggio in $\C$,
				e quindi $v(E_{j}) \in \Cc^{+}_{\rho}$, per ogni $j$.
				In particolare, per additività $v(E) = v(E_{1}) + \dots + v(E_{n})$,
				ma essendo $v(E)$ estremale, non può essere combinazione lineare di altri vettori.
				In altri termini: ogni $v(E_{j})$ è multiplo di $v(E)$,
				ma questo significa che hanno stessa $Z$-pendenza 
				$\nu_{\omega,B}(E) = \nu_{\omega,B}(E_{j})$ anche fuori dal muro.
				Per il \textbf{\Cref{thm:defo}}, se ci spostiamo poco dal muro $\Ww$,
				allora gli $E_{j}$ continuano a essere oggetti nei cuori delle condizioni di stabilità,
				e quindi troviamo un sottoggetto che destabilizza $E$, contraddicendo
				l'ipotesi di $\sigma$-stabilità.
			\end{proof}			  
		\end{cor}
		
		\begin{lemma}
			Siano $B \in NS(X)^{\R}, \omega = \alpha H$ e $E \in \Coh^{\alpha H , B}(X)$ un oggetto
			$(\alpha H, B)$-sesmistabile per ogni $\alpha \gge 0$.
			Allora vale una delle seguenti:
			\begin{enumerate}[label=\roman*)]
				\item $\Hh^{-1}(E) = 0$ e $\Hh^{0}(E)$ è libero da torsione e $\mu_{\alpha H,B}$-semistabile;
				\item $\Hh^{-1}(E) = 0$ e $\Hh^{0}(E)$ è di torsione;
				\item $\Hh^{-1}(E)$ è libero da torsione e $\mu_{\alpha H,B}$-semistabile,
				mentre $\Hh^{0}(E)$ è un fascio di torsione, supportato in dimensione $0$.
			\end{enumerate}
		\end{lemma}
		
		\begin{prop}[\textbf{Large Volume Limit}]
			Poniamo $B = \beta H + B_{0}$ in $NS(X)_{\R}$.
			Sia $v \in K_{num}(X)$ tale che $\ch_{0}(v) > 0$ e $H \cdot \ch_{1}^{B}(v) > 0$.
			Allora esiste $\alpha_{0} > 0$ tale che, per ogni $\alpha > \alpha_{0}$
			si abbia
			\begin{equation*}
				\Set{E \in \Coh^{\beta}(X) \, | \, \substack{v(E) = v\,, \\
				E \text{ è } \sigma_{\alpha,\beta}\text{-sst.}}}
				= \Set{E \in \Coh(X) \, | \,  \substack{v(E) = v\,, \\
				E \text{ è } (H,B_{0}-\frac{1}{2}K_{X})\text{-\textbf{Gieseker} sst.}}}\,.
			\end{equation*}
			L'uguaglianza vale anche sugli stabili.			 
		\end{prop}
		
		\begin{cor}
			Sia $E$ fibrato vettoriale $\mu_{\omega,B}$-stabile
			tale che $\overline{\Delta}_{\omega}^{B}(E) = 0$.
			Allora $E$ è $\sigma_{\omega,B}$-stabile.
			\begin{proof}
				Siccome $E$ è un fibrato vettoriale, allora non può avere torsione,
				il che implica $\ch_{0}(E) > 0$ e $H \cdot \ch_{1}^{B} = \Im Z_{\omega, B}(E) > 0$.
				Per il \textbf{Large Volume Limit}, $E$ o $E[1]$ è
				$\sigma_{\alpha H,B}$-stabile per $\alpha \gge 0$,
				ma allora per il \textbf{Corollario} precedente è stabile anche rispetto
				alla condizione $\sigma_{\omega, B}$.
			\end{proof}
		\end{cor}
		
		\begin{rmk}
			Sia $X$ una superficie liscia e $B = \beta H$. Sia $L = \Oo_{X}(mH)$.
			Allora
			\begin{align*}
				\overline{\Delta}_{H}^{B}(L)
				&= \left( H \cdot (mH - B) \right)^{2} - 2H^{2}\left( \frac{m^{2}H^{2}}{2}- mH \cdot B + \frac{B^{2}}{2} \right) \\
				&= (mH^{2} - B \cdot H)^{2} - m^{2}(H^{2})^{2} + 2m (B \cdot H) H^{2}
				- B^{2}H^{2}  \\
				&= (B \cdot H)^{2} - B^{2} H^{2} = 0\,.
			\end{align*}
			Quindi, a meno di shift, $L$ è $\sigma_{\alpha,\beta}$-stabile su tutto il piano 
			$(\alpha,\beta)$. Per capire quale shift di $L$ è stabile, mi basta capire
			da che parte del muro $\beta = m$ giace $\sigma$: infatti,
			se $\beta< m$, allora $\Oo_{X}(mH)$ è $\sigma$-stabile, 
			poiché $\mu_{\alpha H}(L)>\beta$, quindi
			so di essere in $\Tt_{\omega,B}$; viceversa, se
			$\beta \ge m$, allora $\Oo_{X}(mH)[1]$ è $\sigma$-stabile.
			
			\textcolor{red}{DISEGNO!}
		\end{rmk}
		
		\section{Applicazioni}
		
		L'obiettivo finale di oggi è quello di 
		dimostrare il \textbf{Kodaira Vanishing} con queste tecniche e, 
		tempo permettendo, studiare gli schemi di Hilbert.
		
		\begin{prop}
			Sia $X$ una superficie. Per ogni $p > 0$ e per ogni fibrato ampio $H$ si ha
			\begin{equation}
				H^{p}(\Oo_{X}(H + K_{X})) = 0\,.
			\end{equation}
			\begin{proof}
					Per \textbf{Dualità di Serre} sappiamo che
					\begin{equation}
						H^{2}(H+K_{X}) = H^{0}(-H) = 0\,.
					\end{equation}
					Quindi studiamo
					\begin{equation}
						H^{1}(H+K_{X}) = H^{1}(-H) = \Hom(\Oo_{X}, \Oo_{X}(-H)[1])\,.
					\end{equation}
					
					Per il precedente \textbf{Remark}, siccome $\Oo_{X}$ ha pendenza $0$
					e $\Oo_{X}(-H)$ ha pendenza $-1$, quindi nella fascia $-1 < \beta < 0$
					sia $\Oo_{X}$, sia $\Oo_{X}(-H)[1]$ sono $\sigma_{\alpha,\beta}$-stabili.
					
					Adesso vogliamo trovare $(\alpha, \beta)$ tali che
					\begin{equation}
						\nu_{\alpha, \beta}(\Oo_{X}) > \nu_{\alpha, \beta}(\Oo_{X}(-H)[1]) 
						= \nu_{\alpha, \beta}(\Oo_{X}(-H))\,,
					\end{equation}
					in modo da dedurre l'annullamento della prima coomologia e concludere.
					
					Un conto mostra che
					\begin{equation}
						\nu_{\alpha, \beta}(\Oo_{X}) = \frac{\alpha^{2} - \beta^{2}}{2\alpha \beta}\,,
						\quad \nu_{\alpha,\beta}(\Oo_{X}(-H)) = 
						\frac{\alpha^{2} - (\beta + 1)^{2}}{2\alpha (\beta + 1)}\,.
					\end{equation}
					quindi risolvendo la disequazione
					\begin{equation}
						\frac{\alpha^{2} - \beta^{2}}{2\alpha \beta} >
						\frac{\alpha^{2} - (\beta + 1)^{2}}{2\alpha (\beta + 1)}
					\end{equation}
					otteniamo il semicerchio $\alpha^{2} + (\beta + 1)^{2} < \frac{1}{4}$.
					Quindi, ad esempio, la condizione $(\frac{1}{10}, -\frac{1}{2})$ ci dà
					il vanishing desiderato!
					
					\textcolor{red}{DISEGNO!}
			\end{proof}
		\end{prop}
		
    	
    	
    \chapter{Introduzione alla geometria birazionale}
    
    	
	\lecture[.]{2025-11-12}
		
		\section{Forme differenziali}
		
		Sia $X$ varietà quasi-proiettiva. Definiamo
		\begin{equation}
			\Phi^{r}[X] := \Set{\phi : X \to \bigsqcup_{x \in X} \bigwedge^{r} T^{*}_{X,x}
			\, | \, \forall_{x \in x} \, \phi(x) \in \bigwedge^{r} T^{*}_{X,x}}\,.
		\end{equation}
		
		\begin{df}
			Una $r$-forma differenziale $\phi \in \Phi^{r}[X]$ si dice \textbf{regolare}
			se, per ogni $x \in X$, esiste un aperto affine $U \subset X$ tale che $\phi\vert_{U}$
			appartiene al $k[U]$-sottomodulo di $\Phi^{r}[X]$
			generato da elementi della forma
			\begin{equation}
				df_{1} \wedge \dots \wedge df_{r}\,, \quad \text{con } f_{j} \in k[U]\,,
			\end{equation}
			e indichiamo l'insieme delle $r$-forme regolari con $\Omega^{r}(X)$.
			Se $\omega \in \Omega^{r}(X)$, allora in un aperto affine $U$ si scrive come
			\begin{equation}
				\omega = \sum_{i_{1}, \dots, i_{r}} g_{i_{1}, \dots, i_{r}} df_{i_{1}} \wedge \dots \wedge df_{i_{r}}\,,
			\end{equation}
			con $g_{*}, f_{j} \in k[U]$.
		\end{df}
		
		\begin{thm}
			Sia $p \in X$ un punto liscio. Allora esiste un aperto affine $U \subset X$ tale che
			$\Omega^{1}(U)$ è un $k[U]$-modulo libero di rango $\dim \Oo_{X,x} =: n$.
		\end{thm}
		
		\begin{thm}
			Sia $p \in X$ un punto liscio. Allora esiste un aperto affine $U \subset X$ tale che
			$\Omega^{r}(U)$ è un $k[U]$-modulo libero di rango $\binom{n}{r}$ e una base
			per il modulo è data da
			\begin{equation}
				\Set{du_{i_{1}} \wedge \dots \wedge du_{i_{r}} \, | \, 1 \le i_{1} < \dots < i_{r} \le r}\,,
			\end{equation}
			dove gli $u_{j}$ sono presi da una base di $k[U]$ fissata.
		\end{thm}
		
		In analogia a quanto accade per le funzioni razionali, 
		introduciamo le forme differenziali \emph{razionali},
		che andremo a identificare con una relazione d'equivalenza.
		Consideriamo le coppie $(\omega, U)$, con $\omega \in \Omega^{r}(X)$ e $U \subset X$ un aperto.
		Diciamo che $(\omega, U) \sim (\omega', U')$ se $\omega\vert_{U \cap U'} = \omega'\vert_{U \cap U'}$.
		\begin{df}
			Una classe di equivalenza rispetto a $\sim$ è detta \textbf{$r$-forma differenziale razionale} su $X$.
		\end{df}
		
		Denotiamo con $\Mm^{r}(X)$ l'insieme delle $r$-forme razionali su $X$.
		È facile vedere che $\Mm^{r}(X)$ è un $k(X)$-spazio vettoriale di dimensione $\binom{n}{r}$,
		in analogia a quanto accade per le forme regolari. Inoltre, $\Mm^{r}(X)$ è un inveriante birazionale.
		
		\begin{df}
			Una $r$-forma razionale $\omega$ che ha un rappresentante della forma $(\omega,U)$
			si dice \textbf{regolare su $U$}.
		\end{df}
		
		\begin{oss}
			L'insieme dei punti su cui una $r$-forma razionale $\omega$ \textbf{non} è regolare
			forma un chiuso in $X$.
		\end{oss}
		
		\begin{ex}
			Su $\AA^{n}$, denotiamo con $z_{1}, \dots, z^{n}$ le coordinate.
			Ogni $\omega \in \Mm^{n}(X)$ si scrive unicamente come
			\begin{equation}
				\omega = \phi(z) \, dz_{1} \wedge \dots \wedge dz_{n}\,,
			\end{equation}
			dove $\phi(z) \in k(z_{1}, \dots z_{n})$ è una funzione razionale su $\AA^{n}$,
			che si scrive come $\phi = f/g$, per qualche $f,g \in k[z_{1}, \dots, z_{n}]$.
			L'insieme su cui $\omega$ non è regolare è l'ipersuperficie $\VV(g)$.
		\end{ex}
		
		\begin{lemma}
			$\Omega^{n}(\PP^{n}) = 0$.
			\begin{proof}
				Siano $[x_{0} : \dots : x_{n}]$ coordinate omogenee. Considero due carte affini
				\begin{equation}
					(y_{1}, \dots, y_{n}) = \left( \frac{x_{1}}{x_{0}}, \dots, \frac{x_{n}}{x_{0}} \right)
					\quad \text{e} \quad 
					(z_{0}, \dots, z_{n-1}) = \left( \frac{x_{0}}{x_{n}}, \dots, \frac{x_{n-1}}{x_{n}} \right) \,.
				\end{equation}
				Notiamo che sull'intersezione vale
				\begin{equation}
					y_{1} = \frac{z_{1}}{z_{0}}\,, \dots \,, y_{n-1} = \frac{z_{n-1}}{z_{0}}\,,
					y_{n} = \frac{1}{z_{0}}\,,
				\end{equation}
				quindi possiamo calcolare
				\begin{equation}
					dy_{1} \wedge \dots \wedge dy_{n} 
					= d \left( \frac{z_{1}}{z_{0}} \right) \wedge \dots \wedge d \left( \frac{1}{z_{0}}
					 \right)
					 = \dots = \frac{(-1)^{n}}{z_{0}^{n+1}} \, dz_{0} \wedge \dots \wedge dz_{n-1}\,.
				\end{equation} 
				Poiché  ogni $n$-forma regolare sulla prima carta affine $\AA^{n}_{y}$ è della forma
				$f(x) dy_{1} \wedge \dots \wedge dy_{n}$, su $\AA^{n}_{z}$ si scrive
				\begin{equation}
					\omega\vert_{\AA^{n}_{z}} = 
					\frac{(-1)^{n}}{z_{0}^{n+1}} 
					f \left(\frac{z_{1}}{z_{0}}, \dots , \frac{z_{n-1}}{z_{0}},
					\frac{1}{z_{0}} \right) \, dz_{0} \wedge \dots \wedge dz_{n-1}\,,
				\end{equation}
				che però non è regolare in $\{z_{0} = 0\}$.
			\end{proof}
		\end{lemma}
		
		\begin{df}
			Sia $X$ una varietà liscia $n$-dimensionale. Un divisore $K_{X} \in CaDiv(X)$
			si dice \textbf{canonico} se $K_{X} = \div \omega$, 
			con $\omega \in \Mm^{n}(X) \setminus \{0\}$.
			Chiamiamo \textbf{il divisore canonico} un rappresentante 
			della classe di equivalenza di $K_{X}$.
		\end{df}
		
		\begin{df}
			Il fascio $\Omega^{1}_{X}$ delle $1$-forme differenziali regolari su $X$ è il
			fascio che a ogni aperto $U \subset X$ associa il $k[U]$-modulo $\Omega^{1}(U)$.
			Il fascio delle $r$-forme regolari su $X$ verrà denotato con 
			$\Omega_{X}^{r} := \bigwedge^{r} \Omega_{X}^{1}$, e per $r=n$ otteniamo
			il \textbf{fascio canonico} $\omega_{X} := \Omega_{X}^{n}$.
		\end{df}
		
		\begin{oss}
			Vale $\omega_{X} \simeq \Oo(K_{X})$.
			Infatti, sia $\omega$ tale che $K_{X} = \div \omega$ e definiamo il morfismo di fasci
			$\phi : \Oo(K_{X}) \to \omega_{X}$ come la mappa
			che sull'aperto $U \subset X$ è data da
			\begin{equation}
				\Oo(K_{X})(U) \longrightarrow \Omega^{n}_{X}(U)\,,
				\quad f \longmapsto f \omega\vert_{U}\,.
			\end{equation}
			Allora $\phi$ è ben definita: infatti
			\begin{equation}
				\div(f \omega\vert_{U}) = \div f + \div \omega_{X} = \div f + K_{X}\vert_{U} \ge 0\,, 
			\end{equation}
			poiché $f \in \Oo_{X}(K_{X})(U)$. Quindi deduciamo che $f \omega\vert_{U}$ è regolare.
			L'iniettività segue dal fatto che $f \omega\vert_{U} =0$ se e solo se $f=0$,
			dato che (si dimostra che) $\{ \omega= 0\}$ è un chiuso.
			Per la surgettività, si sfrutta che $\Oo_{X}(K_{X})$ è un fascio invertibile:
			infatti, sia $\eta \in \omega_{X}$ e consideriamo $\{U_{\alpha}\}_{\alpha}$ un
			ricoprimento banalizzante di $U$, i.e. per ogni $\alpha$ vale
			\begin{equation}
				\eta\vert_{U_{\alpha}} = f_{\alpha} \omega\vert_{U_{\alpha}}\,,
				 \quad f_{\alpha} \in k(X)\,.
			\end{equation}
			Se $U_{\alpha} \cap U_{\beta} \ne \emptyset$, allora
			$\eta\vert_{U_{\alpha} \cap U_{\beta}} =  f_{\alpha} \omega =  f_{\beta} \omega$,
			e poiché $\omega \ne 0$ su questa intersezione, $f_{\alpha} = f_{\beta}$;
			da questo deduciamo che possiamo incollare le $f_{\alpha}$ a un elemento $f \in k(X)$.
			...
		\end{oss}
		
		\begin{ex}
			Sia $X = \PP^{n}$. La $n$-forma razionale
			\begin{equation}
				dx_{1} \wedge \dots \wedge dx_{n} 
			\end{equation}			 
			è regolare sull'aperto $\DD(x_{0}) = \{ x_{0} \ne 0 \}$.
			Nell'intersezione $U_{0} \cap U_{1}$ abbiamo le coordinate
			\begin{equation}
				\left( x_{0}, 1, x_{2}, \dots, x_{n} \right) = 
				\left( 1, \frac{1}{x_{0}}, \frac{x_{2}}{x_{0}}, \dots, \frac{x_{n}}{x_{0}} \right)\,,
			\end{equation}
			da cui deduciamo
			\begin{equation}
				dx_{1} \wedge \dots \wedge dx_{n} 
				= \dots = \frac{(-1)^{n}}{x_{0}^{n+1}} dx_{0} \wedge dx_{2} \wedge \dots \wedge dx_{n} 
			\end{equation}
		...
		\end{ex}
		
		\begin{ex}
			Sia $X \subset \PP^{n}$ una ipersuperficie di grado $d$, 
			definita dall'equazione polinomiale $P(x)=0$.
			Consideriamo la $(n-1)$-forma definita su $U_{0} \cap X$ da
			\begin{equation}
				\omega_{i} := (-1)^{i} \frac{dx_{1} \wedge \dots \wedge \widehat{dx_{i}} \wedge \dots\wedge dx_{n} }{\frac{\partial P}{\partial x_{i}}(1, x_{1}, \dots, x_{n})}\,.
			\end{equation}
			Considero $0 \le i \le n$ tale che $\partial P/\partial x_{i} \ne 0$
			e usiamo la liscezza di $P$. Se esiste un altro indice $j$ in cui la derivata lungo $x_{j}$
			non si annulla, allora posso definire $\omega_{j}$ e si verifica con un conto
			che $\omega_{j} = \omega_{i}$.
			
			Sia $X$ liscia e $U = \bigcup_{i} U_{i}$, 
			dove $U_{i}$ sono aperti in cui $\partial P/\partial x_{i} \ne 0$.
			Allora possiamo incollare le forme $\omega_{i}$ in modo da definire 
			$\omega \in \Omega^{n-1}(U_{0})$. Sull'aperto $U_{0} \cap U_{1}$ possiamo scrivere
			\begin{equation}
				\frac{d \left(\frac{1}{x_{0}} \right) \wedge \left(\frac{x_{3}}{x_{0}} \right)
				\wedge \dots \wedge  d\left(\frac{x_{n}}{x_{0}}\right) }{\frac{\partial P}{\partial x_{2}}\left( 1, \frac{1}{x_{0}}, \frac{x_{2}}{x_{0}}, \dots, \frac{x_{n}}{x_{0}} \right)}
				= \frac{(-1)^{n-d}}{x^{n-(d-1)}}\frac{dx_{0} \wedge dx_{3} \wedge \dots \wedge dx_{n} }{\frac{\partial P}{\partial x_{2}}(x_{0},1, x_{2}, \dots, x_{n})}
			\end{equation}
			e quindi vediamo che $K_{X} = -(n+1-d) H_{0} \cap X$.
		\end{ex}
		
		\begin{oss}
			Se $d=n+1$, allora $K_{X} \sim \Oo_{X}$ e quindi l'ipersuperficie è Calabi-Yau.
			Se $d<n+1$, allora $K_{X}$ è anti-ampio e quindi $X$ è Fano;
			viceversa, se $d>n+1$ il canonico $K_{X}$ è ampio, quindi $X$ è di tipo generale.
		\end{oss}
		
		\begin{ex}[\textbf{Blow-up}]
			Sia $X$ una varietà liscia $n$-dimensionale. Sia $Y \subset X$ liscia di codimensione $r$
			e consideriamo il blow-up
			\begin{equation}
				\pi : \widetilde{X} := \operatorname{Bl}_{Y} X \longrightarrow X
			\end{equation}
			con divisore eccezionale $E = \pi^{-1}(Y)$. Allora
			\begin{equation}
				K_{\widetilde{X}} = \pi^{*}K_{X} + (r-1) E\,.
			\end{equation}
			Per dimostrarlo, ragioniamo localmente con le forme differenziali.
			Essendo $X,Y$ lisce, scegliamo coordinate locali $x_{1}, \dots, x_{n}$ su $X$,
			in modo tale che $Y = \Set{x_{1} = \dots = x_{r} = 0}$.
			Quindi, le coordinate del blow-up sono
			\begin{equation}
				\widetilde{X} = \operatorname{Bl}_{Y} X
				= \Set{ \big( (x_{1}, \dots, x_{n}), [u_{1}: \dots: u_{n}] \big) \in X \times \PP^{r-1}\,
				| \, \forall_{1 \le i < j \le r} \, x_{i}x_{j} = x_{j}x_{i} }\,.
			\end{equation}
			Localmente, $\widetilde{X}$ è coperto da carte affini $U_{j} = {u_{j} \ne 0}$;
			in $U_{1}$ e coordinate sono
			\begin{equation}
				x_{1}\,, \quad x_{2} = u_{2} x_{1}\,, \dots\,, x_{r} = u_{r} x_{1}\,, x_{r+1}\,, \dots\,, x_{n}\,, \quad  \text{con } u_{j} = \frac{x_{j}}{x_{1}}\,, 
			\end{equation}
			e quindi il blow-up e localmente
			\begin{equation}
				\pi(x_{1}, u_{2}, \dots, u_{r}, x_{r+1}, \dots, x_{n}) = 
				(x_{1}, u_{2}x_{1}, \dots, u_{r}x_{1}, x_{r+1}, \dots, x_{n})\,.
			\end{equation}
			Considero la $n$-forma regolare $\omega = dx_{1} \wedge \dots \wedge dx_{n}$ su $X$.
			Tirandola indietro tramite il pull-back, si ottiene
			\begin{align*}
				\pi^{*} \left(  dx_{1} \wedge \dots \wedge dx_{n} \right)
				&=  dx_{1} \wedge d(u_{2}x_{1}) \wedge \dots \wedge d(u_{r}x_{1}) \wedge dx_{r+1} \wedge \dots\wedge dx_{n} \\
				&= x_{1}^{r-1}  dx_{1} \wedge du_{2} \wedge \dots \wedge du_{r} \wedge dx_{r+1} \wedge \dots\wedge dx_{n}
			\end{align*}
			...
		\end{ex}
		
		
		\section{La successione esponenziale}
		
			Sia $X$ una varietà proiettiva liscia; la consideriamo adesso come una varietà complessa,
			con la topologia euclidea.
			Sia $\Oo_{an}$ il fascio delle funzioni olomorfe su $X$ e $\Oo_{an}^{*}$
			il sottofascio delle funzioni olomorfe mai nulle.
			Abbiamo il seguente morfismo surgettivo di fasci di gruppi abeliani
			\begin{equation}
			\exp : \Oo_{an} \longrightarrow \Oo_{an}^{*}\,, \quad
			f \longmapsto e^{f}\,.
			\end{equation}
			Il nucleo è il fascio
			\begin{equation}
			\Kk er \exp = \Set{ f \in \Oo_{an} \, | \, \forall_{p \in X} f(p) \in 2\pi i \Z} 
			\simeq \Z_{an}\,,
			\end{equation}
			quindi abbiamo una successione esatta corta di fasci di gruppi abeliani 
			\begin{equation}\label{ses-exp}
				0 \longrightarrow \Z_{an} \longrightarrow \Oo_{an} \overset{\exp}{\longrightarrow}
				\Oo_{an}^{*} \longrightarrow 0\,,
			\end{equation}
			che chiameremo \textbf{successione esponenziale}.
			
			\begin{rmk}
				La coomologia del fascio costante $H^{*}(X, \Z_{an})$ è
				isomorfa alla coomologia singolare $H^{*}(X; \Z)$ di $X$ con topologia euclidea.
			\end{rmk}
			
			\begin{rmk}
				Il fascio $\Oo_{an}$ è coerente nella topologia complessa,
				e per un fatto molto profondo (i.e. il teorema \textbf{GAGA} di Serre),
				esiste un isomorfismo $H^{*}(X; \Oo_{an}) \simeq H^{*}(X, \Oo_{X})$.
				In questo modo, possiamo identificare $\Pic(X) \simeq H^{1}(X, \Oo_{an}^{*}$.	
			\end{rmk}
				
			Quindi, la successione \eqref{ses-exp} induce la successione esatta lunga in coomologia
			\begin{equation}
				0 \longrightarrow \Z \longrightarrow \C \overset{\exp}{\longrightarrow}
				\C^{*} \longrightarrow  H^{1}(X;\Z) \longrightarrow \dots
			\end{equation}
			Siccome $X$ è una varietà priettiva liscia, complessa e compatta, allora
			i gruppi di coomologia singolare $H^{i}(X; \Z)$ sono finitamente generati,
			dunque abbiamo il teorema di struttura per $\Z$-moduli che ci permette di decomporre
			\begin{equation}
				H^{i}(X;\Z) = \Z^{b_{i}(X)} \oplus T_{i}\,,
			\end{equation}
			dove $ T_{i} $è sottogruppo di torsione finito e $b_{i}(X)$ è l'$i$-esimo numero di Betti.
			
			\begin{oss}
				$T_{1} = 0$ poiché la mappa $\gamma$ è iniettiva, quindi
				$H^{1}(X; \Z) \subset H^{1}(X, \Oo_{X}) \simeq \C^{N}$ e quest'ultimo non ha torsione.
			\end{oss}
			
			\begin{df}
				Se $\Ll \in \Pic(X)$ la \textbf{prima classe di Chern} di $\Ll$ è
				$c_{1}(\Ll) \in H^{2}(X; \Z)$. Se $D$ è un divisore di Cartier, 
				scriveremo $c_{1}(D) = c_{1}(\Oo_{X}(D)$.
			\end{df}
			
			Si noti che il primo carattere di Chern è la ``\emph{giusta generalizzazione}'''
del \textbf{grado} di una curva; tuttavia, a differenza della funzione $\deg$,
in generale la $c_{1}$ \textbf{non} è surgettiva.

			\begin{df}
				Il \textbf{gruppo di Neron-Severi} di $X$ è l'immagine del primo carattere di Chern:
				$$NS(X) := \im c_{1} \subset 
				H^{2}(X; \Z)\,.$$
			\end{df}
			
			\begin{rmk}
				Il $NS(X)$ è un gruppo abeliano finitamente generato.
				Il suo rango $\rho_{X}$, detto \textbf{numero di Picard} di $X$, è un
				invariante numerico di $X$; inoltre, $\rho_{X}$ è \textbf{invariante birazionale}. 
				Si noti, per definizione, che $\rho_{X} \le b_{2}(X)$.
				Ad esempio, se $X=C$ è una curva, allora $NS(X) \simeq H^{2}(X;\Z) \simeq \Z$,
				quindi $\rho_{X} = b_{2}(X) = 1$.
				Il Neron-Severi misura la ``\emph{parte discreta}'' di $X$.
			\end{rmk}
			
			\begin{df}
				Il nucleo della prima classe di Chern verrà chiamato \textbf{varietà di Picard}
				\begin{equation}
					\Pic^{0}(X) := \ker c_{1} \subset \Pic(X)\,.
				\end{equation}
			\end{df}
			
			Dalla successione esponenziale \eqref{ses-exp}, si vede che
			\begin{equation}
				Pic^{0}(X) \simeq H^{1}(X, \Oo_{X})/\ker \beta \simeq H^{1}(X, \Oo_{X}) / H^{1}(X;\Z)\,,
			\end{equation}
			quindi notiamo che è un \textbf{toro complesso} di dimensione $h^{1}(\Oo_{X})$,
			per questo diciamo che misura la ``\emph{parte continua}'' di $X$.
			Ovviamente abbiamo la successione esatta corta di gruppi abeliani
			\begin{equation}\label{ses-pic}
				0 \longrightarrow \Pic^{0}(X) \longrightarrow \Pic(X)
				\overset{\exp}{\longrightarrow} NS(X) \longrightarrow 0\,.
			\end{equation}
			
			\subsection{Il caso delle curve}
			
				Sia $X=C$ una curva proiettiva liscia. Per la classificazione delle
				superfici topologiche, sappiamo che $C$ è omeomorfa a una sfera con $g$ manici,
				quindi
				\begin{equation}
					H^{0}(X; \Z) \simeq H^{2}(X; \Z)  \simeq \Z\,, \quad
					H^{1}(X; \Z) \simeq \Z^{2g}\,.
				\end{equation}
				A meno di scegliere il segno dell'isomorfismo $H^{2}(X; \Z) \simeq \Z$,
				allora $c_{1} = \deg$ e possiamo descrivere la varietà di Picard
				esplicitamente, poiché
				\begin{equation}
					\Pic^{0}(X) \simeq H^{1}(X, \Oo_{X})/H^{1}(X; \Z) 
					\simeq \C^{h^{1}(\Oo_{X})}/{\Z^{2g}} \simeq \C^{g}/{\Z^{2g}}\,,
				\end{equation}
				quindi è una varietà abeliana $g$-dimensionale.
    	
    	
	\lecture[Invarianza birazionale dei numeri di Hodge. Introduzione alla teoria dell'intersezione tra curve e divisori.]{2025-11-19}
		
		\section{Invarianza birazionale dei numeri di Hodge esterni}
		
		Sia $X$ una varietà proiettiva con $\codim_{X} \operatorname{Sing}(X) \ge 2$,
		sia $Y$ proiettiva e $f:X \to Y$ una mappa razionale tale 
		che $\codim_{X} (X \setminus \operatorname{dom} f) \ge 2$.
		Allora il dominio di $f$ corrisponde al luogo dei punti su cui $f$ è regolare.
		
		\begin{proof}
			Come primo caso, supponiamo che $X$ sia liscia e $Y \subset \PP^{N}$.
			Dato che $X$ è liscia, per ogni $x \in X$ l'anello locale $\Oo_{X,x}$ è un UFD.
			In un intorno di $x \in X$, possiamo scrivere
			\begin{equation}
				f = (f_{0} : \dots : f_{N})\,, \quad \text{ con } f_{j} = \frac{g_{j}}{h_{j}} \in k(X)\,,
			\end{equation}
			dove $g_{j}, h_{j} \in \Oo_{X,x}$ sono primi tra loro, per ogni $j = 0, \dots, N$.
			A meno di moltiplicare per il denominatore comune, possiamo supporre $f_{j} \in \Oo_{X,x}$,
			quindi
			\begin{equation}
				(X \setminus \operatorname{dom} f) \cap U \subset \VV(f_{0}, \dots, f_{N})\,.
			\end{equation}
			Si noti che l'insieme algebrico sulla destra non può contenere divisori,
			altrimenti se esistesse un divisore primo per $x$, detto $D \in   \VV(f_{0}, \dots, f_{N})$,
			allora l'equazione locale di $D$ in $x$ dividerebbe tutti gli $f_{j}$,
			contraddicendo l'ipotesi di primarietà fatta in precedenza.
			
			In generale, se $\codim_{X} \operatorname{Sing}(X) \ge 2$, allora possiamo usare che
			\begin{equation}
				(X \setminus \operatorname{dom} f) \cap U \subset  \operatorname{Sing}(X)
				\cup \overline{(X_{reg} \setminus \operatorname{dom} f)}\,,
			\end{equation}
			e vedere che gli insiemi sulla destra non contengono divisori, da cui si conclude che
			$\codim_{X} (X \setminus \operatorname{dom} f) \ge 2$.
		\end{proof}
		
		\begin{cor}
			Siano $X$ e $Y$ due curve proiettive lisce. Allora $X \simeq Y$ se e solo se
			$X$ e $Y$ sono birazionalmente equivalenti.
		\end{cor}
		
		\begin{rmk}
			Se $f : X \dashrightarrow Y$ mappa birazionale,
			esistono due aperti $U \subset X$ e $V \subset Y$ tali che $f : U \simeq V$.
			In generale, può accadere che $U \subsetneq \operatorname{dom} f$.
		\end{rmk}
		
		\begin{ex}[\textbf{Mappa di Cremona standard}]
			Consideriamo la mappa
			\begin{equation}
				f : \PP^{2} \dashrightarrow \PP^{2}\,, \quad 
				f\left([x_{0}: x_{1}: x_{2}] \right) := \left[ x_{1}x_{2} : x_{0}x_{2} : x_{0}x_{1} \right] = \left[ \frac{1}{x_{0}} : \frac{1}{x_{1}}  : \frac{1}{x_{2}}\right] \,.  
			\end{equation}
			Il dominio di $f$ è dato da $\PP^{2} \setminus \Set{[1:0:0], [0:1:0], [0:0:1]}$
			e un aperto su cui $f$ è isomorfismo è dato da
			\begin{equation}
			 	U = \Set{[x_{0}: x_{1}: x_{2}] \in \PP^{2} \, | \, x_{0}x_{1}x_{2} \ne 0}\,.
			\end{equation}
			Quando parleremo di risoluzione delle indeterminazioni, vedremo che
			\begin{equation}
				\begin{tikzcd}
					& \operatorname{Bl}_{p_{1},p_{2},p_{3}}(\PP^{2})
					 \ar[dl, "\operatorname{bl}"'] \ar[dr] & \\
					\PP^{2} \ar[rr, dashed, "f"] & & \PP^{2}
				\end{tikzcd}
			\end{equation}
		\end{ex}
		
		\begin{oss}
			Data $X$ proiettiva liscia, $\omega$ una $p$-forma razionale su $X$, allora il chiuso in cui $\omega$ non è regolare è un divisore, cioè ha codimensione $1$. Infatti, basta verificarlo su un 
			aperto in coordinate locali $x_{1}, \dots, x_{n}$, dove possiamo scrivere
			\begin{equation}
				\omega = \sum_{i_{1}< \dots < i_{n}} f_{i_{1}, \dots, i_{n}} \, dx_{i_{1}} \wedge \dots \wedge dx_{i_{n}}\,,
			\end{equation}
			e notiamo che $\omega$ è regolare esattamente dove le funzioni $f_{*}$ sono regolari.
			Invece, il luogo in cui $\omega$ non è regolare è localmente 
			l'unione dei divisori dei poli delle $f_{*}$.
		\end{oss}
		
		\begin{cor}
			Se $U \subset X$ è un aperto tale che $X \setminus U$ 
			abbia codimensione almeno $2$, allora la restrizione
			\begin{equation}
				r : H^{0}(X, \Omega^{p}_{X}) \longrightarrow H^{0}(U, \Omega^{p}_{U})\,,
				\quad \omega \longmapsto \omega\vert_{U}\,,
			\end{equation}
			è un isomorfismo.
			\begin{proof}
				La mappa è ben definita, è un omomorfismo di gruppi ed è iniettiva:
				se $\omega$ è una forma tale che $\omega\vert_{U} = 0$, su $U$ un aperto denso,
				allora è nulla anche su $X$. La surgettività...
			\end{proof}
		\end{cor}
		
		\begin{prop}
			Sia $X$ una varietà proiettiva liscia\footnote{Ma funziona più in generale, ad esempio per varietà normali.}. I numeri di Hodge $h^{0,p}$ sono invarianti birazionali, dove ricordiamo che
			\begin{equation}
				h^{0,p} := \dim H^{p}(X, \Oo_{X}) = \dim H^{0}(X, \Omega_{X}^{p}) =: h^{p,0}\,. 
			\end{equation}
			\begin{proof}
				Siano $X$ e $Y$ varietà birazionali; dimostriamo che esiste un isomorfismo
				\begin{equation}
					H^{0}(X, \Omega_{X}^{p}) \simeq H^{0}(Y, \Omega_{Y}^{p})\,.
				\end{equation}
				Prendiamo $f : \dashrightarrow Y$ e $g : Y \dashrightarrow X$ inverse birazionali,
				con $U \subset X$ e $V \subset Y$ aperti su cui $f$ e $g$ sono isomorfismi.
				Allora abbiamo il diagramma commutativo
				\begin{equation}
					\begin{tikzcd}
				{H^{0}(Y, \Omega_{Y}^{p})} \arrow[r, "f^{*}"] \arrow[rd, "r_{Y}"', "\simeq"] 
				& {H^{0}(\operatorname{dom}(f), \Omega_{\operatorname{dom}(f)}^{p})} 
				& {H^{0}(X, \Omega_{X}^{p})} \arrow[l, "r_{X}", "\simeq"'] \arrow[ld, "g^{*}"] \\
                & {H^{0}(\operatorname{dom}(g), \Omega_{\operatorname{dom}(g)}^{p})} & \,.                                                             
				\end{tikzcd}
				\end{equation}
				Notiamo che le restrizioni $r_{X}$ e $r_{Y}$ sono isomorfismi 
				per le osservazioni precedenti, quindi consideriamo 
				$\alpha := r_{X}^{-1} \circ f^{*}$ e $\beta := r_{Y}^{-1} \circ g^{*}$.
				Dimostriamo che sono omomorfismi mutuamente inversi: 
				per vedere che $\beta \circ \alpha = \id$, 
				consideriamo $\omega \in H^{0}(Y, \Omega_{Y}^{p})$.
				
				...
			\end{proof}
		\end{prop}
		
		È importante tenere a mente che gli altri numeri di Hodge \textbf{non} sono invarianti birazionali.
		
		\begin{ex}
			Consideriamo il blow-up $\operatorname{Bl}_{p} \PP^{2}$ 
			del piano proiettivo in un punto $p \in \PP^{2}$. 
			Questo è birazionale a $\PP^{2}$, ma possiamo vedere dal diamante di Hodge che
			\begin{equation}
				h^{1,1} = b_{2}(\PP^{2}) = 1\,, 
				\quad h^{1,1} = b_{2}(\operatorname{Bl}_{p} \PP^{2}) = 2\,.
			\end{equation}
		\end{ex}
		
		
		
		
		
	\section{Intersezione tra divisori e curve}
	
		Per $X$ una curva proiettiva liscia, abbiamo la funzione \textbf{grado}
		\begin{equation}
			\deg : \Pic(X) \longrightarrow \Z\,, \quad \sum_{i} m_{i} D_{i} \longmapsto \sum_{i} m_{i}\,.
		\end{equation}
		Vogliamo qualcosa di simile per una varietà $X$ di dimensione qualsiasi.
		Dato $D$ un divisore di Cartier su $X$ e $C$ una curva irriducibile in $X$,
		vogliamo definire un \textbf{prodotto d'intersezione}
		\begin{equation}
			D \cdot C \in \Z\,.
		\end{equation}
		Dato che $D$ è Cartier, $\Oo_{X}(D) \in \Pic(X)$. Ora:
		\begin{itemize}
			\item se $C$ è \textbf{liscia}, allora $\Oo_{X}(D)\vert_{C} \in \Pic(C)$ e quindi
			ha un grado ben definito, per cui poniamo $D \cdot C := \deg \Oo_{X}(D)\vert_{C}$.
			Osserviamo che, se $C \nsubseteq \operatorname{supp} D$, allora $D\vert_{C}$ è un
			divisore su $C$ e $\deg D\vert_{C} = D \cdot C$, 
			usando il fatto che $\Oo_{C}(D\vert_{C}) = \Oo(D)\vert_{C}$;
			
			\item se $C$ è \textbf{singolare}, prendo la normalizzazione della curva 
			$\widetilde{\nu}:C^{\nu} \overset{\nu}{\to} C \hookrightarrow X$
			e poniamo $D \cdot C = \deg_{C^{\nu}}(\widetilde{\nu}^{*} \Oo_{X}(D))$.
		\end{itemize}
		
		\begin{oss}
			Ricapitolando, per un divisore di Cartier $D$ e una curva irriducibile $X$ abbiamo i casi:
			\begin{enumerate}
				\item ...
			\end{enumerate}
		\end{oss}
		
		Denoteremo con $Z_{1}(X)$ il gruppo abeliano degli $1$-cicli su $X$,
		i cui elementi sono della forma $\sum_{i} m_{i} C_{i}$, con $m_{i} \in \Z$ e $C_{i}$
		curva irriducibile su $X$. Abbiamo un'applicazione bilineare
		\begin{equation}
			\operatorname{CaDiv}(X) \times Z_{1}(X) \longrightarrow \Z\,,
			\quad \left( D, \sum_{i} m_{i} C_{i} \right) \longmapsto  \sum_{i} m_{i} (D \cdot C_{i})\,.
		\end{equation}
		
		\begin{df}
			Due divisori di Cartier $D_{1}$ e $D_{2}$ su $X$ sono \textbf{numericamente equivalenti}
			se, per ogni $C \subset X$ curva irriducibile, vale $D_{1} \cdot C = D_{2} \cdot C$.
			Scriviamo allora $D_{1} \equiv D_{2}$. 
		\end{df}
		
		Se $D_{1} \sim D_{2}$ allora $D_{1} \equiv D_{2}$, ma il viceversa è falso.
		
		\begin{ex}
			Sia $E$ la curva ellittica in $\PP^{2}$ definita dall'equazione omogenea
			$y^{2}z = x^{3} -xz^{2}$. Siano $O = [0:1:0], P=[0:0:1]$ e $Q=[1:0:1]$,
			allora $D_{1} = P - O$ e $D_{2} = Q-O$ sono due divisori con $\deg D_{1} = \deg D_{2} = 0$,
			quindi $D_{1} \equiv D_{2}$. Tuttavia, su una curva ellittica abbiamo la 
			\textbf{mappa di Abel-Jacobi}, cioè l'isomorfismo\footnote{In generale, se $g(C) > 1$ la mappa è definita, ma vale solo l'iniettività.}
			\begin{equation}
				a : E \overset{\sim}{\longrightarrow} \Pic^{0}(E)\,, \quad p \longmapsto \Oo_{E}(p-O) \,.
			\end{equation}
			Quindi, dato che $P \ne Q$, per iniettività di $a$ segue che $a(Q) \ne a(P)$,
			da cui deduciamo che $P-O \nsim Q-O$, quindi i due divisori non sono linearmente equivalenti.
		\end{ex}
		
		\begin{prop}
			Sia $X$ una varietà proiettiva liscia. Allora $D \equiv 0$ se e solo se
			$c_{1}(D)$ è di torsione in $H^{2}(X;\Z)$.
			\begin{proof}[Idea della dimostrazione]
				È un risultato profondo, 
				per cui bisogna combinare i risultati giusti della Teoria di Hodge:
				per il \textbf{Teorema $(1,1)$ di Lefschetz}, 
				sappiamo che $c_{1}(D) \in H^{1,1}(X) \cap H^{2}(X;\Z)$.
				Usando la \textbf{dualità di Poincaré} e il \textbf{Teorema dell'indice di Hodge},
				segue che $c_{1}(D) = 0$ in $H^{2}(X;\R)$, da cui segue che $c_{1}(D)$ è di
				torsione quando si torna a coefficienti interi.
				
				Per l'implicazione opposta, consideriamo $D_{1}, D_{2}$ sono divisori su $X$
				tali che esista $m \in \N$  per cui $mc_{1}(D_{1}) = mc_{1}(D_{2})$.
				Per ogni $C \subset X$ curva liscia, consideriamo il diagramma commutativo
				\begin{equation}
					\begin{tikzcd}
						\Pic(X) \ar[r, "c_{1}"] \ar[d,"-\vert_{C}"'] 
						& H^{2}(X;\Z) \ar[d] \\
						\Pic(C) \ar[r, "\deg"] & H^{2}(C;\Z) \simeq \Z\,,
					\end{tikzcd}
				\end{equation}
				grazie al quale si nota che $D_{1} \cdot C = \deg_{C}(\Oo_{X}(D_{j})\vert_{C})
				= c_{1}(D_{j})\vert_{C}$.
				Adesso, sapendo che
				\begin{equation}
					m \left( c_{1}(D_{1}) \vert_{C} \right) 
					= \left(m \, c_{1}(D_{1}) \vert_{C} \right)
					=  \left( m \, c_{1}(D_{2}) \vert_{C} \right) 
					= m \left( c_{1}(D_{2}) \vert_{C} \right)\,,
				\end{equation}
				ed essendo un'equazione in $\Z$, allora 
				$c_{1}(D_{1}) \vert_{C} = c_{1}(D_{2}) \vert_{C}$,
				ma essendo su una curva $C$ questo equivale a $D_{1} \cdot C = D_{2} \cdot C$.
			\end{proof}
		\end{prop}
		
		Come conseguenza, notiamo che tutto ciò che sta in $\Pic^{0}(X)$ è numericamente banale:
		\begin{equation}
			\Oo_{X}(D) \in \Pic^{0}(X) \quad \implies \quad D \equiv 0\,.
		\end{equation}
		Infatti, dalla successione esponenziale deduciamo che, se $\Pic^{0}(X) = 0$
		e $H^{2}(X;\Z)$ è senza torsione, allora l'equivalenza numerica coincide con l'equivalenza lineare. In particolare, se $X$ è \textbf{razionale}, allora $\sim$ coincide con $\equiv$:
		infatti, se $X$ è birazionale a $\PP^{n}$, per il \textbf{Vanishing di Kodaira} sappiamo che
		\begin{equation}
			h^{i}(\PP^{n}, \Oo_{\PP^{n}}) = 0\,, \quad \text{ se } i>0\,,
		\end{equation}
		quindi dall'invarianza birazionale dei $h^{0,p}$ si deduce che
		\begin{equation}
			h^{1}(X, \Oo_{X}) = 0 = h^{2}(X, \Oo_{X})\,,
		\end{equation}
		da cui abbiamo $\Pic^{0}(X) = 0$. Quindi la successione esponenziale mostra che
		\begin{equation}
			\Pic(X) \simeq NS(X) \simeq H^{2}(X;\Z)\,,
		\end{equation}
		che è un modulo libero, allora \textcolor{red}{???}
		
		\begin{df}
			Definiamo il reticolo
			\begin{equation}
				N^{1}(X)_{\Z} := \operatorname{Div}(X)/\equiv  \, = \, \Pic(X) / \equiv \,
				\simeq  \, NS(X)/torsione  \, \simeq  \, \Z^{\rho(X)}
			\end{equation}
			e poniamo $N^{1}(X) := N^{1}(X)_{\Z} \otimes \R = \R^{\rho(X)}$
			lo spazio dei divisori a coefficienti reali, modulo l'equivalenza numerica.
		\end{df}	
		
		Il prodotto d'intersezione $\operatorname{Div}(X) \times Z_{1}(X) \to \Z$
		passa al quoziente per l'equivalenza lineare, definendo un'applicazione bilineare non degenere
		\begin{equation}\label{eq:intersezione}
			- \cdot - \, : \, N^{1}(X) \times N_{1}(X) \longrightarrow \R\,,
		\end{equation}
		dove $N_{1}(X) := (Z_{1}/\equiv) \otimes \R$. 
		Si dimostra che $N_{1}(X)$ e $N^{1}(X)$ sono in dualità.
		
		\begin{ex}
			Sia $X = \PP^{n}$, $C \subset \PP^{n}$ una curva irriducibile e $H \subset \PP^{n}$ un iperpiano che non contiene $C$. In questo caso
			\begin{equation}
				C \cdot H = \deg H\vert_{C} = \lvert H \cap C \rvert\,.
			\end{equation}
			Dato che il primo carattere di Chern induce un isomorfismo
			\begin{equation}
				\Pic(\PP^{n}) \simeq \Z H \overset{\sim}{\longrightarrow} H^{2}(\PP^{n}; \Z)\,,
				quad \Oo(D) \longmapsto \deg D\,,
			\end{equation}
			e quindi l'equivalenza numerica coincide con quella lineare.
			Poiché per ogni divisore $D$ si ha $D \sim (\deg D)H$, vediamo che
			\begin{equation}
				D \cdot C = (\deg D) (\deg C)\,;
			\end{equation}
			posta $\ell \subset \PP^{n}$ una retta, 
			indichiamo $[\ell]$ la sua classe di equivalenza numerica, e quindi abbiamo
			\begin{equation}
				N^{1}(\PP^{n}) = \R[H]\,, \quad N_{1}(\PP^{n}) = \R[\ell]\,.
			\end{equation}
		\end{ex}
    	
    	
	\lecture[.]{2025-11-26}
	
		\section{Varietà unirigate e razionalmente connesse}
		
			Sia $X$ una varietà proiettiva su $\C$.
			
			\begin{df}
				Una curva irriducibile $C \subset X$ si dice \textbf{razionale}
				se la sua normalizzazione $C^{\nu} \simeq \PP^{1}$. 
			\end{df}
			
			Per quanto abbiamo visto nelle scorse lezioni,
			possiamo dire equivalentemente che una curva $C$ è razionale 
			se e solo se è birazionale a $\PP^{1}$: infatti, 
			se $C \dashrightarrow \PP^{1}$ è una mappa birazionale,
			componendo con la normalizzazione si ottiene $C^{\nu} \dashrightarrow \PP^{1}$
			mappa birazionale, che per le curve lisce è equivalente alla nozione di isomorfismo.
			
			\begin{oss}
				Un curva razionale $C$ è l'immagine di un morfismo non costante
				\begin{equation*}
					\begin{tikzcd}
						\PP^{1} \ar[rr] \ar[dr, "\nu"'] & & X \\
						& C \ar[ur, hook] & \,.
					\end{tikzcd}
				\end{equation*}
			\end{oss}
			
			\begin{df}
				Una varietà $X$ si dice \textbf{unirigata} se, per ogni $x \in X$,
				esiste una curva razionale $C$ che passa per $x$.
			\end{df}
			
			\begin{ex}
				Lo spazio proiettivo $X = \PP^{n}$ è una varietà unirigata.
			\end{ex}
			
			\begin{oss}
				Essere unirigata è un invariante birazionale.
			\end{oss}
			
			\begin{cor}\label{cor:razionale-unirigata}
				Se $X$ è razionale, allora $X$ è unirigata.
			\end{cor}
			
			\begin{ex}
				Il viceversa del \textbf{\Cref{cor:razionale-unirigata}} è falso:
				infatti, presa $C$ una curva proiettiva liscia di genere $g \ge 1$,
				allora $X = \PP^{1} \times C$ è unirigata, infatti basta considerare le curve
				della forma $C = \PP^{1} \times \{ pt. \}$, ma non può essere
				razionale per via dei suoi numeri di Hodge esterni: infatti,
				per dualità di Serre
				\begin{equation*}
					g = h^{1}(\Oo_{C}) =  h^{0}(\Omega_{C}^{1}) > 0\,,
				\end{equation*}	
				quindi esiste una $1$-forma regolare $\omega \ne 0$ su $C$;
				da questa otteniamo una $1$-forma regolare $\operatorname{pr}_{C}^{*}\omega \in H^{0}(X,\Omega_{X})$,
				da cui concludiamo che $h^{0,1}(X) \ne 0 = h^{0,1}(\PP^{2})$.
			\end{ex}
			
			Sia $f : \PP^{1} \to X$ un morfismo non costante. Per il \textbf{Teorema di Grothendieck}
			sappiamo che
			\begin{equation*}
				f^{*}\Tt_{X} \simeq \Oo(a_{1}) \oplus \dots \oplus \Oo(a_{n})\,,
				\quad \text{con } a_{1} \ge a_{2} \ge \dots \ge a_{n} \text{ interi}\,,
			\end{equation*}
			e dato che $\Tt_{\PP^{1}} = \Oo(2)$, possiamo dedurre che $a_{1} \ge 2$:
			infatti
			\begin{equation*}
				\Hom(\Oo(2), \Oo(a_{1}) \oplus \dots \oplus \Oo(a_{n}))
				= \bigoplus_{i=1}^{n} \Hom(\Oo(2), \Oo(a_{i})) 
				\simeq \bigoplus_{i=1}^{n} H^{0}(\PP^{1}, \Oo(a_{i}-2))\,;
			\end{equation*}
			affinché la mappa sia non nulla, abbiamo quindi bisogno che esista almeno un $a_{i} \ge 2$.
			
			\begin{df}
				Sia $r > 0$ intero. Una curva razionale $f : \PP^{1} \to X$ è \textbf{$r$-free}
				se $f^{*}\Tt_{X} \otimes \Oo(-r)$ è globalmente generato.
				Per $r=0$, chiamiamo $f$ \textbf{free}, per $r=1$ chiamiamo $f$ \textbf{very free}.
			\end{df}
			
			L'idea è che, più è grande $r$, più il fascio è globalmente generato e quindi aumenta la
			positività del fascio, quindi in altri termini aumentano le sezioni globali e questo dà molta
			più libertà di deformare la curva razionale, persino fissando dei punti sulla curva!
			
			
			\begin{prop}
				Sia $X$ varietà proiettiva liscia su $\C$. Allora $X$ è unirigata se e solo se
				ammette una curva razionale free.
			\end{prop}
			
			\begin{ex}
				Se $X$ ha $K_{X}$ nef, allora non esistono curve razionali free: infatti,
				se per assurdo esistesse $f:\PP^{1} \to X$ free, allora
				\begin{equation*}
					f^{*} K_{X} = -\det(f^{*}\Tt_{X}) = \Oo(-a_{1} -a_{2} - \dots -a_{n})\,.
				\end{equation*}
				Per aggiunzione allora possiamo calcolare
				\begin{equation*}
					K_{X} \cdot f_{*}C = f^{*}K_{X} \cdot C = -a_{1}-\dots-a_{n}\,;
				\end{equation*}
				sapendo che $f^{*}\Tt_{X}$ è globalmente generato, allora ogni $a_{i} \ge 0$,
				e inoltre $a_{1} \ge 2$ per l'osservazione precedente, quindi
				$K_{X} \cdot f_{*}C < 0$, contraddicendo l'ipotesi di nefness.
			\end{ex}
			
			\begin{ex}
				Ogni curva razionale di $\PP^{n}$ è very free. Infatti, 
				$\det (\Tt_{\PP^{n}}) = \Oo(n+1)$ è ampio
				e, dato che una curva razionale è data da un morfismo finito $f : \PP^{1} \to \PP^{n}$,
				allora $\det(f^{*}\Tt_{\PP^{n}})$ è ancora ampio, da cui si può dedurre che
				\begin{equation*}
					f^{*} \Tt_{\PP^{n}} = \Oo(a_{1}) \oplus \dots \oplus \Oo(a_{n})\,,
					\quad \text{con ogni } a_{i}>0\,,
				\end{equation*}
				da cui deduciamo che $f^{*}\Tt_{\PP^{n}} \otimes \Oo(-1)$ è globalmente generato.
				\footnote{Per i dettagli sul collegamento tra line bundles e positività,
				si guardi il Lazarsfeld, ``\emph{costruzione di Lutowski}''.}
			\end{ex}
			
			\begin{df}
				Sia $X$ varietza proiettiva su $\C$. Diciamo che $X$ è \textbf{razionalmente connessa}
				se, per ogni coppia di punti $x_{1},x_{2} \in X$, esiste una curva razionale per $x_{1}$ 
				e $x_{2}$.
			\end{df}
			
			\begin{thm}
				Una varietà $X$ è razionalmente connessa 
				se e solo se $X$ contiene una curva razionale very free.
			\end{thm}
			
			\begin{rmk}
				L'idea della dimostrazione, come quella per l'unirazionalità,
				è quella di considerare l'insieme delle curve razionali che passano per due punti
				e mostrare che questo insieme contiene un aperto non vuoto.
				In breve, basta mostrare il risultato per una curva generica in un aperto denso;
				questo mostra anche che essere razionalmente connessi è un \emph{invariante birazionale}.
			\end{rmk}			
			
			\begin{thm}
				Se $X$ è una varietà razionale, allora è anche razionalmente connessa.
			\end{thm}
			
			\begin{prop}\label{prop:RC}
				Sia $X$ varietà proiettiva liscia su $\C$. Se $X$ è razionalmente connessa, allora:
				\begin{enumerate}[label=\roman*)]
					\item per ogni $m,p > 0$ interi, allora $H^{0}\left(X, (\Omega_{X}^{p})^{\otimes m}\right) = 0$;
					\item $X$ non ha rivestimenti connessi étale finiti non banali;
					\item $X$ è semplicemente connessa.
				\end{enumerate}
				\begin{proof}
					Sia $X$ razionalmente connessa.
					\begin{enumerate}[label=\roman*)]
					\item Per ipotesi, esiste $f : \PP^{1} \to X$ curva very free per un punto generale di $X$. Allora, $f^{*}\Omega_{X}^{p} = (f^{*}\Tt_{X})^{\vee}$ è somme diretta di fibrati lineari di grado negativo: segue che $H^{0}(\PP^{1}, f^{*}\Omega_{X}^{p}) = 0$. Se $\omega \in H^{0}(X, \Omega^{p}_{X})$,
					allora  $f^{*}\omega \in H^{0}(\PP^{1}, f^{*}\Omega^{p}_{X}) = 0$,
					quindi $f^{*}\omega=0$. Questo implica che la restrizione di $\omega$ 
					sulla curva $f(\PP^{1})$ è identicamente nulla; dato che $\omega$ si annulla su
					tutte le curve very free che coprono $X$, allora $\omega = 0$.
					
					\item Per dualità, sappiamo che $H^{0}(X, \Omega_{X}^{p}) = H^{p}(X,\Oo_{X})$,
					quindi per il punto $i)$ sappiamo che $\chi(\Oo_{X}) = h^{0}(\Oo_{X}) = 1$.
					Se $\pi : Y \to X$ è un rivestimento étale finito, allora si può vedere che
					le curve very free si sollevano, 
					e in particolare che $Y$ è a sua volta razionalmente connessa.
					Ma allora $\chi(\Oo_{Y}) = 1$, da cui deduciamo che
					\begin{equation*}
						1 = \chi(\Oo_{Y}) = \deg \pi \cdot \chi(\Oo_{X}) = \deg \pi\,,
					\end{equation*}
					ma allora $\pi$ è un isomorfismo.
					
					\item Omessa.
 					\end{enumerate}
				\end{proof}
			\end{prop}
			
			
			\begin{conj*}
				Se $H^{0}\left(X, (\Omega_{X})^{\otimes m}\right) = 0$ per ogni $m > 0$,
				allora $X$ è razionalmente connessa. Si sa che la congettura è vera
				per $\dim X \le 3$.
			\end{conj*}
			
			\begin{cor}
				Se $X$ è razionalmente connessa, allora l'equivalenza lineare 
				e l'equivalenza numerica coincidono.
				\begin{proof}
					Proviamo che, se $D \equiv D'$, allora $D \sim D'$ 
					(l'altra implicazione è sempre vera). 
					Dalla scorsa lezione sappiamo che $c_{1}(D-D')$ è di torsione 
					in $H^{2}(X; \Z)$, ma se $X$ è razionalmente connessa,
					la parte $ii)$ della \textbf{\Cref{prop:RC}} sappiamo che 
					il secondo gruppo di coomologia è senza torsione, quindi
					$D-D'\in \ker c_{1}$. Per la parte $i)$ della \textbf{\Cref{prop:RC}}
					sappiamo che $c_{1}$ è iniettiva, da cui $D \sim D'$.
				\end{proof}
			\end{cor}
			
			\begin{thm}[\textbf{Kóllar, Miyaka, Mori}]\label{thm:Fano-RC}
				Ogni varietà di Fano liscia è razionalmente connessa.
			\end{thm}
			
			\begin{cor}
				Se $X$ è di Fano, allora $D \sim D$ se e solo se $D \equiv D'$.
			\end{cor}
				
				
		\section{Cono relativo di curve}
		
			Sia $N_{1}(X) = Z_{1}(X)/\equiv$ lo spazio degli $1$-cicli su $X$,
			 modulo l'equivalenza numerica, e scriviamp $NE(X)$ per il cono convesso generato
			 dalle classi di curve effettive. La sua chiusura è il \textbf{cono di Mori}
			 \begin{equation}
			 	\overline{NE}(X) = \overline{\Set{\sum a_{i} [C_{i}] \,|\, a_{i} \ge 0, \, C_{i} \subset X \text{ curva irriducibile } }} \subset N_{1}(X)\,.
			 \end{equation}
			 Sia $f : X \to Y$ un morfismo tra varietà proiettive normali.
			 Questa induce un omomorfismo $f_{*} : Z_{1}(X) \to Z_{1}(Y)$ nella seguente maniera:
			 data $C \subset X$ una curva irriducibile, se $C$ viene contratta da $f$, 
			 i.e. $f(C) = \{ pt. \}$, allora $f_{*}(C) = 0$; se $f_{*}(C) \subset Y$ è una curva,
			 allora $f_{*}C = m f(C)$, dove $m = \deg f\vert_{C} \in \Z$. Allora
			 il \textbf{pushforward} si ottiene estendendo $f_{*}$ per linearità su tutto $Z_{1}(X)$.
			 Inoltre, il pushforward passa al quoziente per l'equivalenza lineare, quindi
			 induce le mappe lineari
			 \begin{align*}
			 	f_{*} : N_{1}(X) \longrightarrow N_{1}(Y)\,, 
			 	\quad f^{*} : N^{1}(Y) \longleftrightarrow N^{1}(X)\,,
			 \end{align*}
			 dove il pullback sulle classi dei divisori di Cartier è dato da $f^{*}[D] := [f^{*}(D)]$.
			 Queste mappe sono legate dalla seguente formula:
			 
			 \begin{prop}[\textbf{Formula di proiezione}]\label{formula:proiezione}
			 	Sia $f: X \to Y$ un morfismo di varietà proiettive normali.
			 	Data $C \subset X$ curva e $D$ un divisore di Cartier su $Y$, allora vale
			 	\begin{equation}\label{PF}
			 		f^{*}D \cdot C = D \cdot f_{*}C\,.
			 	\end{equation}
			 \end{prop}
			 
			 \begin{oss}
			 	Se $f$ è un morfismo suriettivo, allora $f^{*}$ è iniettiva e $f_{*}$ è surgettiva.
			 	Infatti, per ogni curva $C \subset Y$, esiste una curva $C'\subset X$
			 	tale che $f(C') = C$ e quindi sappiamo che esiste $m>0$ tale che $f_{*}([C']) = m[C]$.
			 	Dalla surgettività di $f_{*}$ segue che $\dim \ker f_{*} = \rho_{X}-
			 	\rho_{X}$; dalla formula di proiezione si deduce che $\ker f^{*} \perp \im f_{*}$,
			 	ma avendo $\im f_{*} = N_{1}(Y)$, concludiamo che $\ker f^{*} = 0$.
			 \end{oss}
			 
			 \begin{df}
			 	Sia $f : X \to Y$ un morfismo tra varietà proiettive normali.
			 	Il \textbf{cono relativo di $f$} è il sottocono $NE(f)$ di $NE(X)$
			 	generato dalle classi di equivalenza numerica di curve contratte da $f$,
			 	cioè
			 	\begin{align*}
				 	NE(f) := \ker f_{*} \cap \overline{NE}(X)\,.
				\end{align*}			 	 
			 \end{df}
			 
			 \begin{oss}
			 	Dalla formula di proiezione, una curva irriducibile $C$ su $X$ è contratta da $f$ se
			 	e solo se $f_{*}[C] = 0$. Equivlentemente, se $A$ è un divisore ampio su $Y$,
			 	allora $C$ è contratta da $f$ se e solo se $f^{*}A \cdot C = 0$.
			 	La morale è che la proprietà di \textbf{essere contratta} è una
			 	proprietà \emph{numerica} delle curve.
			 \end{oss}	
			 
			 \begin{df}
			 	Un sottocono $F$ di un cono $\Cc$ è una \textbf{faccia estremale} se,
			 	per ogni $a, b \in \Cc$, se $a+b \in F$, allora sia $a$, sia $b$ appartengono a $F$.
			 	Una faccia è un \textbf{raggio estremale di $\Cc$} 
			 	se genera uno spazio vettoriale $1$-dimensionale.
			 \end{df}
			 
			 \begin{df}
			 	Un morfismo surgettivo $f:X \to Y$ tra varietà proiettive normali
			 	viene detto \textbf{contrazione} se le fibre di $f$ sono connesse.
			 \end{df}
			 
			 \begin{lemma}[\textbf{Lemma di rigidità}]\label{lemma:rigid}
			 	Se $f:X \to Y$ una contrazione tra varietà proiettive normali,
			 	allora $NE(f)$ è una faccia estremale di $\overline{NE}(X)$.
			 	Se $Y'$ è una varietà normale e $f':X \to Y'$ 
			 	è un morfismo tale che $NE(f) \subset NE(f')$,
			 	allora esiste un'unico morfismo $g:Y \to Y'$ che fa commutare il triangolo
			 	\begin{equation*}
			 		\begin{tikzcd}
			 			X \ar[rr,"f'"] \ar[dr,"f"'] && Y' \\
			 			& Y \ar[ur, "g"'] & \,.
			 		\end{tikzcd}
			 	\end{equation*}
			 	In altri termini, il cono relativo determina $f$ (a meno di isomorfismi $g$).
			 \end{lemma}
			 
			 \begin{ex}
			 	Vedremo un esempio sulla decomposizione di conic bundles più avanti.
			 \end{ex}
			 
			 \begin{ex}
			 	Il blow-up $f:\operatorname{Bl}_{p}\PP^{2} \to \PP^{2}$ è una contrazione
			 	che contrae il divisore eccezionale $E$. Dunque il cono relativo è 
			 	$NE(f) = \R_{\ge 0} [E]$. Sappiamo che $\rho_{\operatorname{Bl}_{p}\PP^{2}} = 2$,
			 	e infatti il gruppo di Picard è $\Pic(\operatorname{Bl}_{p}\PP^{2}) = \Z H \oplus \Z E$,
			 	dove $H := f^{*}\ell$, con $\ell$ retta in $\PP^{2}$.
			 	Detta $\widetilde{\ell} \subset \operatorname{Bl}_{p}\PP^{2}$ 
			 	la trasformata stretta di una retta in $\PP^{2}$ 
			 	passante per $p$, notiamo che l'anticanonico $-K_{\operatorname{Bl}_{p}\PP^{2}} = 3H-E$
			 	dà
			 	\begin{equation}
			 		-K_{X} \cdot \widetilde{\ell} > 0\,.
			 	\end{equation}
			 	La prossima volta vedremo il \textbf{Teorema delle contrazioni},
			 	che ci garantisce l'esistenza di una contrazione $\widetilde{f}$ che
			 	contrae le rette $\widetilde{\ell}$, dando così origine a
			 	\begin{equation*}
			 		\widetilde{f} : \operatorname{Bl}_{p}\PP^{2} \longrightarrow \PP^{1}\,,
			 	\end{equation*}
			 	il cui cono relativo è $NS(\widetilde{f}) = \R_{\ge 0}[\widetilde{\ell}]$.
			 \end{ex}
    	
     	
	\lecture[Divisori semi-ampi e teorema delle contrazioni associato a tali divisori. Studio delle superfici di Del Pezzo in base ai raggi estremali del loro cono di Mori.]{2025-12-05}
	
		\section{Divisori semi-ampi}
		
		In geometria birazionale si usano tanto perché permettono di ottenere delle \textbf{contrazioni}.
		Sia $D$ divisore di Cartier su $X$, varieta proiettiva.
		Se $H^{0}(X, \Oo_{X}(D)) \ne 0$ ha dimensione $N+1$, allora possiamo associargli una mappa razionale
		\begin{equation}
			\phi_{D} : X \dashrightarrow \PP(H^{0}(X, \Oo_{X}(D))) \simeq \PP^{N}\,,
			\quad \phi_{D}(x) := [s_{0}(x) : \dots : s_{N}(x)]\,,
		\end{equation}
		dove $s_{j}$ formano una base dello spazio delle sezioni globali di $D$.
		Dai primi corsi di geometria algebrica sappiamo che $\phi_{D}$ è un morfismo regolare
		se e solo se $\Oo_{X}(D)$ è globalmente generato, o equivalentemente se $\Oo_{X}(D)$
		non ha punti base, e scriveremo $\mathrm{Bs} \lvert D \rvert = \emptyset$.
		\begin{proof}[Idea]
			$D$  è ``\emph{positivo}'' se $\phi_{D}$ ha delle buone proprietà
			(ad esempio $D$ ampio, $D$ molto ampio...). 
			In geometria birazionale si lavora \emph{asintoticamente}, 
			i.e. si considera $mD$ , con $m \gge 0$.
			Le proprietà di positività tendono a aumentare al crescere di $m$.
		\end{proof}
		
		\begin{ex}
			Sia $D$ un divisore effettivo. Allora
			\begin{equation}
				\Oo_{X}(2D) \simeq \Oo_{X}(D) \otimes \Oo_{X}(D)
			\end{equation}
			e il prodotto delle sezioni induce un'immersione
			\begin{equation}
				\mathrm{Sym}^{2}H^{2}(X, \Oo_{X}(D)) \hookrightarrow H^{0}(X,\Oo_{X}(2D))\,.
			\end{equation}
			Osserviamo che $\mathrm{Bs} \lvert 2D \rvert \subset \mathrm{Bs} \lvert D \rvert$,
			poiché se $x_{0} \in \mathrm{Bs} \lvert 2D \rvert$, allora ogni $s \in H^{0}(X, \Oo_{X}(D))$
			dà
			\begin{equation}
				s^{2}(x_{0}) = s(x_{0})^{2} = 0 \quad \implies \quad  s(x_{0}) = 0 
				\quad \implies  \quad x_{0} \in \mathrm{Bs} \lvert D \rvert\,.
			\end{equation}
		\end{ex}
		
		\begin{df}
			Un divisore $D$ è \textbf{semi-ampio} se esiste 
			$m \in \N$ tale che $\mathrm{Bs} \lvert mD \rvert = \emptyset$.
		\end{df}
		
		\begin{thm}[\textbf{delle fibrazioni semi-ampie}]\label{thm:cntr-semi}
				Sia $X$ varietà proiettiva normale e $D$ divisore semi-ampio su $X$.
				Allora esiste una \textbf{contrazione} $\Phi:X \to Y$, con $Y$ varietà normale proietitva,
				tale che, per ogni $m \gge 0$ per cui $\mathrm{Bs} \lvert D \rvert = \emptyset$,
				si ha $Y_{m} := \overline{\phi_{mD}(X)} = Y$ e che $\phi_{mD} = \Phi$.
				Inoltre, esiste un divisore ampio $A$ su $Y$ tale che $\phi^{*}A = e_{0}D$,
				dove $e_{0}$ è il più piccolo esponente che divide ogni $m \in \N$ per cui
				$\mathrm{Bs} \lvert mD \rvert = \emptyset$.
		\end{thm}
		
		Sia $X$ proiettiva, e consideriamo il prodotto di intersezione $- \cdot -$ 
		 in \eqref{eq:intersezione}. Ogni divisore $D$ definisce in $N_{1}(X)$ l'iperpiano ortogonale
		  \begin{equation}
		  	D^{\perp} = \Set{ C \in N_{1}(X) \, | \, D \cdot C = 0} = \ker(D \cdot - )\,.
		  \end{equation}
		  
		 \begin{thm}[\textbf{Criterio di ampiezza di Kleiman}]\label{thm:kleiman}
		 	Un divisore di Cartier $D$ su $X$ è ampio se e solo se vale $D \cdot z > 0$,
		 	per ogni $z \in \overline{NE}(X) \setminus \{0\}$.
		 \end{thm}
		 
		 \begin{oss}
		 	Se $D$ è nef, ma non ampio, allora $D^{\perp}$ intersecca $\overline{NE}(X)$ nella
		 	sa frontiera; in realtà vale un criterio analogo di quello di Kleiman, 
		 	ma con la disuguaglianza ``$D \cdot z \ge 0$'', noto come \textbf{criterio di Nakai-Moishezon}.
		 \end{oss}
		
		\begin{thm}[\textbf{del cono di Mori}]\label{thm:cono-Mori}
			Sia $X$ una varietà liscia proiettiva. Esiste una famiglia numerabile di curve razionali
			$\Set{ C_{n} }_{n \in \N}$ tale che $0 < -K_{X} \cdot C_{n} \le \dim X +1$ e
			\begin{equation}
				\overline{NE}(X) = \Set{ z \in \overline{NE}(X) \, | \, K_{X} \cdot z \ge 0 }
				+ \sum_{n \in \N} \R^{+}[C_{n}]\,.
			\end{equation}
			Le semirette $\R^{+}[C_{n}]$ sono tutti i raggi estremali, $[C_{n}]$ sono
			generatori primitivi e vivono in 
			$\Set{ z \in N_{1}(X) \, | \, K_{X} \cdot z < 0}$. Questi raggi sono
			localmente discreti in questo semispazio e possono accumularsi \emph{solamente}
			vicino all'iperpiano $K_{X}^{\perp}$.
		\end{thm}
		
		
		\begin{cor}
			Esiste sempre una \textbf{contrazione} $X \to Z$ che contrae un raggio estremale.
		\end{cor}
		
		\subsection{Il caso delle superfici}
			L'obiettivo di questa sezione è quello di classificare le contrazioni di raggi estremali
			nel caso di $X$ una superficie e, in seguito, fornire esempi su $\overline{NE}(X)$
			nel caso $X$ varietà di Fano.
			
			Siano $C_{n}$ le curve razionali in $X$ fornite dal \textbf{Teorema del cono di Mori~\ref{thm:cono-Mori}} quindi in particolare
			\begin{equation}
				0 < - K_{X} \cdot C_{i} \le 3\,,
			\end{equation}
			e usando la formula del genere $2g(C_{n})-2 = (K_{X} + C_{n}) \cdot C_{n} = K_{X} \cdot K_{X}+ C^{2}_{n}$ scopriamo che
			\begin{equation}
				-1 \le C_{n}^{2} \le 1\,.
			\end{equation}
			
			\begin{itemize}
				\item[\textbf{Caso 1}:] Supponiamo che $C^{2}_{n} = 0$. Allora $X$ è una superficie
				rigata $X \to C$, con $C$ una curva proiettiva liscia e fibre $F \simeq \PP^{1}$,
				$\rho_{X} = 2$.
				\begin{proof}
					Sia $H$ ampio su $X$. Dal \textbf{Criterio di Kleiman~\ref{thm:kleiman}},
					sappiamo che $H \cdot C > 0$, allora se $m > \frac{K_{X} \cdot H}{C \cdot H}$
					segue che
					\begin{equation}
						(K_{X} - mC) \cdot H < 0\,.
					\end{equation}
					Per ampiezza di $H$, deduciamo che $(K_{X} - mC)$ non può
					essere effettivo, quindi per dualità di Serre
					\begin{equation}
						0 = H^{0}(X, (K_{X} - mC)) = H^{2}(X,mC)\,.
					\end{equation}
					La formula di Riemann-Roch ci dice che
					\begin{equation}
						\chi(X,C) = \frac{C^{2} -  C \cdot K_{X}}{2} + \chi(X, \Oo_{X})
					\end{equation}
					che quindi si riscrive come
					\begin{equation}
						h^{0}(mC) - h^{1}(mC) = m + \chi(X, \Oo_{X})\,.
					\end{equation}
					Quindi esiste un $m \in \N$ tale che $1 \ge h^{0}((m-1)C) < h^{0}(mC)$,
					da cui concludiamo $\dim \lvert mC \rvert \ge 1$.
					Consideriamo la successione esatta
					\begin{equation}
						0 \longrightarrow H^{0}(X, (m-1)C) \longrightarrow  H^{0}(X, mC) \overset{r}{\longrightarrow} H^{0}(C, mC)
					\end{equation}
					e notiamo che $\deg \Oo_{C}(mC) = mC^{2} = 0$, ma essendo $C$ razionale
					segue che $\Oo_{C}(mC) \simeq \Oo_{C}$. Ma allora $H^{0}(C, mC) = \C$,
					e la mappa di restrizione $r$ è surgettiva, 
					quindi $\lvert mC \rvert$ non ha punti base.
					Ma $C^{2}=0$ segue anche che non ci sono punti base in $X \setminus C$
					(altrimenti in questo punto base sarebbe contato dall'intersezione? Noccapito),
					e quindi concludiamo che $\Oo_{X}(mC)$ è globalmente generato.
					Per definizione, $C$ è semi-ampio e quindi per il \textbf{Teorema~\ref{thm:cntr-semi}}
					esiste una contrazione $\Phi : X \to Y_{m}$ per $m \gge 0$ tale che 
					$mC=\Phi^{*}A$ per un opportuno divisore ampio $A$ su $Y_{m}$
					e $\Phi^{*}A \cdot F=0$ se e solo se $F$ è contratta da $\Phi$.
					Poiché $C$ non è big, allora $Y_{m}$ non può essere una superficie,
					allora $Y_{m}$ è una curva proiettiva liscia e $\Phi$ contrae $C$, poiché
					\begin{equation}
						\Phi^{*}A \cdot C= mC \cdot C = 0\,.	
\end{equation}										
					La generica fibra $F$ di $\Phi$ è liscia e $F \equiv aC$, con $a>0$.
					So che $-K_{X} \cdot C = 2$, quindi $K_{X} \cdot F < 0$ e,
					essendo $F$ fibra con $F^{2}=0$, allora $K_{F} = (K_{X} + F)\vert_{F}$ è
					anti-ampio, da cui segue $F \simeq \PP^{1}$. Per concludere che \emph{tutte}
					le fibre sono $\PP^{1}$, 
					da $-K_{X} \cdot F = -K_{X} \cdot aC = 2 = -K_{X} \cdot C$,
					quindi $a=1$ e $F \equiv C$. Ma siccome $[C]$ genera un raggio estremale, in particolare non  è divisibile in $N_{1}(X)$, allora tutte le fibre sono ridotte e irriducibili.
				\end{proof}
				
				\item[\textbf{Caso 2}] Supponiamo che $C_{n} \simeq \PP^{1}$, generatore di un raggio
				estremale, abbia $C^{2}_{n} = 1$ Allora $\rho_{X}=1$ e $X \simeq \PP^{2}$.
				\begin{proof}
					Dato $H$ ampio su $X$, allora consideriamo l'aperto
					\begin{equation}
						U := \Set{ z \in N_{1}(X) \, | \, z^{2} > 0, \, H \cdot z > 0}\,.
					\end{equation}
					Notiamo che $U$ contiene gli $1$-cicli effettivi con quto-intersezione positiva,
					in particolare $[C_{n}] \in U$. Ma essendo $[C_{n}]$ estremale,
					può trovarsi nell'interno di $\overset{NE}(X)$ solo se $\overset{NE}(X) = \R_{+}[C_{n}]$, da cui $\rho_{X}=1$. Ma allora esiste $a \in \N$ tale che $-K_{X} \equiv aC$,
					e intersecando con $C$ abbiamo
					\begin{equation}
						3 = -K_{X} \cdot C = a C^{2} = a \quad \implies \quad a =3\,.
					\end{equation}
					Allora $-K_{X} \equiv 3C$ è ampio, di grado $(-K_{X})^{2} = 9$,
					ma la Del Pezzo di grado $9$ è $X \simeq \PP^{2}$.
				\end{proof}
				
				\item[\textbf{Caso 3}] Sia $C$ una curva razionale che genera un raggio estremale
				e $C^{2}=-1$. Per il \textbf{Teorema di Castelnuovo} sappiamo che
				$X \simeq \mathrm{Bl}_{p}X'$, per una qualche superficie liscia $X'$,
				e $C$ è il divisore eccezionale del blow-up.
				\begin{proof}
					L'idea è quella di prendere $H$ un divisore molto ampio e,
					a meno di sostituire $H$ con un multiplo,
					possiamo supporre che $h^{1}(H) = 0$.
					Sia $k = H \cdot C > 0$, e si noti che $D := H + kC$
					è tale che $D \cdot C=0$. Si dimostra che $\Oo_{X}(D)$ è semi-ampio
					ed è il divisore che dà la contrazione.
					Per i dettagli, scritti molto bene, si veda \parencite[V, Theorem~{5.12}]{AG}.
				\end{proof}
			\end{itemize}
		
		
		\begin{ex}
			Se $X$ è una varietà di Fano, allora dal \textbf{Teorema del cono~\ref{thm:cono-Mori}}
			si ha
			\begin{equation}
				\overline{NE}(X) = NE(X) = \sum_{n} \R_{+}[C_{n}]
			\end{equation}
			e, in particolare, siccome $-K_{X}$ è ampio, 
			non esiste la parte misteriosa $K_{X} \cdot - > 0$ del cono di Mori,
			su cui i raggi estremali \emph{potrebbero accumularsi};
			in altri termini, i raggi estremali sono una quantità finita.
			
			Supponiamo $\dim X = 2$, i.e. $X$ superficie di Del Pezzo.
			Da quanto osservato prima, se $C_{n}$ sono curve razionali che generano i raggi estremali,
			allora ho $3$ possibilità: se esiste $C_{n}$ tale che...
			\begin{enumerate}[label=\roman*)]
				\item abbia $C_{n}^{2}=1$, allora $X \simeq \PP^{2}$;
				\item abbia $C_{n}^{2}=0$,	allora $X$ è rigata, 
				quindi può essere $X \simeq \PP^{1} \times \PP^{1}$, 
				oppure è la proiezione $X \simeq \mathrm{Bl}_{p}\PP^{2} \to \PP^{1}$;
				\item abbia $C_{n}^{2}=-1$, allora $X \simeq \mathrm{Bl}_{p_{1}, \dots, p_{r}} \PP^{2}$,
				con $r \le 8$.
			\end{enumerate}
		\end{ex}


%    \appendix
    
%        \input{APPENDICE/}

\nocite{*}

\backmatter\KOMAoption{chapterprefix}{false}
\printbibliography[heading=bibintoc, title={Bibliografia}]{}

\end{document}