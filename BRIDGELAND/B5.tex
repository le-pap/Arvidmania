	
	\lecture[Applicazione della teoria di Bridgeland per lo studio del problema di Weak Brill-Noether.]{2025-10-04}
		
	\begin{warn*}
			Continuiamo a camminare sui cocci di vetro perché anche questa 
			lezione è stata tenuta da \textbf{Filippo Papallo} (cioè da me).
	\end{warn*}		
		
		Nella lezione di oggi vedremo insieme un'applicazione delle 
		condizioni di stabilità di Bridgeland
		allo studio di un problema classico, quello di \emph{weak Brill-Noether}.
		
		\section{Wall-crossing per stabilità di Bridgeland}
		
			Sia $S$ una superficie complessa proiettiva 
			e fissiamo $H$ un figrato lineare ampio su di essa.
			La volta scorsa abbiamo visto che esiste un embedding continuo
			\begin{equation*}
				\sigma : \operatorname{NS}(S)_{\R} \times Amp(S)_{\R} \longrightarrow Stab^{\dagger}(S)\,, \quad (B, \omega) \mapsto \sigma_{\omega,B}\,,
			\end{equation*}
			che, in generale, non sarà surgettivo. Ma questo non è davvero un problema
			per studiare gli spazi di moduli: infatti, 
			abbiamo accennato alla divisione di $Stab^{\dagger}(S)$
			in \emph{muri} e \emph{camere} e abbiamo accennato al fatto
			che la nozione di stabilità rimane costante in ciascuna camera, 
			analogamente a quanto accade
			alla classica stabilità con la suddivisione del \emph{cono ampio}.
			Quindi, possiamo accontentarci di studiare una fettina \textbf{bidimensionale}
			di $Stab^{\dagger}(S)$, che da Macrì viene chiamata \textbf{piano $(\alpha,\beta)$},
			ma io chiamerò \textbf{piano $(s,t)$}:
				\begin{equation}
					\sigma : \R \times (0,+\infty) \longrightarrow Stab^{\dagger}(S)\,,\quad
					\sigma_{(s,t)} := \sigma_{tH, sH}\,.
				\end{equation}
				
				\begin{warn*}
					Nelle loro note, \textbf{Macr\`i} e \textbf{Schmidt} scrivono che questa $\sigma$
					è un embedding, ma in realtà non è vero che ogni coppia $(s,t)$ definisce
					una condizione di stabilità: questo è espresso chiaramente da Bridgeland
					nel suo articolo \cite{B-stab-k3} sulle K3, e in \cite{CNY-K3} vengono
					descritte due regioni $U_{+}$ e $U_{-}$ 
					dove si è certi di avere condizioni di stabilità.
				\end{warn*}
			
				Preso $E \in \Db{S}$, per ogni $\cat{w} \in K_{num}(S)$ i \textbf{muri numerici} 
				per $E$	sono definiti dalle equazioni
				\begin{equation}\label{formula:num-wall}
					\Im Z_{t H, s H}(\cat{v}(E)) \Re Z_{tH, sH}(\cat{w})
					= \Im  Z_{t H, sH}(\cat{w}) \Re Z_{tH, sH}(\cat{v}(E))\,,
				\end{equation}
				che, scritte esplicitamente, descrivono una quadrica in $s$ e $t$ facile da
				studiare. A \textbf{Maciocia} è dovuta la caratterizzazione completa
				della struttura dei muri per qualsiasi superficie $S$.
				
				\begin{thm}[\textbf{Struttura dei Muri per Superfici}]
					Sia $\cat{v} \in K_{num}(S)$ fissato.
					\begin{enumerate}[label=\roman*)]
						\item Tutti i muri numerici per $\cat{v}$ sono semicerchi
						con centro sull'asse $s$, oppure sono semirette verticali;
						\item muri distinti non si intersecano;
						\item se $\ch_{0}(\cat{v}) \ne 0$, allora il muro
						verticale è unico ed è definito dall'equazione
						\begin{equation}
							s = \frac{H \cdot \ch_{1}(\cat{v})}{H^{2}\ch_{0}(\cat{v})}\,, \quad t > 0\,;
						\end{equation}
						In questo caso, tutti i muri semicircolari sono inscatolati, e giacciono da un solo lato del muro.
						\item se $\ch_{0}(\cat{v}) = 0$, allora ci sono solo muri circolari
						inscatolati.
						\item se un muro numerico ha un punto in cui è un muro effettivo
						(i.e. esiste un oggetto $E \in \Aa_{\sigma}$ che diventa strettamente semistabile), allora il muro numerico è un muro effettivo.
					\end{enumerate}
				\end{thm}
				
			Vale la pena sottolineare che non tutti i muri numerici ``dividono'' condizioni
			di stabilità diverse: possono esistere muri \emph{superflui},
			spesso chiamati \emph{fake walls}, e muri che, una volta attraversati,
			trasformano gli spazi di moduli secondo fenomeni detti di \emph{wall-crossing}.
			Stupefacentemente, in alcuni casi è possibile classificare esplicitamente la natura
			di tutti i muri e le trasformazioni birazionali ad essi legate.
			
			\begin{df}
				Fissato $\cat{v} \in K_{num}(S)$, sia $\Ww \subset Stab^{\dagger}(S)$
				un muro per $\cat{v}$. Indichiamo con $\sigma_{0} \in \Ww$ una condizione sul muro,
				mentre con $\sigma_{+}$ e $\sigma_{-}$ due condizioni che stanno in camere distinte,
				separate da $\Ww$. Diremo che $\Ww$ è:
				\begin{itemize}
					\item un \textbf{fake wall} se in 
					$M_{\sigma_{+}}(\cat{v})$ e $M_{\sigma_{-}}(\cat{v})$ non esistono curve
					di oggetti S-equivalenti tra loro rispetto alla stabilità indotta 
					da $\sigma_{0}$;
					\item un \textbf{muro totalmente semistabile} 
					se $M_{\sigma_{0}}(\cat{v}) = \emptyset$;
					\item un \textbf{flopping wall} se esiste una \emph{flopping contration}
					$M_{\sigma_{+}}(\cat{v}) \dashrightarrow M_{\sigma_{-}}(\cat{v})$;
					\item un \textbf{muro divisoriale} se esiste una contrazione divisoriale
					$\pi_{\pm} : M_{\sigma_{\pm}}(\cat{v}) \to \overline{M}_{\pm}$
					verso una varietà proiettiva irriducibile e normale.
				\end{itemize}
			\end{df}
			
			Non mi soffermo sul significato delle definizioni per due motivi: il primo è che non 
			conosco queste trasformazioni, note in geometria birazionale; il secondo è che
			saremo interessati solamente ai \textbf{muri totalmente semistabili}.
			
			\begin{ex}\label{K3-walls}
				Sia $S$ una superficie K3 proiettiva. 
				Esistono molti risultati di classificazione degli spazi $M_{S,H}(\cat{v})$
				in base al vettore di Mukai $\cat{v} = (r,c,a)$ fissato, e in particolare
				\textbf{Bayer} e \textbf{Macrì} (tra gli altri) hanno portato
				a conclusione lo studio degli spazi dei moduli di oggetti Bridgeland semistabili:
				
				\begin{prop}
					Sia $\cat{v} = m \cat{v}_{0}$ con $m > 0$ e $\cat{v}_{0}$ una classe primitiva.
					Supponiamo che $\sigma$ sia \textbf{$\cat{v}$-generica}, ovvero non giaccia
					su alcun muro numerico per $\cat{v}$. Se $\cat{v}_{0}^{2} \ge -2$,
					allora $M_{\sigma}(\cat{v}) \ne \emptyset$,  ha dimensione $\cat{v}^{2}+2$ 
					e $M_{\sigma}^{s}(\cat{v}) \ne \emptyset$ 
					(con l'eccezione $\cat{v}_{0}^{2} \le 0$ e $m > 1$). Inoltre,
					se $\cat{v}_{0}^{2} > 0$, allora $M_{\sigma}(\cat{v})$ è
					una varietà normale, proiettiva e irriducibile.
				\end{prop}
				
				Ora che sappiamo quando i moduli esistono, concentriamoci sui muri.
				Il \textbf{reticolo di Mukai} di $S$ è definito come
				\begin{equation*}
					H_{alg}^{*}(S; \Z) 
					:= H^{0}(S; \Z) \oplus \operatorname{NS}(S) \oplus H^{4}(S; \Z) \,,
				\end{equation*}
				e il \textbf{vettore di Mukai}
				\begin{equation*}
					\cat{v} : K_{num}(S) \longrightarrow H_{alg}^{*}(S; \Z)\,, \quad
					\cat{v}(E) := \ch(E) \smile \sqrt{\td(S)} 
					= (\rk(E), c_{1}(E), \rk(E) + \ch_{2}(E))
				\end{equation*}
				induce un'isometria, dove il prodotto scalare è dato dal \textbf{pairing di Mukai}
				\begin{equation*}
					\langle (r,c,a), (r_{1},c_{1},a_{1}) \rangle := c \cdot c_{1}- ra_{1} - r_{1}a\,,
				\end{equation*}
				che dunque per $E,F \in \Coh(S)$ coincide con il prodotto di Eulero
				\begin{equation*}
					\langle \cat{v}(E), \cat{v}(F) \rangle 
					= - \chi(E,F) = \sum_{p=0}^{2} (-1)^{p+1} \operatorname{ext}^{p}(E,F)\,.
				\end{equation*}								
				Nell'$(s,t)$-plane, la formula esplicita \eqref{ZwB-explicit} della carica centrale
				diventa
				\begin{equation*}
				Z_{(s,t)}(\cat{v})  = \langle e^{sH + itH}, (r,c,a,) \rangle 
				= \left( s(c \cdot H) - a - r H^{2} \frac{s^{2} - t^{2}}{2} \right) + i t H \cdot (c - rsH)\,,
				\end{equation*}
				pertanto, se $\cat{v}_{1} \in H^{*}_{alg}(S; \Z)$ è un altro vettore,
				l'equazione \eqref{formula:num-wall} del muro per $\cat{v}$ definita da $\cat{v}_{1}$
				si può riscrivere come:
				\begin{equation}
					 \frac{s^{2}+t^{2}}{2} [r_{1}(c \cdot H) - r (c_{1} \cdot H)] 
					 - s(r_{1}a-ra_{1})
					 = \frac{1}{H^{2}} [a_{1}(c \cdot H) - a (c_{1} \cdot H)]\,.
				\end{equation}
				Possiamo semplificare l'espressione come 
				$\frac{\epsilon}{2}(s^{2}+t^{2}) - \gamma s = \frac{\delta}{H^{2}}$
				ponendo
				\begin{equation}\label{not:edg}
					\epsilon := r_{1}(c \cdot H) - r (c_{1} \cdot H) \,, \quad
					\delta := a_{1}(c \cdot H) - a (c_{1} \cdot H) \,, \quad
					\gamma := r_{1}a-ra_{1} \,;
				\end{equation}
				si noti che $\epsilon = 0$ significa che $\cat{v}$ e $\cat{v}_{1}$ hanno la
				stessa pendenza $\mu$ e, in tal caso, il muro identificato da $\cat{v}_{1}$
				è verticale. Altrimenti, abbiamo una semicirconferenza centrata sull'asse $s$.
				La seguente classificazione è dovuta a \textbf{Bayer} e \textbf{Macrì} \cite[\textbf{Theorem~{5.7}}]{BM-MMP}:
				\begin{thm}
					Supponiamo $\cat{v}^{2}>0$. Un muro numerico $\Ww \subset Stab^{\dagger}(S)$ dato dall'equazione \eqref{formula:num-wall} è totalmente semistabile se e solo se $\cat{w}$ soddisfa una delle due condizioni: 
					\begin{itemize}
						\item $\cat{w}^{2} = 0$ e $\langle \cat{v}, \cat{w} \rangle = 1$;
						\item $\cat{w}^{2} = -2$, i.e. è una classe sferica effettiva, 
						e $\langle \cat{v}, \cat{w} \rangle < 0$.
					\end{itemize}
				\end{thm}
			\end{ex}
			
			\begin{ex}\label{Enriques-walls}
				Sia $S$ una superficie di Enriques proiettiva. Ci concentreremo sul caso di una
				generica Enriques \textbf{non-nodale}, i.e. $S$ non continene $(-2)$-curve;
				in questo caso, il numero di Picard è $10$ e quindi il diamante di Hodge
				appare così:
				\begin{equation*}
					\begin{tikzcd}[row sep = tiny, column sep = tiny]
						& & 1 &  & \\
						& 0 & & 0  & \\
						0 & & 10 &  & 0 \\
						& 0 & & 0  & \\
						& & 1 &  & \,.
					\end{tikzcd}
				\end{equation*}
				Per una tale $S$, consideriamo il reticolo
				\begin{equation*}
					\Lambda = H^{*}_{alg}(S; \Z) := \Set{ \left(r,c,\frac{a}{2} \right) \, | \, 
					r, a \in \Z \text{ tali che } r \equiv_{2} a, \, c \in \operatorname{NS}(S)/\langle K_{X} \rangle }
					\subset H^{*}(S; \Q)\,.
				\end{equation*}
				Ricordando che $\td(S) = (1,0,\frac{1}{2})$, notiamo che è ben definito
				il vettore di Mukai
				\begin{equation*}
					\cat{v} : K_{num}(S) \longrightarrow H_{alg}^{*}(S; \Z)\,, \quad
					\cat{v}(E) :=  
					= (\rk(E), c_{1}(E), \frac{\rk(E)}{2} + \ch_{2}(E))\,,
				\end{equation*}
				compatibile con il pairing di Mukai
				\begin{equation*}
					\langle \left(r,c,\frac{a}{2}\right),\left(r_{1},c_{1},
					\frac{a_{1}}{2}\right) \rangle := c \cdot c_{1}- \frac{ra_{1}}{2} - \frac{r_{1}a}{2}\,,
				\end{equation*}
				quindi è tutto formalmente simile al caso delle K3 e, in effetti, è compatibile:
				detto $\pi : \widetilde{S} \to S$ il rivestimento universale $2 : 1$ dell'Enriques,
				esso induce un embedding di reticoli:
				\begin{equation*}
					\pi^{*} : H_{alg}^{*}(S; \Z) \hookrightarrow H_{alg}^{*}(\widetilde{S}; \Z)\,,
					\quad \text{ tale che } \langle \pi^{*} \cat{v}, \pi^{*} \cat{w} \rangle 
					= 2 \langle \cat{v}, \cat{w} \rangle\,.
				\end{equation*}
				Fissato $\cat{v} \in  H_{alg}^{*}(S; \Z)$, usando
				la notazione \eqref{not:edg}, l'equazione del muro individuato
				da $\cat{v}_{1} = (r_{1}, c_{1}, \frac{a_{1}}{2})$ è
					\begin{equation*}
							\epsilon(s^{2}+t^{2}) - \gamma s = \frac{\delta}{H^{2}}\,.
					\end{equation*}									
				Lo studio degli spazi di moduli $M_{\sigma}(\cat{v})$ per le Enriques
				è stato effettuato nel dettaglio da \textbf{Yoshioka}; non riporto qui
				i criteri di esistenza, ma basti sapere che per $\cat{v}^{2} \ge 4$ e $\sigma$
				condizione $\cat{v}$-generica, allora $M_{\sigma}(\cat{v})$ è non vuoto,
				irriducibile e normale. La classificazione completa dei muri in $Stab^{\dagger}(S)$
				è dovuta a \textbf{Nuer} e \textbf{Yoshioka} in \cite{NY-MMP}.
				\begin{thm}\label{thm:K3-tss-walls}
					Sia $S$ una generica Enriques non-nodale e $\cat{v} \in H^{*}_{alg}(S; \Z)$
					tale che $\cat{v}^{2} \ge 4$. Allora un muro numerico 
					$\Ww \subset Stab^{\dagger}(S)$ dato dall'equazione \eqref{formula:num-wall} 
					è totalmente semistabile se e solo se 
					$\cat{w}=(r_{1},c_{1},\frac{a_{1}}{2})$ soddisfa una delle seguenti
					condizioni: 
					\begin{itemize}
						\item $\cat{w}^{2} = -1$ e $\langle \cat{v}, \cat{w} \rangle < 0$;
						\item $\cat{w}^{2} = 0$, $\ell(\cat{w}):= \gcd(r,c,a) = 2$ 
						e $\langle \cat{v}, \cat{w} \rangle = 1$.
					\end{itemize}
				\end{thm}
			\end{ex}
			
			
			
	
		
		\section{La teoria di Brill-Noether}
		
			Come ogni problema in geometria, il primo passo per formularlo è per le curve.
			Sia $C$ una curva complessa di genere $g$. Con Arvid abbiamo parlato
			dell'esistenza degli spazi di moduli $M_{C}(r,d)$ di fibrati vettoriali
			di rango $r$ e grado $d$, e in particolare nel \textbf{\Cref{thm:moduli-curve}}
			abbiamo annunciato alcune sue proprietà. Il passo successivo è
			quello di capirne meglio la geometria e alcuni sottoluoghi speciali,
			andando a studire, ad esempio, il comportamento coomologico del generico
			fibrato in $M_{C}(r,d)$. 
			
			Nel caso $r=1$, sappiamo che i fibrati lineari
			di grado $d$ sono parametrizzati da $\Pic^{d}(C)$, che è una varietà abeliana,
			e il \textbf{Teorema di Riemann-Roch} mostra che $\chi(\Ll) = d - g +1$,
			quindi la coomologia del generico $\Ll \in \Pic^{d}(C)$ è essenzialmente determinato
			dal grado:
			\begin{prop}\label{prop:WBN-curve}
				Sia $C$ curva di genere $g$ e $\Ll \in \Pic^{d}(C)$ generico, allora
				\begin{equation*}
					h^{0}(C,\Ll) = \max\{0,d-g+1\}\,, \quad h^{1}(C, \Ll) = \max\{0, g-d-1\}\,.
				\end{equation*}
				In particolare, un generico $\Ll$ con $\chi(\Ll) \ge 0$ ha $h^{1}(\Ll) = 0$.
			\end{prop}
			
			\begin{df}\label{df:WBN}
				Un fascio coerente $\Ff$ su una varietà proiettiva (complessa) $X$ soddisfa
				la condizione di \textbf{weak Brill-Noether} (in breve \textbf{WBN}) 
				se $\Ff$ ha \emph{al più} un gruppo di
				coomologia non nullo. Chiamiamo $\Ff$ \textbf{non-speciale} se
					$H^{i}(X, \Ff) = 0$ per ogni  $i > 0$,
				e \textbf{speciale} altrimenti.
			\end{df}
			
			La classica Teoria di Brill-Noether si concentra sullo studio dei luoghi
			in $\Pic^{d}(C)$ di fibrati \emph{speciali}, 
			quindi con un comportamento omologico inaspettato: solitamente, a $h > 0$ fissato,
			si studiano i cosiddetti
			\begin{align*}
				W^{h}_{d} := \Set{ \Ll \in \Pic^{d}(C) \,|\, h^{0} \ge h + 1}\,.
			\end{align*}
			Per capirne la natura, è sufficiente il \textbf{numero di Brill-Noether}
			\begin{equation}
				\rho(g,h,d) := g - (r+1)(g-d+r)\,.
			\end{equation}
			
			\begin{thm}
				Se $C$ è una \emph{generica} curva di genere $g$, allora
				\begin{enumerate}[label=\roman*)]
					\item $W^{h}_{d} \ne 0$ se e solo se $\rho(g,h,d) \ge 0$;
					\item Se $\rho(g,h,d) \ge 0$, allora $W^{h}_{d}$ è normale, Cohen-Macaulay
					di dimensione $\rho(g,h,d)$, liscio al di fuori di $W^{h+1}_{d}$;
					inoltre, se $\rho(g,h,d) > 0$, lo spazio $W^{h}_{d}$ è irriducibile.
				\end{enumerate}
			\end{thm}
			
			Ci sono anche generalizzazioni 
			per fibrati vettoriali \textbf{stabili} di rango $r \ge 2$ sulle curve, 
			e per studiarli confronta la pendenza $\mu$ con le quantità descritte sopra.
			Non entreremo nei dettagli, ma ci concentreremo sul problema per le superfici.
			Se $S$ è una superficie complessa, emergono le seguenti difficoltà:
			\begin{itemize}
				\item la coomologia di un fibrato $E$ su $S$ è concentrato in ben tre gradi, i.e.
				$H^{p}(S, E) = 0$ per $p<0$ e $p>2$, mentre su una curva avevamo solo due possibili
				coomologie non nulle; in particolare, \textbf{\Cref{prop:WBN-curve}}
				mostra che tutte le curve soddisfano WBN, ma su $S$ possono esserci più coomologie non nulle, nonostante qualche vanishing (e.g. se $S$ è una K3, $h^{1}(\Oo_{X}) = 0$ ma $h^{0}(\Oo_{X}) = h^{2}(\Oo_{X}) = 1$);
				\item a differenza del caso delle curve, 
				lo stack dei fasci coerenti con invarianti fissati su $S$ è quasi sempre riducibile;
				per ottenere proprietà migliori è necessario considerare i \textbf{fasci stabili}
				rispetto a un fibrato ampio $H$. Nonostante tutto, esistono casi in cui
				$M_{S,H}(\cat{v})$ sia ridotto o disconnesso, e quindi non ha senso parlare
				di un `\emph{elemento generico}'.
			\end{itemize}
			
			\begin{problem*}[\textbf{Weak Brill-Noether}]
				Data una componente irriducibile di $M_{S,H}(\cat{v})$, calcolare la 
				coomologia dell'elemento generico $E$ in tale componente.
			\end{problem*}
			
			Esistono già risultati dettagliati nel caso di superfici razionali
			e nel caso di alcune superfici $K$-triviali, come le superfici K3, 
			abeliane e biellittiche; un resoconto dettagliato sull'argomento è \cite{CHN-WBN-survey}.
			Oggi vedremo come approcciare il problema per le superfici K3,
			usando la strategia di \textbf{Coskun, Nuer} e \textbf{Yoshioka};
			la stessa strategia può essere usata per altre superfici $K$-triviali,
			come ad esempio le Enriques.
			
	\section{Il problema di Weak Brill-Noether via Bridgeland}

		Le superfici $K$-triviali $S$ hanno una ricca struttura derivata e
		il \textbf{Teorema di ricostruzione di Bondal-Orlov} fallisce per queste varietà.
		In particolare, esistono molte autoequivalenze di Fourier-Mukai
		\begin{equation*}
			\Phi^{\Ee} : \Db{S} \longrightarrow \Db{S}\,, \quad 
			\Phi^{\Ee}(E) := (\operatorname{pr}_{2})_{*}(\operatorname{pr}_{1}^{*}(E) \otimes \Ee)\,,
		\end{equation*}
		con $\Ee \in \Db{S \times S}$, che permettono (sotto opportune ipotesi) 
		di passare tra spazi di moduli diversi. In particolare, utilizziamo una particolare
		trasformata di Fourier-Mukai per cambiare punto di vista al problema di WBN.
		
		Sia $S$ una superficie K3 proiettiva con $\Pic(S) \simeq \Z H$, con $H$ ampio.
		Se $E$ è un generico fascio stabile con vettore $\cat{v}(E) = (r,dH,a)$, dove $d > 0$,
		allora sappiamo già che 
		\begin{equation*}
			H^{2}(S, E) \simeq \Ext^{2}(\Oo_{S}, E) \simeq \Hom(E, \Oo_{S})^{\vee} = 0\,,
		\end{equation*}
		dove l'annullamento è dovuto al fatto che $\mu(E) > 0 = \mu(\Oo_{S})$.
		Quindi il problema di WBN si riduce al calcolo dell'$h^{1}(E)$.
		
		\begin{lemma}\label{lemma:FM-WBN}
			Sia $\Delta \subset S \times S$ la diagonale e indichiamo con $\Ii_{\Delta}$
			l'ideale della diagonale. Sia $E \in \Coh(S)$ senza torsione in dimensione zero.
			Se $\Phi^{\Ii_{\Delta}}(E)^{\vee}$ è un fascio (in grado zero),
			allora $E$ soddisfa WBN non-speciale (e inoltre è genericamente globalmente generato).
			\begin{proof}
				Dalla successione esatta corta di fasci
				\begin{equation*}
					0 \longrightarrow \Ii_{\Delta} \longrightarrow \Oo_{S \times S}
					\longrightarrow \Oo_{\Delta} \longrightarrow 0\,,
				\end{equation*}
				si ottiene un triangolo esatto di trasformate di Fourier-Mukai applicate a $E$,
				la cui successione esatta lunga in coomologia è
				\begin{equation*}
        \begin{tikzcd}[row sep=small]
            0 \ar[r] 
            & \Hh^{0}(\Phi^{\Ii_{\Delta}}(E)) \ar[r]
            & H^{0}(S,E) \otimes \Oo_{S} \ar[r, "ev_{E}"]
            & E  \\
            {} \ar[r] 
            & \Hh^{1}(\Phi^{\Ii_{\Delta}}(E)) \ar[r, "\alpha_{1}"]
            & H^{1}(S,E) \otimes \Oo_{S}\ar[r]
            & 0   \\
            {} \ar[r] 
            & \Hh^{2}(\Phi^{\Ii_{\Delta}}(E)) \ar[r, "\alpha_{2}"]
            & H^{2}(S,E) \otimes \Oo_{S}) \ar[r]
            & 0\,. 
        \end{tikzcd}
    \end{equation*}
    		Riscrivendo le coomologie come
    		\begin{equation*}
    			\Hh^{p}(\Phi^{\Ii_{\Delta}}(E)) = 
    			\Hh^{p}(R\Hh om (\Phi^{\Ii_{\Delta}}(E)^{\vee}, \Oo_{S}))
    			= \Ee xt^{p}(\Phi^{\Ii_{\Delta}}(E)^{\vee}, \Oo_{S})\,,
    		\end{equation*}
    		l'ipotesi che $\Phi^{\Ii_{\Delta}}(E)^{\vee}$ sia un fascio
    		implica che il fascio $\Ee xt^{p}$ sia di torsione per ogni $p > 0$, 
    		da cui $\alpha_{1} = \alpha_{2} = 0$ poiché i codomini sono fasci
    		liberi da torsione; la tesi segue.
			\end{proof}
		\end{lemma}	
		
		Usando i criteri di Bondal e Orlov, si può anche dimostrare il seguente
		\begin{lemma}\label{lemma:autoequiv}
			Il funtore $\Phi : E \mapsto \Phi^{\Ii_{\Delta}}(E)^{\vee}$ induce
			un'autoequivalenza di $\Db{S}$.
		\end{lemma}
		
		Quindi, per studiare il problema di weak Brill-Noether per $E$, ci siamo 
		ricondotti a studiare quando la trasformata $\Phi^{\Ii_{\Delta}}(E)^{\vee}$ è un fascio
		(localmente libero). Per capirlo, useremo proprio le tecniche della teoria di Bridgeland!
		Infatti, all'interno di $Stab^{\dagger}(S)$ esiste una camera molto speciale:
		dato un qualsiasi punto $x \in S$, si verifica che 
		il fascio $\Ii_{\Delta}\vert_{x \times S}^{\vee}$
		è un oggetto $\sigma_{(s,t)}$-stabile, per ogni 
		$0 < s \lle 1$ e $t > 0$ in una determinata regione.
		Sia $\Ww$ il muro per $E$ determinato da 
		$\cat{w} = \cat{v}(\Ii_{\Delta}\vert_{x \times S}^{\vee}[1])=(-1,0,0)$,
		cioè la semicirconferenza
		\begin{equation*}
			\Ww_{0} : t^{2} + s \left( s - \frac{2a}{dH^{2}} \right) = 0\,.
		\end{equation*}
		Sia $\Cc_{0}$ la camera adiacente a $\Ww_{0}$ `\emph{da sopra}', 
		in cui vale la disuguaglianza
		$\mu_{Z_{(s,t)}}(\cat{w}) > \mu_{Z_{(s,t)}}(\cat{v}(E))$.
		
		\begin{prop}
			Per ogni $(s,t) \in \Cc_{0}$, la trasformata $\Phi$ del \textbf{\Cref{lemma:autoequiv}}
			induce un isomorfismo tra gli spazi di moduli
			\begin{equation*}
				\Phi : M_{\sigma_{(s,t)}}(r,dH,a) \overset{\sim}{\longrightarrow} M_{S,H}(a,dH,r)\,.
			\end{equation*}
		\end{prop}
		
		D'altra parte, a causa del \textbf{Large Volume Limit~\ref{LVL}} sappiamo che
		per $t \gge 1$ cadiamo nella camera di Gieseker $\Gg$, i.e. esiste un isomorfismo
		di spazi di moduli $M_{\sigma_{(s,t)}}(r,dH,a) \simeq  M_{S,H}(r,dH,a)$.
		Mettendo tutto insieme, ricapitoliamo quindi: se esiste un percorso
		in $Stab^{\dagger}(S)$ che parte da $\Gg$, arriva in $\Cc_{0}$
		e \emph{non attraversa nessun muro totalmente semistabile}, allora
		il generico elemento $E \in M_{S,H}(r,dH,a)$,
		che è $\sigma_{(s,t)}$-semistabile per $(s,t) \in \Cc_{0}$,
		viene trasformato da $\Phi$ in un \emph{fascio} $\mu$-semistabile 
		$\Phi^{\Ii_{\Delta}}(E)^{\vee} \in M_{S,H}(a,dH,r)$ e quindi
		deduciamo che $E$ non ha coomologie in gradi positivi
		per via del \textbf{\Cref{lemma:FM-WBN}}. 
		
		Ora, come fare a capire se esistono muri totalmente semistabili?
		Grazie alla classificazione presentata nell'\textbf{Esempio~\ref{K3-walls}},
		il problema si riduce a studiare delle disequazioni molto elementari:
		infatti, osservando che $\Ww_{0}$ passa dall'origine dell'$(s,t)$-plane,
		per capire se esistono dei muri più in alto di questo, è sufficiente studiare
		quando l'asse $s=0$ interseca muri di equazione \eqref{formula:num-wall}:
		quindi, una condizione necessaria che si aggiunge a quelle del
		\textbf{\Cref{thm:K3-tss-walls}} è che $\epsilon/\delta > 0$.
		In realtà, queste condizioni sono anche sufficienti, e riassumiamo tutto
		quanto detto nel seguente risultato:
		\begin{thm}
			Sia $S$ una generica K3 proiettiva, con $\Pic(S) \simeq \Z H$,
			e fissiamo un vettore di Mukai $\cat{v} = (r,dH,a)$, 
			con 
				\begin{equation*}
					r \ge 0\,, \quad  d > 0\,, \quad  r+a \ge 0 \quad  \text{e} \quad \cat{v}^{2} > 0\,.
				\end{equation*}							
			Se non esistono vettori $\cat{w} = (r_{1}, d_{1}H, a_{1}) \in H_{alg}^{*}(S; \Z)$
			tali che
			\begin{equation}\label{num-conditions}
				\cat{w}^{2} \in \Set{0,-1}\,,
				\quad \langle \cat{v}, \cat{w} \rangle < \cat{w}^{2} + 2\,,
				\quad d \ge d_{1} > 0 \quad \text{e} \quad
				\frac{ad_{1}-a_{1}d}{rd_{1}-r_{1}d} >0\,,
			\end{equation}
			allora $E$ soddisfa la proprietà di weak Brill-Noether (non-speciale).
		\end{thm}
		
		\begin{rmk}
			La condizione $d \ge d_{1} > 0$ in \eqref{num-conditions}
			deriva dalla definizione dei cuori tiltati: infatti, supponiamo
			che $\cat{w}$ sia il vettore di Mukai di un fascio $F$ destabilizzante
			tale che $F[1] \in \Ff_{(s,t)}$, i.e.
			 $\Im Z_{(s,t)}(\cat{w}) = (d_{1}-r_{1}s)tH^{2}\ge 0$.
		\end{rmk}		
		
		\begin{thm}[\textbf{Coskun-Nuer-Yoshioka}]
			Sia $S$ una generica K3 proiettiva, con $\Pic(S) \simeq \Z H$ e $H^{2} = 2h$. 
			Consideriamo un vettore di Mukai $\cat{v} = (r,dH,a)$, 
			con 
				\begin{equation*}
					r \ge 2\,, \quad  d > 0 \quad  \text{e} \quad \cat{v}^{2} \ge -2\,.
				\end{equation*}							
			Allora:
			\begin{enumerate}
				\item fissato $r \ge 2$, esiste un numero finito di tuple $(r,d,h,a) \in \Z^{4}$
				per cui il vettore $\cat{v} = (r,dH,a)$ \textbf{non} soddisfi la 
				proprietà di weak Brill-Noether;
				\item se $h \ge r$, allora $\cat{v}$ soddisfa weak Brill-Noether;
				\item se $a \le 1$, allora $\cat{v}$ soddisfa weak Brill-Noether.
			\end{enumerate}
		\end{thm}
		
		Questo criterio offre quindi un modo sistematico per capire quando un (generico) fascio
		non ha coomologie superiori solamente studiandone il vettore di Mukai!
		I teoremi enunciati sopra possono offrire anche
		un modo per andare a individuare possibili controesempi al problema di WBN;
		tuttavia, bisogna fare attenzione, poiché non è sempre detto che l'esistenza di un $\cat{w}$
		che soddisfi le condizioni \eqref{num-conditions} implichi che il generico
		elemento $E \in M_{S,H}(\cat{v})$ \emph{non} soddisfi WBN. In queste situazioni,
		è necessario raffinare i criteri e studiare più nel dettaglio cosa accade 
		a $E$ sul muro, ad esempio capendo la sua filtrazione di HN $E \in M_{\sigma_{0}}(\cat{v})$.
		
		
		
	