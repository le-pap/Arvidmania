
	\lecture[Inquadramento storico e motivazione che ha portato alla formulazione delle condizioni di stabilità su categorie triangolate da parte di T. Bridgeland.]{2025-09-30}
	
	\section{Motivazioni}
	
		Come sostiene sempre Arvid, per poter capire a fondo alcuni concetti matematici
		e il perché di alcune definizioni, è necessario capire il contesto storico in cui sono nate
		e le motivazioni che vi stanno dietro. Come molte teorie della matematica moderna,
		le ragioni dietro le condizioni di stabilità di Bridgeland vanno ricercate nella
		\emph{fisica}, dove già era importante il concetto di \textbf{stabilità di fibrati vettoriali}.
		
		Data $M$ una varietà $C^{\infty}$ e $E$ un fibrato vettoriale $C^{\infty}$ complesso su $M$,
		si può definire su esso una \textbf{connessione} $D$, cioè una mappa $C^{\infty}(M)$-lineare
		\begin{equation*}
			D : A^{0}(E) \longrightarrow A^{1}(E)
		\end{equation*}
		che può a sua volta essere estesa a una $D : A^{1}(E) \to A^{2}(E)$ grazie alla quale
		si definisce l'\textbf{opertatore di curvatura} $R_{D} := D \circ D$.
		
		\begin{fact*}
			Questo operatore può essere interpretato come $R_{D} \in A^{2}(End(E))$.
		\end{fact*}
		
		In geometria complessa siamo interessati a varietà \emph{più ricche}: se $X$ è una varietà
		olomorfa e $E$ un fibrato vettoriale $C^{\infty}$ complesso su $X$, 
		allora abbiamo la decomposizione di Hodge
		\begin{equation*}
			A^{1}(E) = A^{1,0}(E) \oplus A^{0,1}(E)\,,
		\end{equation*}
		da cui si deduce una decomposizione $D = D^{1,0} + D^{0,1}$, 
		per una qualsiasi connessione $D$ su $X$.
		
		\begin{df*}
			Una connessione $D$ si dice \textbf{compatibile} con la struttura complessa su $X$
			se $D^{0,1} = \overline{\partial}$.
		\end{df*}
		
		\begin{fact*}
			Nel caso $D$ sia compatibile , l'operatore di curvatura  $R_{D} \in A^{1,1}(End(E))$.
		\end{fact*}
		
		Quando studiamo un fibrato vettoriale complesso $E$, possiamo studiare delle
		\textbf{metriche hermitiane} $h$ su di esso; anche in questo caso è 
		si può studiare l'interazione di una connessione rispetto a una metrica:
		
		\begin{df*}
			Se $h$ è una matrica hermitiana su $E$, una connessione $D$ 
			si dice \textbf{compatibile con h} se, per ogni $a,b \in A^{0}(E)$ vale
			\begin{equation*}
			 	D(h(a,b)) = h(Da,b) + h(a,Db)\,.
			\end{equation*}
		\end{df*}
		
		\begin{thm*}
			Dato un fibrato vettoriale complesso $E$ su una varietà olomorfa $X$ e $h$ una metrica hermitiana su $E$, esiste un'unica connessione $D_{h}$, detta \textbf{connessione di Chern},
			compatibile sia con la struttura complessa di $X$, sia con $h$.
		\end{thm*}
		
		D'altra parte, possiamo considerare una metrica hermitiana $g$ sul tangente complessificato
		$T_{M} \otimes \C$ e con questa definire l'\textbf{operatore di Hodge} $\ast_{g}$
		e l'\textbf{operatore di Lefschetz} $L_{g}$. Grazie all'aggiunto
		\begin{equation*}
			\Lambda_{g} := \ast_{g}^{-1} \circ L_{g} \circ \ast_{g} : 
			A^{p,q}(E) \longrightarrow A^{p-1,q-1}
		\end{equation*}
		viene definita la sezione $\Lambda_{g}(R_{h}) \in A^{0}(End(E))$.
		
		\begin{df*}
			La metrica $h$ si dice \textbf{$g$-Hermite-Einstein} se esiste  $\lambda \in \R$
			tale che $$\Lambda_{g}(R_{h}) = \lambda \id_{E} \,.$$
		\end{df*}
		
		La costante $\lambda$ nella definizione viene talvolta chiamata \textbf{costante cosmologica}
		e ha un'importante interpretazione in fisica: infatti, le connessioni Hermite-Einstein
		corrispondono alle soluzioni delle \textbf{equazioni di Yang-Mills} che descrivono le
		particelle del modello standard. 
		Inoltre, queste connessioni hanno un inaspettato legame con la geometria algebrica:
		infatti, data $h$ una metrica hermitiana su $E$, 
		possiamo considerare la \textbf{prima classe di Chern} 
		$$c_{1}(E,h) \in H^{1,1}(X) \cap H^{2}(X;\Z)\,,$$
		che indicheremo solo con $c_{1}(E)$ poiché, nonostante la $2$-forma dipenda da $h$,
		la sua classe in coomologia è sempre la stessa. 
		Grazie ad essa, definiamo la \textbf{$g$-slope}
		del fibrato $E$ come la quantità
		\begin{equation*}
			\mu_{g}(E) := \frac{c_{1}(E) \cdot \omega_{X}^{n-1}}{\rk(E)} \in \R\,, 
			\quad \text{dove } n = \dim X\,.
		\end{equation*}
		
		\begin{fact*}
			Per $h$ metrica $g$-Hermite-Einstein si dimostra che $\mu_{g}(E)$ è essenzialmente
			la costante cosmologica $\lambda$ nella definizione, i.e. il rapporto $\mu_{g}/\lambda$
			è una costante che non dipende da $E$.
		\end{fact*}
		
		\begin{df*}
			Un fibrato vettoriale olomorfo $E$ su $X$ si dice \textbf{$\mu_{g}$-stabile} se,
			per ogni sottofascio coerente $0 \ne F \subsetneq E$ vale
				$\mu_{g}(F) < \mu_{g}(E)$. 
			Diremo che $E$ è \textbf{$\mu_{g}$-polistabile} 
			se $E \simeq E_{1} \oplus \dots \oplus E_{l}$, dove ogni $E_{i}$ è $\mu_{g}$-stabile
			e $\mu_{g}(E_{1}) = \dots = \mu_{g}(E_{l})$.
		\end{df*}
		
		\begin{thm*}[\textbf{Corrispondenza di Kobayashi-Hitchin}]
			Sia $X$ una varietà K\"ahler compatta, $E$ un fibrato vettoriale olomorfo su $X$ e $g$
			metrica K\"ahleriana su $E$. Allora
			\begin{equation*}
				E \text{ è } g\text{-Hermite-Einstein} \quad \iff \quad 
				E \text{ è } \mu_{g}\text{-polistabile}\,.
			\end{equation*}
		\end{thm*}
		Questo risultato è difficilissimo! 
		
		Altre motivazioni dietro la ricerca delle condizioni di stabilità vanno ricercate:
		 nella \textbf{teoria delle stringhe}: a seguito della \emph{homological mirror symmetry} 
		 cogetturata da Kontsevich in \cite{K-hms}, si è cominciato a pensare che le particelle
		 descritte dalle equazioni di Yang-Mills dovrebbero corrispondere a \emph{qualcosa di stabile}
		 nella categoria derivata dei \textbf{fasci coerenti di una varietà}: ecco di nuovo i fasci
		 stabili! In \cite{D-Dbranes}, Douglas formalizza 
		 il concetto di $\pi$-stabilità e sviluppa il collegamento con la mirror symmetry.
		 Tuttavia, i fisici si ritrovano confusi dal seguente problema: cosa significa
		 prendere un sottofascio di un complesso di fasci? 
		 La categoria derivata di una categoria abeliana non è quasi mai abeliana, 
		 quindi non ha senso parlare di sottoggetti...
		 
		 Nel 2002, Tom Bridgeland formalizza per bene in 
		 il concetto di \textbf{condizioni di stabilità} 
		 per una categoria triangolata in una preprint di arXiv (pubbicata poi come \cite{B}), 
		 e in \cite{B-stab-k3} dà una teoria fatta e finita
		 per superfici K3.
		 
		 \begin{quote}
		 	\emph{Imitiamo quello che succede per le curve, di cui conosciamo bene la teoria!}
		 \end{quote}
		 Tra gli strumenti principali ad aver avuto impatto in questa teoria
		 ricordiamo la \textbf{filtrazione di Harder-Narasimhan} e gli \textbf{spazi di moduli}.
		 
		 \begin{question*}[\textbf{di Arvid}]
		 	Nello studio della stabilità ``classica'' interviene lo studio delle metriche sui fibrati
		 	che crea un collegamento tra la stabilità in senso \emph{differenziale} e quello \emph{algebrico}.
		 	Passando alle condizioni di Bridgeland, questo pezzo viene a mancare?
		 	Esiste una qualche nozione di ``metrica hermitiana'' per complessi di fibrati
		 	che permetta di costruire un ponte tra il mondo differenziale e quello algebrico?
		 \end{question*}
		 
		 
		 
		 
		 \section{Stabilità su curve}
		 
		 Sia $C$ una curva proiettiva liscia su $\C$. Un problema classico in geometria
		 è la classificazione di tutti i fibrati vettoriali olomorfi $E \to \C$.
		 Nel caso $C = \PP{1}{}$ la classificazione è dovuta a \textbf{Grothendieck}:
		 \begin{thm}
		 	Dato $E$ è un fibrato olomorfo su $\PP{1}{}$, esiste un'unica sequenza di interi
		 	$a_{1} > a_{2} > \dots > a_{n}$ e di spazi vettoriali $V_{1}, \dots, V_{n}$ tali che
		 	\begin{equation}
		 		E \simeq (V_{1} \otimes \Oo(a_{1})) \oplus \dots \oplus (V_{n} \otimes \Oo(a_{n}))\,.
		 	\end{equation}
		 \end{thm}
		 Se la dimostrazione di questo fatto si riconduce, essenzialmente, 
		 a un problema di algebra lineare, la classificazione per $C$ di genere positivo è stata
		 risolta utilizzando tecniche di \emph{stabilità}.
		 
		 \begin{df}
		 	Sia $C$ una curva e fissiamo $H$ un fibrato lineare ampio su di essa.
		 	Dato $E$ un fibrato olomorfo su $C$, definiamo la sua \textbf{pendenza} 
		 	(dall'inglese ``\textbf{slope}'') come
		 	\begin{equation*}
		 		\mu(E) := \frac{\deg(E)}{\rk(E)}\,,
		 	\end{equation*}
		 	dove $\deg(E) := \deg(\det E)$ è il grado rispetto alla polarizzazione $H$.
		 \end{df}
		 
		 \begin{fact}
		 	Se $F$ è un fascio coerente su una curva $C$, allora in $\cat{Coh}(C)$ esiste una
		 	successione esatta corta canonica
		 	\begin{equation*}
		 		0 \to \Tt_{F} \to F \to \Ff_{F} \to 0\,,
		 	\end{equation*}
		 	dove $\Ff_{F}$ è un fascio localmente libero e $\Tt_{F}$ è un fascio di torsione.
		 	Grazie a questa decomposizione, possiamo definire
		 	\begin{equation*}
		 		\rk(F) := \rk(\Ff_{F})\,, \quad \deg(F) = \deg(\Ff_{F}) 
		 		+ \operatorname{length}(\Tt_{F})\,.
		 	\end{equation*}
		 \end{fact}
		 
		Grazie a questo fatto, possiamo quindi dare un senso alla pendenza $\mu(F)$ di qualsiasi
		fascio coerente usando la convenzione che, se $\rk(F)=0$, allora $\mu(F)=+\infty$.	
		Si noti in particolare che $\rk(F)=0$ implica $\deg(F) =  \operatorname{length}(\Tt_{F}) > 0$.	 
		 
		 \begin{df}
		 	Un fascio $F \in  \cat{Coh}(C)$ è detto \textbf{$\mu$-stabile} 
		 	(resp. \textbf{$\mu$-semistabile}) se, per ogni sottofascio $0 \ne F'\subsetneq F$ vale
		 	\begin{equation*}
		 		\mu(F') < \mu(F) \quad (\text{resp. } \mu(F') \le \mu(F))\,.
		 	\end{equation*}
		 \end{df}
		 
		 \begin{thm}[\textbf{Harder-Narasimhan}, 1975]\label{HN-filtration}
		 	Dato $F \in \cat{Coh}(C)$, esiste un'unica filtrazione
		 	\begin{equation*}
		 		0 = F_{0} \subsetneq F_{1} \subsetneq \dots \subsetneq F_{l-1} \subsetneq F_{l} = F
		 	\end{equation*}
		 	tale che, per ogni $0 < i \le l$, ogni fattore $E_{i}/E_{i-1}$ sia $\mu$-semistabile e 
		 	si abbia pendenza decrescence $\mu(F_{i}/F_{i-1}) > \mu(F_{i+1}/F_{i})$.
		 \end{thm}
		 
		 Alla luce di questo risultato, per classificare tutti i possibili fibrati su una curva
		 (ma in generale, su una varietà), è sufficiente conoscere tutti i possibili
		 fasci \textbf{$\mu$-semistabili}.
		 
		 \begin{ex}
		 	Se $C=\PP{1}{}$, un fibrato $E$ è semistabile se e solo se $E = V \otimes \Oo(a)$,
		 	per un qualche $a \in \Z$ e $V$ spazio vettoriale.
		 \end{ex}
		 
		 \begin{ex}
		 	Se $C$ è una curva ellittica, la classificazione dei fasci $\mu$-semistabili
		 	è stata risolta da Atiyah. Se $\gcd(\rk, \deg)=1$, allora valgono le equivalenze
		 	\begin{equation*}
		 		E \text{ è } \mu\text{-semistabile } \quad \iff \quad 
		 		E \text{ è } \mu\text{-stabile } \quad \iff \quad
		 		E \text{ è indecomponibile}\,,
		 	\end{equation*}
		 	e si sa che i fasci indecomponibili sono della forma $E \simeq L \otimes F$,
		 	dove $L \in \Pic^{0}(C)$ e $F$ si ottiene iterando un numero finito di volte
		 	estensioni di $\Oo_{C}$. Se $\gcd(\rk, \deg) > 1$, la classificazione è
		 	più complicata, ma esiste.
		 \end{ex}
		 
		 Per genere $g(C) \ge 2$, utilizziamo invece un altro tipo di tecnica:
		 dati $r \in \N$ e $d \in \Z$, definiamo 
		 \begin{equation*}
		 	\xM_{C}(r,d) : \cat{Sch}/\C \longrightarrow \cat{Set}\,,
		 \end{equation*}
		 il funtore che su uno schema $S$ vale
		 \begin{equation*}
		 	\xM_{C}(r,d)(S) := \left\{
		 	E \text{ fibrato su } C \times S  \, \middle| \,
		 	\begin{aligned}
		 		\text{per ogni }& s \in S\,,   \\
		 		E_{s} := E\vert_{\operatorname{pr}_{S}^{-1}(s)} &\text{ è } \mu\text{-semistabile }\\
		 		\text{ e } \rk(E_{s}) =& r\,, \, \deg(E_{s}) = d
		 	\end{aligned}
		 	\right\}\,.
		 \end{equation*}
		 
		 \begin{thm}\label{thm:moduli-curve}
		 	Il funtore $\xM_{C}(r,d)$ è \textbf{pro-rappresentabile} da uno schema
		 	$M_{C}(r,d)$, chiamato \textbf{spazio dei moduli} di fasci semistabili (di rango $r$ e grado $d$). Inoltre, $M_{C}(r,d)$ è una varietà proiettiva, integrale, normale e localmente fattoriale 
		 	di dimensione $r^{2}(g-1) - 1$ e il suo gruppo di Picard è
		 	\begin{equation*}
		 		\Pic(M_{C}(r,d)) \simeq \Pic(\Pic^{d}(C)) \oplus \Z\,.
		 	\end{equation*}
		 \end{thm}
		 
		 \begin{rmk*}
		 	Si noti che il grado $d$ non entra nella dimensione, 
		 	conseguenza del fatto che i jacobiani di ogni grado sono tutti isomorfi.
		 \end{rmk*}
		 
		 All'interno di $M_{C}(r,d)$ troviamo lo spazio $M_{C}^{s}(r,d)$
		 che parametrizza i fasci $\mu$-stabili.
		 
		 \begin{fact*}
		 	Se $\gcd(r,d) = 1$, allora lo spazio dei moduli è liscio e $M_{C}(r,d) = M_{C}^{s}(r,d)$.
		 	Se invece $\gcd(r,d) > 1$, allora $M_{C}(r,d) \sm M_{C}^{s}(r,d)$ è il luogo
		 	singolare, che vive in codimensione maggiore o uguale a $2$.
		 \end{fact*}
		 
		 Invece di considerare il funtore $\xM_{C}(r,d)$, è interessante fissare $\Ll \in \Pic(C)$
		 e studiare il funtore $\xM_{C}(r,\Ll)$ i cui fasci $E \in \xM_{C}(r,\Ll)(S)$
		 hanno, per ogni $s \in S$, il determinante fissato $\det(E_{s}) = \Ll$;
		 anche in questo caso lo spazio dei moduli $M_{C}(r,\Ll)$ esiste come schema proiettivo
		 e ha dimensione $(r^{2}-1)(g(C)-1)$, esiste un divisore
		 ampio $\theta$ tale che $\Pic(M_{C}(r,\Ll)) \simeq \Z \theta$ e 
		 $K_{M_{C}(r,\Ll)} \simeq -2u\theta$, con $u=\gcd(r,d)$.
		 
		 Nel caso in cui, al posto di una curva $C$ andiamo a considerare una varietà di dimensione
		 maggiore, la situazione diventa più complicata. Noi saremo maggiormente interessati
		 al caso in cui $X$ è una superficie complessa, compatta (probabilmente anche proiettiva).
		 Anche in dimensione più alta, la filtrazione di Harder-Narasimhan~\ref{HN-filtration}
		 continua a esistere, ma abbiamo un'altra filtrazione molto utile,
		 che ci permette di ridurre il problema della classificazione dei
		 fasci $\mu$-semistabili solamente a quelli $\mu$-stabili.
		 
		 \begin{thm}[\textbf{Filtrazione di Jordan-H\"older}]
		 	Sia $F$ un fascio semistabile su $X$. Esiste una filtrazione (non unica!)
		 	\begin{equation*}
		 		0 = F_{0} \subsetneq F_{1} \subsetneq \dots \subsetneq F_{l-1} \subsetneq F_{l} = F
		 	\end{equation*}
		 	tale che le sue componenti graduate $gr_{i}(F) := F_{i}/F_{i-1}$ siano
		 	$\mu$-stabili, con stessa pendenza $\mu(gr_{i}(E)) = \mu(E)$, 
		 	per ogni $0 < i \le l$. Inoltre, il fascio graduato $gr(F) := \oplus_{i=1}^{l} gr_{i}(F)$
		 	non dipende dalla filtrazione scelta.
		 \end{thm}
		 
		 Parallelamente, si possono studiare altre nozioni di stabilità, diverse dalla $\mu$-stabilità,
		 come ad esempio la \textbf{stabilità di Gieseker}, definita in \eqref{Gieseker-stable}: 
		 anche in questo caso otteniamo
		 filtrazioni analoghe a quella di Harder-Narasimhan e Jordan-H\"older.
		 
		 \begin{rmk}
		 	Nel caso $X=C$ sia una curva, le nozioni di $\mu$-stabilità e Gieseker stabilità coincidono, 
		 	infatti, dato $F \in \cat{Coh}(C)$ un fascio, 
		 	possiamo calcolare il polinomio di Hilbert di $F$ rispetto a un fibrato $H$ di classe
		 	$h \in H^{2}(C;\Z) \simeq \Z$ fissato grazie a \textbf{Hirzebrich-Riemann-Roch} come
		 	\begin{align*}
		 		P_{H}(F)(n) &= \int_{X} \ch(F \otimes \Oo_{X}(nH)) \smile \td(X) \\
		 		&= \int_{X} \ch(F) \smile \ch(\Oo_{X}(nH)) \smile \td(X) \\
		 		&= \int_{X} (\rk(F), \deg(F)) \smile (1,hn) \smile (1,1-g) \\
		 		&= \int_{X} (\rk(F), \rk(F) hn + \deg(F)) \smile (1,1-g) \\
		 		&= \int_{X} (\rk(F), \rk(F) hn + \deg(F) + (1-g)\rk(F))  \\
		 		&=  h\rk(F) n  + \deg(F) + (1-g)\rk(F)\,,
		 	\end{align*}
		 \end{rmk}
		 e quindi il polinomio di Hilbert ridotto è
		 \begin{equation*}
		 	p_{H}(F)(n) =  n  + \frac{\deg(F)}{h\rk(F)} + \frac{(1-g)}{h} 
		 	= n + \frac{\mu(F)}{h} + costante\,.
		 \end{equation*}
		 
		 In dimensione più alta, però, la classe di Todd $\td(X)$ diventa più complicata
		 e, nonostante la pendenza $\mu(F)$ compaia all'interno di $p_{H}(F)$,
		 non possiamo dedurre un'equivalenza come in questo caso, ma le disuguaglianze strette
		 \begin{equation*}
		 	\mu\text{-stabilità} \implies H\text{-stabilità} 
		 	\implies H\text{-semistabilità} \implies \mu\text{-semistabilità}\,.
		 \end{equation*}
		 A questo punto, per studiare la stabilità in dimensione alta, sono state seguite
		 più strade:
		 \begin{itemize}
		 	\item studiare i \textbf{fasci twistati}, sempre importanti per la fisica
		 	per le loro relazioni con i \emph{B-fields};
		 	\item studiare la \textbf{stabilità twistata} di fasci, i.e. presa una
		 	classe $B \in H^{1,1}(X) \cap H^{2}(X;\R)$, si rifanno i conti della pendenza
		 	con il carattere di Chern twistato
		 	\begin{equation*}
		 		\ch^{B}(F) := \ch(F) \smile e^{-B}\,;
		 	\end{equation*}
		 	\item studiare le \textbf{condizioni di stabilità di Bridgeland},
		 	che offrono un punto di vista completamente nuovo
		 	e un formalismo non ancora studiato, in quanto vengono considerati
		 	non più fasci, ma complessi $E^{\bullet} \in \Db{X}$.
		 \end{itemize}
		 
		 
		 
		 
		 
		 
		 
		 
		 
		 