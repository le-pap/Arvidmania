
	\lecture[Moduli di oggetti (semi)stabili come soluzioni di un problema GIT.]{2025-09-24}
		
		Abbiamo quindi visto che aggiungendo condizioni sul tipo di
		fasci che vogliamo parametrizzare, si ha più speranza nell'esistenza di buoni spazi di moduli.
		In particolare, una proprietà che funziona bene è la \textbf{stabilità},
		di cui ne esistono diverse nozioni, ad esempio:
		\begin{enumerate}
			\item \textbf{stabilità di Takemoto-Mumford} o \textbf{$\mu$-stabilità}:
			dato $F \in \Coh(X)$ fascio \textbf{torsion-free}, definiamo la sua \textbf{pendenza}
			rispetto a un fibrato lineare ampio $H$ come
			\begin{equation}
				\mu(F) := \mu_H(F) = \frac{c_{1}(F) \cdot H}{\rk(F)}\,,
			\end{equation}			 
			dove la prima classe di Chern $c_{1}(F)$ si definisce come $c_{1}(\det F\vert_{U})$,
			dove $U \subset X$ è un aperto di Zariski su cui $F$ è localmente libero.
			Un fascio $F$ si dice \textbf{$\mu$-stabile} 
			(resp. \textbf{$\mu$-semistabile}) se, per ogni sottofascio proprio 
			non banale $0 \neq F'\subsetneq F$ con $0 < \rk(F') < \rk(F)$, la pendenza cresce:
			\begin{equation*}
				\mu(F') < \mu(F) \quad (\text{resp. } \mu(F') \le \mu(F))\,.
			\end{equation*}
			Se nella definizione di $\xM_{P}(X,H)$ vi si aggiunge la condizione 
			``\emph{$F_{s}$ è $\mu_{H}$-stabile, per ogni $s \in S$}'', allora
			il funtore diventa \textbf{pro-rappresentabile}!
			
			\item \textbf{stabilità di Gieseker-Maruyama} o \textbf{$H$-stabilità}:
			per ogni fascio coerente $F \in \Coh(X)$ \textbf{puro}\footnote{Dato $G \in \Coh(X)$, si definisce la \textbf{dimensione di $G$}come la dimensione del supporto di $G$, cioè il sottoschema chiuso $Supp(E) = \Set{x \in X \,|\, G_{x} \ne 0}$. Un fascio $G$ si dice \textbf{puro} di dimensione $d$ se ogni sottofascio $0 \ne G' \subset G$ ha $\dim(G) = d$.}, di $\dim(F) = d$, 
			andiamo a considerarne il polinomio di Hilbert rispetto a un fibrato ampio $H$
			\begin{equation*}
				P_{H}(F) := \chi(F \otimes \Oo_{X}(nH)) = \alpha_{d} n^{d} + altri \, termini...
				\quad \text{per } n \gge 0\,,
			\end{equation*}
			e lo rendiamo monico, introducendo così il \textbf{polinomio di Hilbert ridotto}
			$p_{H}(F) := P_{H}(F)/\alpha_{d}$. Diremo che $F$ è \textbf{$H$-stabile} 
			(resp. \textbf{$H$-semistabile}) se, per ogni sottofascio $0 \neq F'\subsetneq F$
			si ha
			\begin{equation}\label{Gieseker-stable}
				p_{H}(F') < p_{H}(F) \quad (\text{resp. } p_{H}(F') \le p_{H}(F) )\,,
			\end{equation}
			dove l'ordine è da intendersi come ordinamento lessicografico. Anche
			in questo caso, gli oggetti $H$-stabili danno spazi di moduli belli e proiettivi!
		\end{enumerate}
		
		Ma come mai queste nozioni danno spazi di moduli?
		Grothendieck aveva scoperto che una condizione necessaria per la rappresentabilità
		è la \textbf{limitatezza}:
		\begin{df}
			Una famiglia di fasci $\Ff \subset \Coh(X)$ si dice \textbf{limitata}
			se esiste $S$ uno schema su $\C$ di tipo finito 
			e un fascio $\Oo_{S \times X}$-coerente $G$ tale che
			$\Ff = \Set{F\vert_{s \times X} \,|\, s \in U \text{ punto chiuso}}$,
			per un determinato \emph{sottoinsieme} $U \subset S$.
		\end{df}
		Sottolineiamo che $U$ nella definizione è solo un \emph{insieme}
		che a priori non ha nessuna struttura geometrica. 
		Questa proprietà è più facile da verificare rispetto alla rappresentabilità,
		grazie a criteri elaborati dallo stesso Grothendieck; uno di questi
		è legato incredibilmente alla \textbf{regolarità di Castelnuovo-Mumford}.
		
		\begin{prop}
			Dato $H$ un fibrato lineare ampio su $X$, se esiste $m \in \N$ tale che 
			ogni $F \in \Ff$ sia $m$-regolare rispetto a $H$, allora $\Ff$ è limitata.
		\end{prop}
		
		\begin{prop}
			Dato $H$ un fibrato lineare ampio su $X$ e $P \in \Q[x]$, la famiglia
			\begin{equation}
				\Ff = \Set{F \in \Coh(X) \, | \, F \text{ è } \ast\text{-semistabile e } P_{H}(F) = P}\,,
			\end{equation}
			con $\ast = H$ oppure $\ast = \mu$, è limitata.
		\end{prop}
		
		In particolare, questo implica che esiste un $m \in \N$ tale che, per ogni $F \in \Ff$,
		il fascio $F \otimes \Oo_{X}(mH)$ e globalmente generato, cioè la valutazione è surgettiva:
		\begin{equation*}
			H^{0}(X, F \otimes \Oo_{X}(nH)) \otimes \Oo_{X}(-mH) \twoheadrightarrow F\,.
		\end{equation*}
		Notiamo inoltre che la dimensione dell'$H^{0}$ non dipende da $F$, infatti per la 
		scelta di $m$ si ha
		\begin{align*}
			P_{H}(F)(m) = \chi(F \otimes \Oo_{X}(mH))
			= \sum_{i=0}^{\dim(F)} (-1)^{-1} h^{i}(F \otimes \Oo_{X}(mH)) = h^{0}(F \otimes \Oo_{X}(mH)\,,
		\end{align*}
		quindi $H^{0}(X, F \otimes \Oo_{X}(nH)) \otimes \Oo_{X}(-mH)$ è sempre lo stesso fascio,
		indipendentemente da $F$,
		che denoteremo $\Hh_{m}$. Il vantaggio, ora, è che possiamo vedere un $F \in \Ff$ qualsiasi
		come un quoziente $\Hh_{m} \twoheadrightarrow F$, ma quindi $F$ è un punto di uno schema Quot:
		\begin{equation*}
			[F] \in \cat{Quot}_{X}(\Hh_m, P)\,.
		\end{equation*}
		
		\begin{df}
			Sia $\xP$ una proprietà di varietà su $\C$. Diremo che $\xP$ è \textbf{aperta}
			se, per ogni famiglia piatta $f : \xX \to B$, l'insieme
			\begin{equation*}
				\Set{b \in B \,|\, \xX_{b} := f^{-1}(b) \text{ verifica } \xP}
			\end{equation*}
			è un aperto di $B$. Analogamente, una una proprietà $\xP$ di fasci coerenti su una varietà $X$ è detta \textbf{aperta} se, per ogni famiglia piatta $f : \xX \to B$ e $\Ff \in \Coh(\xX)$,
			 l'insieme
			\begin{equation*}
				\Set{b \in B \,|\, \Ff_{b} := \Ff\vert_{f^{-1}(b)} \text{ verifica } \xP}
			\end{equation*}
			è un aperto di $B$.
		\end{df}
		
		\begin{ex}
			Essere K\"ahleriana è una proprietà aperta di varietà; essere proiettiva
			non lo è: si pensi ad esempio allo spazio dei moduli delle K3, dove le K3 proiettive vivono in codimensione $1$.
		\end{ex}
		
		\begin{ex}
			Essere $\ast$-(semi)stabili, con $\ast=H,\mu$, è una condizione aperta per fasci!
		\end{ex}
		
		Segue da questo fatto che $R = \Set{[F] \in \cat{Quot}_{X}(\Hh_{m},P) \, | \, F \text{ è stabile}}$ è un aperto dello schema $\cat{Quot}$. Tuttavia, quest'ultimo \textbf{non} è lo spazio
		di moduli che stavamo cercando... in effetti, la notazione $[F]$ per un punto di $R$
		è un abuso di notazione: infatti, quello che tale schema parametrizza sono
		classi di equivalenza di
		 \textbf{morfismi surgettivi} $[\Hh_{m} \twoheadrightarrow F]$,
		 che in generale sono molti di più dei fasci stabili che vogliamo classificare.
		 Posto $V := H^{0}(X, \Oo_{X}(mH))$, 
		 si noti che $\GL(V)$ agisce per cambio di base su $V \otimes \Oo_{X}(-mH)$. 
		 L'azione $\GL(V) \curvearrowright V \otimes \Oo_{X}(-mH)$ 
		 si solleva a un'azione $\GL(V) \curvearrowright \Hh_{m}$;
		 in realtà, dato che gli scalari non cambiano la classe di isomorfismo del fibrato,
		 il gruppo che vi agisce è $\PGL(V)$ e si ottiene
		 \begin{align*}
		 	M_{H}^{ss}(X,P) &= R^{ss}/ \PGL(V) \quad \text{buon quoziente,}\\
		 	M_{H}^{s}(X,P) &= R^{s}// \PGL(V) \quad \text{quoziente geometrico!}
		 \end{align*}