	
	\lecture[.]{2025-10-09}
	

	Sia $X$ uno schema proiettivo connesso (su un campo algebricamente chiuso $k$, di caratteristica $0$).
	Fissiamo $H$ un fibrato lineare ampio e $P \in \Q[t]$ un polinomio.
	\begin{thm}
		Esiste lo spazio dei moduli dei fasci $H$-semistabili $M_{P}(X,H)$, 
		che è uno schema \textbf{proiettivo}.
	\end{thm}
	
	Questo risultato è generalissimo. È un po' debole perché sappiamo solo che lo spazio è
	uno schema, ma siamo felici perché sappiamo essere proiettivo! Questo fatto deriva dalla costruzione
	GIT. Inoltre il senso di `stabile' coincide, sia nel senso di Gieseker, sia nel senso GIT (condizione sulle orbite).
	
	Ma cerchiamo ora di capire cosa rappresentano i punti di questi schemi.
	
	\section{Filtrazioni}
	
		\subsection{Harder-Narasimhan}
		
		Per ogni fascio $F \in \Coh(X)$, dove $X$ è come sopra, esiste un'unica filtrazione
		 	\begin{equation*}
		 		0 = F_{0} \subsetneq F_{1} \subsetneq \dots \subsetneq F_{l-1} \subsetneq F_{l} = F
		 	\end{equation*}
		 	tale che, per ogni $0 < i \le l$, ogni fattore $E_{i}/E_{i-1}$ sia $\mu$-semistabile e 
		 	si abbia pendenza decrescence $\mu(F_{i}/F_{i-1}) > \mu(F_{i+1}/F_{i})$.
		 Lo stesso risultato vale se rimpiazziamo $\mu$-semistabili con $H$-semistabili
		 e la pendenza con il polinomio di Hilbert ridotto.
		 
		 Quindi per capire i fasci su uno schema proiettivo connesso $X$ è sufficiente
		 classificare i suoi fasci semistabili (in un qualsiasi senso).
		 Ma possiamo decomporre ulteriormente i fasci semistabili in fattori stabili:
		  \begin{thm}[\textbf{Filtrazione di Jordan-H\"older}]\label{JH-filtration}
		 	Dato $F \in \cat{Coh}(X)$ un fascio $\mu$-semistabile, esiste una filtrazione (non unica!)
		 	\begin{equation*}
		 		0 = F_{0} \subsetneq F_{1} \subsetneq \dots \subsetneq F_{l-1} \subsetneq F_{l} = F
		 	\end{equation*}
		 	tale che, per ogni $0 < i \le l$, ogni fattore $F_{i}/F_{i-1}$ sia $H$-stabile e 
		 	tutti i termini hanno la stessa pendenza e 
		 	$\mu(F) = \mu(F_{i}/F_{i-1}) = \mu(F_{i+1}/F_{i})$.
		 \end{thm}
		 
		 Quindi è sufficiente classificare solo i fasci stabili? Non proprio, perché abbiamo
		 appena detto che questa filtrazione non è unica... Quindi dobbiamo considerarla
		 a meno di una certa relazione di equivalenza.
		 Si dimostra che, se
		\begin{equation*}
		 		0 = F'_{0} \subsetneq F'_{1} \subsetneq \dots \subsetneq F'_{s-1} \subsetneq F'_{s} = F
		 	\end{equation*}
		 	è un'altra filtrazione di JH, allora $l=s$ e, per ogni $1 \le i \le l$, 
		 	esiste un unico $1 \le j \le l$ tale che
		 	\begin{equation}
		 		F_{i}/F_{i-1} \simeq F'_{j}/F'_{j-1}\,.
		 	\end{equation}
		 	Quindi a essere unico è il \textbf{graduato di Jordan-H\"older} associato a $F$!
		 	
		 	\begin{df}
		 		Due fasci $H$-semistabili $F$ e $F'$ sono \textbf{S-equivalenti} se e solo se
		 		$gr_{JH}(F) \simeq gr_{JH}(F')$.
		 	\end{df}
		 	Si deduce  che i punti chiusi di $M_{P}(X,H)$ parametrizzano le classi di S-equivalenza
		 	dei fasci $H$-semistabili di polinomio di Hilbert fissato $P$, mentre
		 	$M^{S}_{P}(X,H)$ rappresenta le classi di S-equivalenza di fasci stabili,
		 	quindi le classi di \textbf{isomorfismo} di fasci stabili di polinomio fissato $P$.
		 	Quindi possiamo concludere che $M^{S}_{P}(X,H)$ è uno schema che rappresenta
		 	il funtore dei fasci stabili? Purtroppo no...
		 	
		 	Cerchiamo di capire cosa impedisce la rappresentabilità del funtore $\xM$...
		 	\begin{prop}
		 		Esiste una Ses.....
		 	\end{prop}
		 	
		 	Tuttavia ci sono dei casi in cui possiamo concludere che $\xM^{s}$ è rappresentabile.
		 	Per enunciarlo, abbiamo bisogno di parlare di un altro fatto.
		 	
		 	Solitamente, leggendo gli articoli, in generale non si fissa $P \in \Q[t]$,
		 	ma è consuetudine fissare degli \textbf{invarianti numerici}.
		 	Adesso assumiamo che $X$ sia uno schema proiettivo \textbf{liscio}
		 	(forse nel caso singolare ce la caviamo con in complessi perfetti...).
		 	In questo caso, possiamo considerare il \textbf{gruppo di Grothendieck numerico} di $X$
		 	\begin{equation}
		 		K_{num}(X) := K_{0}(X)/T\,,
		 	\end{equation}
		 	con il paring $\chi$. Dato $F \in \Coh(X)$, associamo $c(F) \in K_{num}(X)$.
		 	Osserviamo che, per definizione di $P_{H}(F)$, questo dipende dalla polarizzazione
		 	$X$ e dal carattere di Chern: quindi fasci con stesso $\ch$ danno lo stesso polinomio.
		 	Ora
		 	\begin{equation}
		 		\ch : K_{num}()
		 	\end{equation}